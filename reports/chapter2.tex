\chapter{Symmetric instability}
\label{chap:2}
\begin{quote}
    \textit{In real life there is no algebra} --- Audrey Horne
\end{quote}

In this chapter we will explore the theory behind symmetric instability and its subclassifications. In section~\ref{sec:FirstLook}, we will state some properties of symmetric instability, and consider ways in which it can be excited. We will then go onto derive some of the previously stated properties from the Boussinesq equations of motion for a fluid, starting in section~\ref{sec:InviscidInstabilities}, where we consider the classical problem of the stability of an inviscid, symmetric, parallel shear flow. In section~\ref{sec:InviscidCentrifugal} we review how the properties of symmetric instability change in axisymmetric flow. Next, in section~\ref{sec:ViscousSI} we examine the special case of symmetric instability in flows free of vertical shear, performing linear stability analyses and looking at how the results apply to currents such as the North Brazil Current and its rings and the Deep Western Boundary Current. Section~\ref{sec:DrasticSI} examines how symmetric instability changes under the inclusion of the complete Coriolis force.

Sections~\ref{sec:InviscidInstabilities} and~\ref{sec:InviscidCentrifugal} primarily review well studied properties of symmetric instability, whereas the analysis and calculations presented in sections~\ref{sec:ViscousSI} and~\ref{sec:DrasticSI} break new ground, and are based on extracts from a paper published in the Journal of Physical Oceanography which I took a leading role in authoring \citep{Goldsworth2021a}.

\section{A first look at symmetric instability}
\label{sec:FirstLook}
Symmetric instability is a type of sub-mesoscale instability that generates overturning cells in a plane perpendicular to a mean flow. The overturning cells tend to be oriented so that they are parallel to isopycnal surfaces, and so symmetric instability is often said to produce slantwise convection. Although the instability leads to mixing predominantly along isopycnals, it can also induce diapycnal mixing. Secondary shear instabilities further enhance this mixing.

Symmetric instability is intimately linked to the scaler quantity, Ertel potential vorticity. In the ocean this is defined as
\begin{equation}
    \label{eq:PVDefinition}
    Q = (\mathbf{f} + \curl{\mathbf{u}})\cdot\grad b    
\end{equation}
where $\mathbf{f}$ is the planetary vorticity vector, $\mathbf{u}$ is the velocity field in the rotating frame of reference, and $b = -  g \flatfrac{\rho}{\rho_0}$ is the buoyancy field. $g$ is gravitational acceleration and $\rho_0$ is a constant reference density. One can interpret the potential vorticity as either the component of absolute vorticity normal to isopycnal surfaces and scaled by the isopycnal thickness, or the stratification along a vortex tube and scaled by the absolute vorticity of the vortex, with the former being the most common way of understanding the quantity.

In a fluid subjected to no momentum or buoyancy forcing, potential vorticity is materially conserved, i.e.
\begin{equation}
    \frac{DQ}{Dt} = 0 \, .
\end{equation}
This means if we follow an individual fluid parcel around we expect its potential vorticity to remain constant. This is a fundamental material invariant which arises from a particle relabelling symmetry on isopycnal surfaces in an inviscid fluid~\citep{Salmon1998}.

If a fluid is subject to external or dissipative momentum and buoyancy forcing the conservation of potential vorticity  becomes
\begin{equation}
    \frac{DQ}{Dt} = \frac{D\mathbf{\zeta}}{Dt} \cdot  \grad{b} + \mathbf{\zeta} \cdot \frac{D\grad{b}}{Dt} \, .
\end{equation}

A necessary but not sufficient condition for the excitement of symmetric instability is that the vertical component of planetary vorticity and the potential vorticity of a fluid have opposite signs. Mathematically we require that
\begin{equation}
    \label{eq:PVConservation}
    f Q < 0 \, ,
\end{equation}
where $f$ is the vertical component of the planetary vorticity vector, also known as the Coriolis parameter. This criterion is derived in section~\ref{subsec:InstabilityCriterion}. Waters which satisfy this instability criterion are often described as having anomalous potential vorticity.

The overturning cells generated during the excitement of symmetric instability causes the mixing of waters with anomalous potential vorticity. The momentum and stratification of the fluid are reconfigured by this mixing, producing waters with potential  vorticity of approximately zero. This corresponds to the absolute vorticity vector being parallel to isopycnal surfaces.

Consider now a fluid initially stable to symmetric instability. For it to become symmetrically unstable we must make the quantity $fQ$ negative. There are then two distinct ways to generate symmetric instability in such a fluid: the first is to change the sign of the planetary vorticity, and the second is to change the sign of the potential vorticity.

\subsection{Cross equatorial symmetric instability}
At the equator planetary vorticity changes from being positive in the Northern Hemisphere to negative in the Southern Hemisphere. If we consider a fluid parcel originating in one hemisphere, its potential vorticity will typically match the planetary vorticity to ensure symmetric stability. We can than imagine advecting this fluid parcel across the equator. As potential vorticity is materially conserved and planetary vorticity changes sign at the equator, we would expect the symmetric instability criterion to be satisfied in the new hemisphere.

This mechanism does not involve buoyancy or momentum forcing, processes which typically must occur in either a surface or bottom boundary layer. This means symmetric instability resulting from a change in sign of planetary vorticity may occur in the ocean interior and has a distinct character when compared to boundary layer instability. A number of recent observational and theoretical studies have investigated the excitement of symmetric instability resulting from the cross-equatorial change in sign of planetary vorticity, notably CITATIONS. These studies have, however, focused on weak circulations in the ocean interior as opposed to vigorous western boundary currents. In these western boundary currents there is a constant supply of waters with large magnitude, anomalous potential vorticity. These equatorial western boundary currents are rich with temporally varying features, such as mesoscale eddies which interact differently with symmetric instabilities depending on the flow regime.

In a deep western boundary current, to leading order, potential vorticity is conserved as the current is separated from the surface and bottom boundary layers in which mechanical and diabatic forcing can act. This means symmetric instability can only be induced via a change in sign of planetary vorticity, as occurs at the equator. Cross-equatorial symmetric instability is unique in this sense --- it need not be confined to either the surface boundary layer or sloping bottom boundary layer \citep{Haine1998, Wenegrat2020}. A host of recent studies have identified regions where cross-equatorial symmetric instability may be excited \citep{Jakoboski2022, Goldsworth2021, Forryan2021, Zhou2022}.

\subsection{Ekman driven symmetric instability}
\begin{itemize}
    \item Explain mechanism
    \item Discuss frequency and temporal variability of these events.
    \item Discuss what this means in terms of AMOC and seasonality.
    \item Link to OSNAP observations and what they miss out.
\end{itemize}
Ekman driven symmetric instability occurs when potential vorticity is made negative by winds blowing along a geostrophically balanced current. Consider a southwards flowing current in the Northern Hemisphere that is surface intensified. In order to balance the vertical shear, thermal wind balance requires the outcropping of dense waters in the East. A southerly wind-stress blowing along the current will induce a Westwards Ekman transport (in the Northern Hemisphere), which will act to steepen the isopycnals. For a current in thermal wind balance, potential vorticity is given by
\begin{equation}
    Q = (f + \xi) \pdv{b}{z} - \frac{1}{f} (\pdv{b}{x})^2 \, .
\end{equation}
If the stratification is stable, and the planetary vorticity dominates over relative vorticity as is typical when more than a few degrees of the equator, then the first term in the equation will have the same sign as $f$. The quantity $(\partial_x b)^2$ is however positive semi-definite, so the second term will always act to make the potential vorticity more anomalous. As the isopycnals steepen the $\partial_zb$ term decreases and the $\partial_x b$ term increases, meaning eventually, if the isopycnals become sufficiently steep the  potential vorticity can become negative, rendering the flow unstable to symmetric instability. SOMETHING ABOUT Gravitational INSTABILITY.








\section{Inviscid symmetric instability of parallel shear flow}
\label{sec:InviscidInstabilities}
    In this section we will derive some classical results concerning the properties of symmetric instability, by considering the stability of an inviscid parallel shear flow on an $f$-plane. In subsequent sections we will build on this classical analysis to investigate how these properties change in less idealised scenarios. Throughout, we will assume the flow is meridionally symmetric --- i.e. invariant in the meridional direction.

    The Boussinesq equations of motion for a meridionally symmetric, inviscid flow on an $f$-plane are
    \begin{subequations}
    \label{eqs:InviscidBoussinesq}
    \begin{equation}
        % \bigg( \pdv{t} + u \pdv{x} + w \pdv{z} \bigg)
        \frac{Du}{Dt} - fv - \pdv{\phi}{x} = 0 \, ,
    \end{equation}
    \begin{equation}
        \frac{Dv}{Dt} + fu = 0 \, ,
    \end{equation}
    \begin{equation}
        \frac{Dw}{Dt} - \pdv{\phi}{z} - b = 0 \, ,
    \end{equation}
    \begin{equation}
        \pdv{u}{x} + \pdv{w}{z} = 0 \, ,
    \end{equation}
    and
    \begin{equation}
        \frac{Db}{Dt} = 0 \, .
    \end{equation}
    \end{subequations}
    Here, $u$, $v$ \& $w$ are the zonal, meridional \& vertical components of velocity, $f$ is the Coriolis parameter, $\phi$ is the geopotential pressure and $b$ is the buoyancy.

    We now consider a flow which is, initially, purely meridional, with $(u, v, w) = (0, V, 0)$, $\phi = \Phi$ and $b = B$. The flow is steady and in both geostrophic and hydrostatic balance. Mathematically
    \begin{subequations}
    \label{eqs:InviscidAssumptions}
    \begin{equation}
        f V = \pdv{\Phi}{x} \, ,
    \end{equation}
    \begin{equation}
        B = \pdv{\Phi}{z} \, ,
    \end{equation}
    \begin{equation}
        \pdv{B}{z} = N^2 \, ,
    \end{equation}
    and
    \begin{equation}
        \pdv{B}{x} = M^2 \, .
    \end{equation}
    \end{subequations}
    We can then perturb this balanced flow, giving perturbed variables $(u, v, w) = (u', V + v', w')$, $\phi = \Phi + \phi'$ and $b = B + b'$. Substituting these values into the equations of motion and considering terms only of linear order or lower in the perturbed variables, we find that
    \begin{subequations}
    \label{eqs:InviscidPerturbed}
    \begin{equation}
        \label{eq:ZonalMomentum}
        % geostrophic balance
        \pdv{u'}{t} - fv' + \pdv{\phi '}{x} = 0 \, ,
    \end{equation}
    \begin{equation}
        \pdv{v'}{t} + u' \pdv{V}{x} + w' \pdv{V}{z} + fu' = 0 \, ,
    \end{equation}
    \begin{equation}
        \label{eq:VerticalMomentum}
        % Hydrostatic balance
        \pdv{w'}{t} + \pdv{\phi '}{z} - b' = 0 \, ,
    \end{equation}
    \begin{equation}
        \pdv{b'}{t} + w' \pdv{B}{z} + u' \pdv{B}{x} = 0 \, ,
    \end{equation}
    and
    \begin{equation}
        \label{eq:Continuity}
        \pdv{u'}{x} + \pdv{w'}{z} = 0 \, .
    \end{equation}
    \end{subequations}

    From equation~(\ref{eq:Continuity}) we see that we can write $u'$ and $w'$ in terms of an overturning stream-function $\psi$, where $u' = - \partial_z \psi$ and $w' = \partial_x \psi$. We will now aim to obtain an equation of motion for the overturning stream-function. The first step is to differentiate the zonal and vertical momentum equations (equations~\ref{eq:ZonalMomentum} and~\ref{eq:VerticalMomentum}) with respect to the vertical and zonal coordinates, respectively. Subtracting the two equations, differentiating with respect to time, and substituting in our equations for $\partial_t b'$ and $\partial_t v'$ gives
    \begin{equation}
    \label{eq:OriginalHoskins}
    \pdv[2]{}{t} \bigg(\pdv[2]{}{x} + \pdv[2]{}{z}\bigg) \psi + \bigg( N^2 \pdv[2]{}{x} - 2 f \pdv{V}{z} \pdv[2]{}{x}{z} + f \zeta \pdv[2]{}{z} \bigg) \psi = 0 \, ,
    \end{equation}
    where $\zeta = f + \partial_x V$ is the absolute vorticity of the mean flow. This equation is the Hoskins-Ooyama equation \citep{Hoskins1974, Ooyama1966}, and it describes how overturning motions develop within a parallel shear flow.
    
    \subsection{The instability criterion}
    \label{subsec:InstabilityCriterion}
    To examine the stability of solutions to equation~\ref{eq:OriginalHoskins}, \citet{Hoskins1974} proposed a trial solution of the form
    \begin{equation}
        \psi = e^{i k(x\sin\phi + z\cos \phi) + i \omega t} \, ,    
    \end{equation}
    where $\omega$ is the oscillatory frequency of the perturbations, $k$ the wave-number, and $\phi$ the angle the perturbation makes with the horizontal. Substituting this into equation~\ref{eq:OriginalHoskins} we find that
    \begin{equation}
        \label{eq:GrowthRate}
        \omega^2 = \cos^2\phi (N^2 \tau^2 - 2f\partial_zV\tau + f\zeta) \, ,
    \end{equation}
    where $\tau = \tan\phi$. If $\omega^2$ is less than zero then the corresponding overturning mode will grow exponentially. For this to occur, the discriminant of the quadratic expression in the above must be negative, i.e. we need $N^2 f \zeta - (f\partial_z V)^2 < 0$. Note that not \textit{all} modes will be unstable if this is the case as the growth rate still depends on the orientation of the perturbations, $\phi$. It is true, however, that if $N^2 f \zeta - (f\partial_z V)^2 \geq 0$ all the overturning modes will be stable. The instability criteria can be simplified somewhat by noting that as the flow under consideration is in geostrophic balance, $f\partial_zV$ = $\partial_x B$, and so we can rewrite the instability criteria as
    \begin{equation}
        f \bigg[\bigg(f + \pdv{V}{x}\bigg)\pdv{B}{z} - \pdv{V}{z}\pdv{B}{x}\bigg] < 0 \, .
    \end{equation}
    The term in square brackets is recognisable as the potential vorticity of the background flow and so the classical instability criterion of
    \begin{equation}
        f Q < 0
    \end{equation}
    is recovered \citep{Hoskins1974}.

    \subsection{Orientation of the least stable modes}
    From equation~\ref{eq:GrowthRate} we can see that $\omega$ depends on the direction of the perturbation, and so there should be some value of $\phi$ which maximises the growth rate --- i.e. minimises $\omega^2$. The maxima and minima of $\omega^2$ can be found by differentiating equation~\ref{eq:GrowthRate} with respect to $\phi$ and setting the result to zero. The extrema of $\omega^2$ then satisfy
    \begin{equation}
        \sin(2 \phi + \arctan(-\frac{2M^2}{N^2 - f \zeta})) = 0 \, .
    \end{equation}
    We can simplify this somewhat if we assume that $\abs{N^2} \gg \abs{f \zeta}$ and that $\abs{N^2} \gg \abs{M^2}$, to give
    \begin{equation}
        \label{eq:UnstableDirections}
        \sin\big(2(\phi - \phi_{iso})\big) = 0 \, ,
    \end{equation}
    where $\phi_{iso}$ is the angle the isopycnals make to the horizontal. This equation has four solutions, corresponding to $\phi$ being either parallel or perpendicular to $\phi_{iso}$. By evaluating the second order derivative of equation~\ref{eq:GrowthRate} we can determine the character of the extrema. We find that with stable stratification, the modes parallel to the isopycnals give minima of $\omega^2$ whereas the modes perpendicular give maxima. Note that the sign of the maxima and minima may be either positive or negative depending on the parameters of the flow, so these modes may be either entirely stable, entirely unstable or a mix of the two.

    In the presence of instability, assuming the above approximations remain valid, and assuming that the stratification is stable, the fastest growing mode will be that which is parallel to isopycnals. The growth rate is independent of the magnitude of the wave-number; however, in the ocean we typically have $\abs{k \sin\phi_{iso}} \ll \abs{k \cos\phi_{iso}}$ and so the fastest growing modes will tend to have a larger horizontal length scale than in the vertical.

    \subsection{Gravitational \& inertial instability as limiting cases of symmetric instability}
    \citet{Xu1985} explore in detail how, under appropriate coordinate transformations, the equations describing the evolution of symmetric instability are mathematically similar to those describing either inertial instability or gravitational instability (an idea first proposed in~\citet{Hoskins1974}). This leads to a useful corollary, that symmetric instability is in many ways a hybrid of gravitational and inertial instability~\citep[e.g.][]{Haine1998}. To understand this we can perform a simple thought experiment.

    Consider a flow, which is free of vertical shear, and therefore has horizontal isopycnals. In this scenario, we have maxima and minima in $\omega^2$ for perturbations which are either purely horizontal or purely vertical. Evaluating $\omega^2$ we find that the two vertical modes correspond to oscillations (or exponentially growing modes if the stratification is unstable) at the Brunt-V\"ais\"al\"a frequency --- this corresponds to gravitational instability. The two horizontal modes correspond to either inertial oscillations or the excitement of inertial instability (depending on the sign of $f \zeta$).

    In the limit of weakly sloping isopycnals, symmetric instability becomes equivalent to inertial instability occurring along isopycnals. Another (and slightly less intuitive) interpretation is that it is equivalent to gravitational instability along iso-lines of absolute vorticity. Both these interpretations can be traced back the interpretation of potential vorticity as being proportional to either the absolute vorticity normal to an isopycnal surface or the stratification along absolutely vorticity iso-lines.
    
    \subsection{Energetics of the instability}
    The rate of change of the kinetic energy of the perturbations is given by
    \begin{equation}
        \label{eq:TurbulentKE}
        \pdv{E_k}{t} = \frac{1}{2}\frac{\partial}{\partial t} (u'^2 + v'^2 + w'^2) \, .
    \end{equation}
    Substituting in equations~\ref{eq:ZonalMomentum} to~\ref{eq:VerticalMomentum}, we can show that this gives
    \begin{equation}
        \pdv{E_k}{t} = - v' (\mathbf{u}'\cdot \grad) V - (\mathbf{u}'\cdot \grad) \phi ' + w' b' 
    \end{equation}
    The first term on the right-hand side gives the shear production. It corresponds to a transfer of energy from the shear of the mean flow to the turbulent. The term is occasionally split into a ``geostrophic'' and ``lateral'' shear production term. Typically, the excitement of symmetric instability leads to the excitement of overturning cells oriented along isopycnals, with a horizontal length scale much greater than the vertical. If the turbulent motion is along isopycnals then this term can be thought of as an extraction of energy from the along isopycnal shear of the mean flow.
    
    The second term on the right-hand side is the turbulent pressure work term. Using incompressibility of the turbulent flow we can show that
    \begin{equation}
       - (\mathbf{u}'\cdot \grad) \phi ' = - \div{(\mathbf{u'}\phi ')} \, .
    \end{equation}
    This leads to the interpretation of the pressure work term as being a flux. Due to the no normal flow boundary condition, we know this flux must be zero at the boundaries, and all this term does is flux turbulent kinetic energy around the domain, rather than increase the total turbulent kinetic energy.
    
    The final term in equation~\ref{eq:TurbulentKE} is the turbulent buoyancy flux and corresponds to the turbulent flow extracting gravitational potential energy from the advection of buoyancy anomalies.

    \subsection{The classical and energetic definitions of symmetric instability}
    In the above (and in what follows), we have been referring to the overturning instability that occurs when the potential vorticity of a flow has the opposite sign to the planetary vorticity. We refer to this as the \textit{classical} definition of symmetric instability, as proposed by~\citet{Hoskins1974}. Under this definition, gravitational and inertial instability are both special cases of symmetric instability. 
    
    This isn't, however, the only definition of symmetric instability. \citet{Thomas2013} propose a fine-grained taxonomic system, in which different instabilities can be distinguished by their energy source and the source of their anomalous potential vorticity. They define symmetric instability as getting its energy from the geostrophic shear of the mean flow, inertial instability as getting its energy from the lateral shear of the mean flow and gravitational instability as getting its energy from buoyancy fluxes. Under these definitions we can have any of pure inertial, hybrid inertial symmetric, pure symmetric, hybrid symmetric gravitational, and pure gravitational instabilities. We refer to this as the \textit{energetic} definition of symmetric instabilities.

    For the purposes of this work we adopt the classical definition. This is a purely aesthetic choice --- we enjoy the simplicity of the definition and how it preserves the link between symmetric instability, pure gravitational and pure inertial instabilities, potential vorticity, and the extraction of turbulent kinetic energy from along isopycnal shear.

\section{Inviscid symmetric instability of axisymmetric flow}
\label{sec:InviscidCentrifugal}
Parallel shear flow in geostrophic balance is a good first order approximation of a western boundary current's velocity structure. These current systems are often unstable to a myriad of other instabilities, however, and the ``background'' flow is rarely as simple as a parallel shear flow. In the North Brazil Current, for instance, we see the spinning up of the North Brazil Current rings --- huge anticyclonic eddies resulting from barotropic instability --- a few degrees North of the equator. With this in mind, we can see how it might be useful to examine how symmetric instability changes when the background flow is that of an eddy rather than that of a western boundary current.

To first order, we can think of eddies as axisymmetric vortices in cyclogeostrophic balance. The natural coordinate system for such a flow is cylindrical polar, and we can assume symmetry along the azimuthal direction. It is possible to write out the Boussinesq equations of such a flow and linearise them with respect to axisymmetric perturbations. We can then manipulate the equations in order to find an analogue of equation~\ref{eq:OriginalHoskins} for an axisymmetric flow. The resulting equation, however, is nowhere near as elegant, due to the considerably more complex form of $\nabla$ in cylindrical polar coordinates and the presence of centrifugal forces\footnotemark. Thankfully, the revised instability condition, as derived in \citet{Buckingham2021}, \textit{is} just as elegant, and is given by
\begin{equation}
    \bigg(f + \frac{2 V_\phi}{r}\bigg)Q < 0 \, .
\end{equation}
Consider now an anticyclone, meaning the term $\flatfrac{2V_\phi}{r}$ has opposite sign to $f$. If the anticyclone is sufficiently strong, then planetary vorticity will be overwhelmed by the vorticity of the vortex, and flows with potential vorticity  opposite in sign to the planetary vorticity will still be stable to symmetric instability. Thus we see that anticyclonic vortices are more robust to symmetric instability than the classical instability criterion suggests.
\footnotetext{A full derivation of the axisymmetric analogue is given in appendix A of \citet{Buckingham2021}.}

To see this more clearly we can analyse the dispersion relation, given by
\begin{equation}
    \label{eq:GrowthRateCurv}
    \omega^2 = \cos^2\phi \Bigg(N^2 \tau^2 - 2M^2\tau + \bigg(f + \frac{2V_\phi}{r}\bigg)\zeta\Bigg) \, .
\end{equation}
We can see that this is identical to equation~\ref{eq:GrowthRate} which gives the dispersion relation for the case of parallel shear flow, but with the Coriolis parameter $f$ replaced by $f' = f + 2 \flatfrac{V_\phi}{r}$. Assuming anticyclonic relative vorticity, the curvature term will always act to increase $\omega^2$ and so have a stabilising effect on the flow.

A common model for an eddy is that of the Rankine vortex, in which the eddy core is assumed to be in solid body rotation, and is surrounded by an irrotational outer shell. If we consider an anticyclonic eddy, then $f'$ will be more negative in the core than in the outer shell, meaning we expect waters within an anticyclonic eddy core to be more stable to symmetric instability than those in the outer regions.


%\begin{itemize}
%    \item Look at a Rankine vortex type flow
%    \item Explain stabilising and destabilising forces of rotation
%    \item Suggest we expect eddy cores to be more stable than outer shells
%\end{itemize}



\section{Viscous inertial and centrifugal instabilities}
\label{sec:ViscousSI}
\subsection{Viscous inertial instability of parallel shear flow}
We have seen how equation~\ref{eq:OriginalHoskins} describes the evolution of symmetric instability in an inviscid parallel shear flow. However, in real currents, secondary instabilities and dissipation are present. We may then ask how symmetric instability evolves in a viscous current. Unfortunately the shear term of equation~\ref{eq:OriginalHoskins} makes this problem very difficult to solve, especially when we consider a current in which the region of negative potential vorticity is bounded. For currents near to the equator, due to the smallness of planetary vorticity, these shear terms become small and so the problem can be approximated as an inertial instability problem. We can generalise equation~\ref{eq:OriginalHoskins} to viscous flows by replacing the $\partial_t$ operator with $\partial_t - A_r \partial^2_{zz}$, where $A_r$ is the vertical viscosity. If we consider the inertial limit, we then find that the overturning stream-functions of inertially unstable motions satisfy
\begin{equation}
    \label{eq:BarotropicHoskins}
    \bigg(\pdv{}{t} - A_r \pdv[2]{}{z} \bigg)^2 \bigg(\pdv[2]{}{x} + \pdv[2]{}{z}\bigg) \psi + \bigg( N^2 \pdv[2]{}{x} + f \zeta \pdv[2]{}{z} \bigg) \psi = 0 \, .
\end{equation}

We can now try to find solutions to this equation of the form
\begin{equation}
    \psi(x, z, t) = \hat{\psi}(x)e^{i(mz - \omega t)} \,.
\end{equation}
Here we are making the assumption that the instability is bounded in the horizontal (i.e. the zonal extent of negative potential vorticity is finite) and that the vertical scale of the current is much larger than the overturning cells which will develop. Substituting the trial solution into equation~\ref{eq:BarotropicHoskins}, we obtain the following boundary value problem
\begin{equation}
    \frac{(\hat{\omega}^2 - N^2 )}{m^2}\, \dv[2]{\hat{\psi}}{x} + f \zeta \hat{\psi} = \hat{\omega}^2 \hat{\psi}\, ,
    \label{eq:FullSchrodinger}
\end{equation}
where $\hat{\omega} = \omega + i A_r m^2$. Upon appropriate non-dimensionalisation of the coordinates and variables, we find that equation~\ref{eq:FullSchrodinger} is identical to equation 4 of~\citet{Plougonven2009}, who identify it as a Schr\"odinger equation.

The eigenfunctions of the equation are $\hat{\psi}$. They define the horizontal structure of the overturning stream-function. The eigenvalues of the equation are $\hat{\omega}^2$. If the eigenvalue is negative, then it is possible for $\omega$ to be imaginary. If $\omega$ is imaginary, then the overturning circulation may either grow or decay exponentially. It is useful to introduce the quantity $\sigma = \Im(\omega)$, which, if positive, corresponds to the exponential growth rate; if negative, it gives the decay rate. The value of $\sigma$ is maximised for the smallest real eigenvalue of equation~\ref{eq:FullSchrodinger}. For each eigenfunction, there exists a spectrum of vertical wave-numbers, each with a characteristic growth rate, or decay rate and frequency if the mode is stable\footnotemark. The relationship between the growth rate and the vertical wave-number is determined by the eigenvalue, $\hat{\omega}^2$.

\footnotetext{Note that equation~\ref{eq:FullSchrodinger} is Hermitian. This means that its eigenvalues, $\hat{\omega}^2$, are real, and that $\hat{\omega}$ will always be either purely real or purely imaginary. If $\hat{\omega}$ is imaginary then $\sigma$ is also imaginary and so the mode is purely decaying or growing. If $\hat{\omega}$ is real, then the mode will have an oscillatory frequency of $\pm \hat{\omega}$ and decay exponentially over a timescale of $\tau_e = \flatfrac{1}{A_r m^2}$.}

\subsection{Viscous centrifugal instability of axisymmetric flow}
In the above we have formulated a boundary value problem which describes how overturning cells develop in a viscous parallel shear flow, by neglecting the horizontal buoyancy gradient and vertical shear terms, and modifying the temporal derivative of the Hoskins-Ooyama equation (equation~\ref{eq:OriginalHoskins}.) We can do the same for an axisymmetric flow by neglecting these same terms in the generalised Hoskins-Ooyama equation of \citet{Buckingham2021}. This gives the axisymmetric analogue of equation~\ref{eq:BarotropicHoskins} which is

\begin{equation}
    \label{eq:BarotropicBuckingham}
    \bigg(\pdv{}{t} - A_r \pdv[2]{}{z} \bigg)^2 \bigg(\pdv[2]{}{r} + \frac{1}{r}\pdv{}{r} - \frac{1}{r^2} + \pdv[2]{}{z} \bigg) \psi + \bigg( N^2 \bigg(\pdv[2]{}{r} + \frac{1}{r}\pdv{}{r} - \frac{1}{r^2}\bigg) + f' \zeta \pdv[2]{}{z} \bigg) \psi = 0 \, ,
\end{equation}
where $\psi$ now represents a stream-function in cylindrical polar coordinate, and has been defined as $u' = \partial_z \psi$ and $w' = - \flatfrac{\partial_r \psi}{r}$. This equation describes the evolution of centrifugal instability which is the axisymmetric counterpart of inertial instability.

We will now try to find solutions of the form
\begin{equation}
    \psi(r, z, t) = \hat{\psi}(r) e^{i(mz - \omega t)} \, .
\end{equation}
Substituting this trial solution into equation~\ref{eq:BarotropicBuckingham} we find
\begin{equation}
    \frac{(\hat{\omega}^2 - N^2 )}{m^2}\, \Bigg(\dv[2]{\hat{\psi}}{r} + \frac{1}{r} \dv{\hat{\psi}}{r} - \frac{\hat{\psi}}{r^2}\Bigg) + f' \zeta \hat{\psi} = \hat{\omega}^2 \hat{\psi}\, .
    \label{eq:BuckinghamSchrodinger}
\end{equation}
The eigenvalues and eigenfunctions of this equation have the same physical significance as discussed in the previous subsection.



\subsection{Application to the North Brazil Current and its rings}
\label{subsec:NBClsa}
We will now consider a parallel shear flow similar to the North Brazil Current and an axisymmetric flow similar to the North Brazil Current rings. We will solve equations~\ref{eq:FullSchrodinger} and~\ref{eq:BuckinghamSchrodinger} for these flows and examine the structure and growth rate of the overturning cells which are generated by the inertial and centrifugal instabilities.

For the North Brazil Current, we will assume the flow looks something like a Bickley jet, with the meridional velocity of the mean state given by
\begin{equation}
    V(x) = V_0 \Bigg( 1 - \tanh[2](\frac{x - x_{mid}}{\delta_b})\Bigg)
\end{equation}
where $V_0$ is the peak velocity of the jet, $x$ is the across stream coordinate, $x_{mid}$ gives the distance of the peak velocity of the jet from the western boundary and $\delta_b$ is the width of the jet --- this is sketched in figure~\ref{fig:InitialVelocity}a.

We will use the Rankine vortex model to represent the North Brazil Current rings. The azimuthal velocity in this model is given by
\begin{equation}
    V_{\phi}(r) = 
    \begin{cases}
        \flatfrac{V_0 r}{R_0} \, , r < R_0 \\
        \flatfrac{V_0 R_0}{r} \, , r \geq R_0
    \end{cases}
\end{equation}
where $V_0$ is the velocity at the eddy's core-shell interface and $R_0$ is the radius of the core. This is sketched in figure~\ref{fig:InitialVelocity}b. For the Bickley jet we will use $V_0 = 1$ m$\,$s$^{-1}$, and for the Rankine vortex we will use $V_0 = - 1$ m$\,$s$^{-1}$. We will set $x_{mid} = 40$ km, $\delta_b = 30$ km and $R_0 = 100$ km. These values are chosen based on the observations of~\citet{Johns1998} and~\citet{Castelao2011}. The velocity profiles are shown in figure~\ref{fig:InitialVelocity}. The background stratification is set to $N^2 = 2.5 \times 10^{-5}$ s$^{-2}$, which is the mean buoyancy frequency at a depth of between 200 m and 400 m, as estimated from 82 neutral density profiles taken by an Argo float off the coast of Brazil between January 2016 and February 2017~\citep{Argo2019}. The trajectory of the float and the mean neutral density profile are shown in figure~\ref{fig:InitialStratification}. The use of neutral density in calculating the buoyancy frequency means the results of the linear stability analysis will be more applicable to what is seen in the numerical models presented in the following sections, rather than the ocean. This is because the neutral density calculation does not reliably preserve vertical buoyancy gradients~\citep{Eden1999}. Data from a single Argo float was used as it was readily available and provides a plausible estimate of the density structure of the region.
\begin{figure}
    \centering
    \includegraphics[]{../figures/lsa_velocity.pdf}
    \caption{Velocity profiles of the Bickley jet and Rankine vortex used in the linear stability analyses.}
    \label{fig:InitialVelocity}
\end{figure}

\begin{figure}
    \centering
    \includegraphics[]{../figures/argo_trajectory.pdf}
    \caption{Trajectory of Argo profile (a) and mean neutral density along the profile used to calculate $N^2$ (b).}
    \label{fig:InitialStratification}
\end{figure}

\subsubsection{Instability at a fixed latitude}
We will first consider the properties of the eigenvalues and eigenfunctions of equations~\ref{eq:FullSchrodinger} and~\ref{eq:BuckinghamSchrodinger} at a fixed latitude of around $4^\circ$N, meaning $f = 1.01 \times 10^{-5}$~s$^{-1}$. From both of these equations we can see that the growth rate of the instabilities (or equivalently the equations' eigenvalues) depends on the value of the vertical wavelength of the perturbations (or equivalently vertical wave-number) and the viscosity. In figure~\ref{fig:DispersionRelation} we have plotted the growth rate of both the centrifugal and inertial instabilities as a function of vertical wavelength, $\lambda$, and the viscosity. These plots are generated by numerically solving the boundary value problem.

\begin{figure}
    \centering
    \includegraphics[]{../figures/vertical_scale_selection.pdf}
    \caption{Dispersion relations of (a) inertial instability of a Bickley jet and (b) centrifugal instability of a Rankine vortex as a function of vertical wavelength of the mode and viscosity.}
    \label{fig:DispersionRelation}
\end{figure}

We see in figure~\ref{fig:DispersionRelation} that for a given viscosity, there is a vertical wavelength which maximises the growth rate of the instabilities. As the growth is exponential, within a few e-folding timescales, the vertical mode which maximises the growth rate will dominate the structure of the instability --- assuming non-linear effects have not taken hold before this time. Thus, although a discrete set of horizontal modes and a continuous spectrum of vertical modes may be excited, we expect only a single horizontal and vertical mode to dominate the structure of the instability.

%In the northern hemisphere, we would expect to see symmetric instability develop in a region to the right of the jet's center. In the southern hemisphere, we may expect to see symmetric instability between the western boundary and the jet's center. In such a configuration, the western boundary would input anomalous vorticity into the flow, something that this idealized framework is unable to represent. For this reason, and the fact that this study focuses on cross-equatorial flows, we do not apply the linear stability analysis to northward flowing jets in the southern hemisphere.

Below a maximum `critical' viscosity, there is a maximum and minimum vertical wavelength at which unstable modes exist; at higher viscosities, all modes are stable. The maximum wavelength is a result of stratification inhibiting vertical motions. Any mode with a wavelength smaller than the minimum will experience strong viscous damping, rendering the mode stable. The minimum vertical wavelength tends to zero in the inviscid limit. The maximum `critical' viscosity is found when the maximum wavelength allowed by the stratification and the minimum wavelength allowed by viscosity are equal --- i.e. it is related to the Ozimov scale.

We see in figure~\ref{fig:DispersionRelation}b that the range of unstable modes is slightly more limited for the anticyclonic Rankine vortex than for the Bickley jet shown in figure~\ref{fig:DispersionRelation}a. The growth rate is also smaller for the case of the centrifugal instability.

It is not \textit{a priori} clear whether the viscosity which sets properties of the instabilities is a molecular or turbulent viscosity. If one is looking for signs of symmetric instability in a sufficiently coarse ocean model, then it is the turbulent viscosity that will set the vertical length scale. This makes sense as it is the only mixing process the fluid is `aware' of. For real fluids, matters become less clear. \citet{Griffiths2003a} shows that secondary Kelvin-Helmholtz instabilities form as a result of symmetric instability and play a more dominant role in the vertical scale selection than viscosity does. We suggest that when thinking about the properties of symmetric instabilities in the real ocean, it will not be the molecular viscosity which matters but the turbulent viscosity. Secondary Kelvin-Helmholtz processes are more efficient at generating mixing on a molecular scale than symmetric instability and so the rate at which anomalous potential vorticity is removed will depend on their strength as opposed to the strength of the molecular mixing generated directly by the excitement of symmetric instability.

For a given vertical viscosity we can also plot the two-dimensional structure of the overturning that the instability generates, as shown in figure~\ref{fig:2DStructure}. We use a viscosity of $4 \times 10 ^{-4}$~m$^2$\,s$^{-1}$, which corresponds to a typical turbulent viscosity used in high-resolution idealised models that do not resolve Kelvin-Helmholtz instabilities. For the inertial mode, the e-folding timescale is 1.4 days and the wavelength 110 m, whereas for the centrifugal mode the timescale is 3.0 days and the wavelength is 97 m. We see that both instabilities generate a stack of alternating overturning cells. Overlain in the figure is the absolute vorticity (solid line), highlighting that in the case of inertial instability the cells are strongly localized to the region of negative potential vorticity. Despite the localization, the overturning is non-zero outside the region of negative absolute vorticity. This enables the mixing of waters with positive and negative potential vorticity, which over time will create a neutrally stable potential vorticity configuration.

In the case of the centrifugal instability, we see that despite its negative potential vorticity, most of the eddy core is stable and the overturning occurs mainly at the interface between the inner core and outer shell. This acts to mix waters from both within and outside the eddy core, eventually leading to a marginally stable potential vorticity configuration within the flow. It will take some time for waters within the core to be entrained by the instability, meaning the negative potential vorticity within the eddy core may persist for some time.

\begin{figure}
    \centering
    \includegraphics{../figures/overturning_structure.pdf}
    \caption{Normalised overturning stream-functions generated by (a) inertial instability of a Bickley jet and (b) centrifugal instability of a Rankine vortex. Overlain is the absolute vorticity of the flows. $A_r$ is set to $4 \times 10^{-4}$ m$^2\,$s$^{-1}$.}
    \label{fig:2DStructure}
\end{figure}

Seawater has a viscosity of $10^{-6}$ m$^{2}$\,s$^{-1}$, which puts a lower limit on both the vertical wavelength and the growth rate of both inertial and centrifugal instabilities at this latitude. For inertial instability this would give an e-folding timescale of 23 hours and wavelength of 15 m, and for centrifugal instability an e-folding timescale of 1.3 days and a wavelength of 10 m. 

A more rigorous analysis of the solutions to the inertial instability problem described in this section is given in \citet{Plougonven2009}, who consider barotropic zonal shear flows on an $f$-plane. On an $f$-plane there is no physical distinction between meridional and zonal flows so, their findings are broadly applicable, although the vorticity profiles they use differ from those considered here. There also exists a broad literature base on linear stability analyses of symmetric and inertial instabilities on a $\beta$-plane, with a focus on zonal currents~\citep[e.g.][]{Kloosterziel2017, Ribstein2014,Griffiths2003, Hua1997}.

\subsubsection{Instability at varying latitudes}
We have seen above how viscosity, stratification and eddying motions can act to stabilise the excitement of symmetric instability. We will now look at how the growth rates of inertial and centrifugal instability vary as a function of latitude. To do this we will solve the boundary problem above whilst varying $f$, and setting the viscosity to either $10^{-6}$~m$^{2}\,$s$^{-1}$ or $4 \times 10^{-4}$~m$^{2}\,$s$^{-1}$. The former viscosity is chosen as it places an upper limit on the growth rate of the instabilities as a function of latitude, and the latter as it corresponds to the viscosity used in the numerical models we will present in subsequent chapters.

Figure~\ref{fig:SigmaAndLat} shows how the maximum growth rates of inertial and symmetric instability vary as a function of latitude at the two different viscosities. For both inertial and centrifugal instabilities, at both high and low viscosity the growth rate varies approximately parabolically with latitude, peaking some distance north of the equator (for a northward flowing current). The velocity profiles are unstable close to the equator; however, when we get far enough away from it, planetary vorticity starts to overwhelm the relative vorticity of the flows and so the unstable modes disappear. The centrifugal instability has a lower growth rate than the inertial instability at all latitudes, suggesting that not only are anticyclonic eddies able to maintain reserves of negative potential vorticity in their core, but that the instability will spread more slowly than for a parallel shear flow.

This linear stability analysis helps explain the findings of~\citet{Castelao2011} who observe zero vorticity in North Brazil Current rings as far north as $11^\circ$N. One may expect these rings to have become symmetrically unstable much closer to the equator than this and for the marginally stable potential vorticity to have been modified by other processes by the time they reach this latitude. The above analysis, however, suggests that the growth rate of the centrifugal instability is slower than that of inertial instability and that it will take a while for the instability to spread to the rings' cores given that the unstable overturning is strongest at the interface between the inner core and outer shell of the eddies.

\begin{figure}
    \centering
    \includegraphics{../figures/sigma_lat.pdf}
    \caption{Growth rates of inertial (solid line) and centrifugal (dot-dashed line) instabilities as a function of latitude for (a) a low viscosity of $A_r = 1 \times 10^{-6}$ m$^2\,$s$^{-1}$ and (b) a high viscosity of $4 \times 10^{-4}$ m$^2\,$s$^{-1}$.}
    \label{fig:SigmaAndLat}
\end{figure}

\subsection{Application to the Deep Western Boundary Current}
We now apply the linear stability analysis to a parallel shear flow with parameters similar to the southward flowing Deep Western Boundary Current, close to the equator. We use the Bickley jet model with $V_0 = - 0.2$~m$\,$s$^{-1}$, $x_{mid} = 60$~km and $\delta_b = 30$~km. Let us set $f = - 1.91 \times 10^{-6}$~s$^{-1}$ which corresponds to a latitude of approximately $0.75^\circ$S and $N^2 = 1 \times 10^{-6}$~s$^{-2}$ which is a typical value for the deep ocean close to the equator. We choose the $f$ of a lower latitude than before, as we suspect the slower advective timescale of the Deep Western Boundary Current will mean that the instability occurs closer to the equator than in the surface currents.
\begin{figure}
    \centering
    \includegraphics{../figures/dwbc_lsa.pdf}
    \caption{(a) Dispersion relation, and (b) growth rate of inertial instability as a function of latitude, for the Deep Western Boundary Current. Update b to have South not north.}
    \label{fig:lsaDWBC}
\end{figure}

The dispersion relation for the Deep Western Boundary Current is shown in figure~\ref{fig:lsaDWBC}a and tells much the same story as the dispersion relation for the North Brazil Current. Note, however, that for a viscosity of $4 \times 10^{-4}$~m$^2\,$s$^{-1}$ the timescale of the instability is much smaller, at 11 days, and that the size of the overturning cells is much larger at 180 m. For a viscosity of $10^{-6}$~m$^2\,$s$^{-1}$ the timescale is 5 days and the wavelength 26 m. The increased vertical length scales can be understood in terms of the reduced stratification, which means the vertical extent of the overturning cells is less limited. The reduced growth rate is a combination of the reduced magnitude of $f$ and the peak velocity.

The growth rate is almost an order of magnitude smaller for the Deep Western Boundary Current than for the North Brazil Current. The latitude of maximal instability is much closer to the equator and the velocity profile ceases to be unstable much closer to the equator too. This suggests that when looking for signs of symmetric instability in the Deep Western Boundary Current we will likely see it occurring much closer to the equator than it does in the North Brazil Current.

\section{Viscous symmetric instability with the complete Coriolis force}
\label{sec:DrasticSI}
    At the equator, the meridional component of the planetary vorticity vector is a maximum. In studies of geophysical fluid dynamics this term is often neglected, which we describe as making the `traditional Coriolis approximation'. Conversely, retention of the meridional component of the planetary vorticity vector is described as including the `complete', `full', or `non-traditional' components of the Coriolis force~\citep{Stewart2011}. When treating the vertical component of the planetary vorticity as constant, we make the $f$-plane approximation. Close to the equator we may also make the non-traditional $f$-plane approximation in which we treat the meridional component of planetary vorticity as constant instead.

    \citet{Zeitlin2018a} states that ``symmetric instability drastically changes upon inclusion of the full Coriolis force.'' The author goes on to perform a linear stability analysis of a zonally symmetric shear flow on an $f$-plane, taking into account the non-traditional component of the full Coriolis force. They find that the growth rate and spatial structure of unstable modes may be significantly altered by inclusion of the full Coriolis force. Given that on an ordinary $f$-plane there is no difference between a meridional and zonal flow (as the system has no preferred orientation), one may na\"ively expect this finding to apply to the meridional parallel shear flows we have considered in the preceding sections; however, the inclusion of the non-traditional component of the Coriolis force breaks this meridional-zonal symmetry and reintroduces a natural meridional direction. To understand how symmetric instability in meridional flows changes under the complete Coriolis force, we must repeat the linear stability analysis of~\citet{Zeitlin2018a} but for the case of a meridional flow.

    The Boussinesq equations of motion for a meridionally symmetric flow on an $f$-plane, with the complete Coriolis force, and in the limit in which the horizontal length scale is much larger than the vertical length scale, are
    \begin{subequations}
    \begin{equation}
        % \bigg( \pdv{t} + u \pdv{x} + w \pdv{z} \bigg)
        \frac{Du}{Dt} - fv + F_{NT}w - \pdv{\phi}{x} = A_r \pdv[2]{z} u \, ,
    \end{equation}
    \begin{equation}
        \frac{Dv}{Dt} + fu = A_r \pdv[2]{z} v \, ,
    \end{equation}
    \begin{equation}
        \frac{Dw}{Dt} - F_{NT}u - \pdv{\phi}{z} - b = A_r \pdv[2]{z} w \, ,
    \end{equation}
    \begin{equation}
        \pdv{u}{x} + \pdv{w}{z} = 0 \, ,
    \end{equation}
    and
    \begin{equation}
        \frac{Db}{Dt} = \kappa \pdv[2]{z} b \, ,
    \end{equation}
    \end{subequations}
    where $A_r$ is the viscosity, $\kappa$ is the buoyancy diffusion coefficient and $F_{NT}$ is the non-traditional component of the Coriolis parameter.

    Similarly to section~\ref{sec:InviscidInstabilities} we will now consider a flow which is, initially, purely meridional, with $(u, v, w) = (0, V, 0)$, $\phi = \Phi$ and $b = B$. The flow is steady, in geostrophic balance and hydrostatically balanced. Mathematically
    \begin{subequations}
    \begin{equation}
        f V = \pdv{\Phi}{x} \, ,
    \end{equation}
    \begin{equation}
        B = \pdv{\Phi}{z} \, ,
    \end{equation}
    \begin{equation}
        \pdv{B}{z} = N^2 \, ,
    \end{equation}
    \begin{equation}
        \pdv[2]{B}{z} = 0 \, .
    \end{equation}
    \begin{equation}
        \pdv{B}{x} = M^2 \, ,
    \end{equation}
    and
    \begin{equation}
        \pdv[2]{V}{z} = 0 \, .
    \end{equation}
    \end{subequations}
    Again, we will perturb this balanced flow, giving perturbed variables $(u, v, w) = (u', V + v', w')$, $\phi = \Phi + \phi'$ and $b = B + b'$. Substituting these values into the equations of motion and considering terms only of linear order or lower in the perturbed variables, we find that
    \begin{subequations}
    \begin{equation}
        \label{eq:ZonalMomentumViscous}
        % geostrophic balance
        \pdv{u'}{t'} + F_{NT}w' - fv' + \pdv{\phi '}{x} = 0 \, ,
    \end{equation}
    \begin{equation}
        \pdv{v'}{t'} + u' \pdv{V}{x} + w' \pdv{V}{z} + fu' = 0 \, ,
    \end{equation}
    \begin{equation}
        \label{eq:VerticalMomentumViscous}
        % Hydrostatic balance
        \pdv{w'}{t'} + \pdv{\phi '}{z} - b' - F_{NT}u' = 0 \, ,
    \end{equation}
    \begin{equation}
        \pdv{b'}{t} - \kappa \pdv[2]{b'}{z} + w' \pdv{B}{z} + u' \pdv{B}{x} = 0 \, ,
    \end{equation}
    and
    \begin{equation}
    \label{eq:ContinuityViscous}
        \pdv{u'}{x} + \pdv{w'}{z} = 0 \, ,
    \end{equation}
    \end{subequations}
    where the operator $\flatfrac{\partial}{\partial t'}$ is defined as
    \begin{equation}
        \pdv{t'} = \pdv{t} - A_r \pdv[2]{z} \, .
    \end{equation}
    
    From equation~\ref{eq:ContinuityViscous} we see that we can again write $u'$ and $w'$ in terms of an overturning stream-function $\psi$, where $u' = - \partial_z \psi$ and $w' = \partial_x \psi$. We can then attempt to find an equation of motion for the stream-function, analogous to equation~\ref{eq:OriginalHoskins}. As before, we differentiate the zonal and vertical momentum equations (equations~\ref{eq:ZonalMomentumViscous} and~\ref{eq:VerticalMomentumViscous}) with respect to the vertical and zonal coordinates respectively and then subtract them, giving
    \begin{equation}
        \pdv{}{t'} \bigg( \pdv[2]{x} + \pdv[2]{z} \bigg) \psi - f \pdv{v'}{z} + \pdv{b'}{x} + F_{NT} \bigg( \pdv{u'}{x} + \pdv{w'}{z} \bigg) = 0 \, .
    \end{equation}
    We note that the only place $F_{NT}$ enters this equation is in the final term, and that because of the continuity equation (equation~\ref{eq:ContinuityViscous}) this term is equal to zero. If we now differentiate the above with respect to $t'$ and substitute in the equations for $\partial_{t'}b'$ and $\partial_{t'}v'$ we get
    \begin{equation}
        \frac{\partial^2}{\partial t'^2} \bigg(\pdv[2]{}{x} + \pdv[2]{}{z}\bigg) \psi + \bigg( N^2 \pdv[2]{}{x} - 2 f \pdv{V}{z} \pdv[2]{}{x}{z} + f \zeta \pdv[2]{}{z} \bigg) \psi = 0 \, ,
    \end{equation}
    which, apart from the viscous terms, is identical to equation~\ref{eq:OriginalHoskins}\footnotemark. This has absolutely no dependence on $F_{NT}$ at all and from this we conclude that the complete Coriolis force does not alter symmetric instability in meridional, parallel shear flows. This finding is not in any way contradictory to the findings of~\citet{Zeitlin2018a}. The difference arises due to the asymmetry between the purely meridional flow considered here and the zonal flow considered in the aforementioned study.
    \footnotetext{Here we have set $\kappa=A_r$. This is done to simplify the notation and the arguments subsequently made do not lose their generality.}

    It is, in fact, possible to make a more general statement about the types of forces which do not alter the evolution of meridional symmetric instabilities. We can modify the momentum equations (\ref{eq:ZonalMomentum} and~\ref{eq:VerticalMomentum}) with the addition of any irrotational force acting in the $xz$-plane and which satisfies the relationship
    \begin{equation}
        \pdv{\mathcal{F}_x}{z} + \pdv{\mathcal{F}_z}{x} = 0 \, ,
    \end{equation}
    where $\mathcal{F}_x$ and $\mathcal{F}_z$ are the zonal and meridional components of the force, respectively. This can be understood in the framework of~\citet{Marshall2011} as follows --- an irrotational force is divergent and so will project on to the pressure gradient terms of the momentum equation. A rotational force is non-divergent and so projects entirely onto the acceleration term. An irrotational (divergent) force is not able to alter the acceleration term. In the system described above, the inclusion of the complete Coriolis force may alter the pressure field but not the motions within the $xz$-plane due to its irrotational nature.

    Although the crossing of the equator is a meridional phenomenon, western boundary currents, such as the North Brazil Current or the deep western boundary current, will be oriented at some angle to a meridian, and so have both zonal and meridional components of velocity. In the zonal limit, symmetric instability can change drastically with the inclusion of the full Coriolis force, whereas in the meridional limit there is no change at all. For a realistic (not purely meridional) western boundary current crossing the equator, the structure of symmetric instability may therefore have some dependence on the complete Coriolis force.
    
    The relative importance of non-traditional effects depends on the direction of the current relative to the meridional direction. For a current oriented at an angle $\theta$ to the meridional direction, the findings of~\citet{Zeitlin2018a} apply but with the value of $F_{NT}$ scaling with $\sin \theta$. \citet{Zeitlin2018a} defines a ``non-traditionality'' parameter and, using the findings of our work, we can generalise it to flows with a meridional component, giving
    \begin{equation}
        \gamma = \frac{\cot\phi\,\sin \theta\,H}{L} \, ,
    \end{equation}
    where $\gamma$ is the  ``non-traditionality parameter'' and $\phi$ is the latitude. Non-traditional effects are important when $\gamma \sim 1$. Close to the equator, the coast of Brazil forms an angle of $\theta \sim 60^\circ$ to the meridian. From work that will be discussed in subsequent chapters, we suspect symmetric instability in the North Brazil Current to occur at a latitude of around $4^\circ$N, and the current has $H\sim100$ m and $L\sim30$ km giving $\gamma \sim 0.04$. This suggests that non-traditional effects are unlikely to be hugely important in the North Brazil Current. For the deep western boundary current we expect $\phi \sim 1^\circ$N, $H\sim 1,000$ m and $L \sim 50 km$. This gives $\gamma \sim 1$ suggesting non-traditional effects may be important here. Finally, the Irminger current sits at around $60^\circ$N, forms an angle of $\theta \sim 30^\circ$ to the meridian, has $H \sim 100 m$ and $L \sim 30 km$. Due to its high latitude we suspect non-traditional effects will be very weak, and this is reflected in a value of $\gamma \sim 10^{-3}$.

\section{Summary}
We started this chapter by deriving the Hoskins-Ooyama equation for a parallel shear flow, and using this to derive the classical instability criterion for symmetric instability. We then went on to use the equation to explore some key characteristics of symmetric instability. We showed that symmetric instability generates overturning cells parallel to isopycnals and saw how gravitational and inertial instability can be seen as limiting cases of symmetric instability. We saw how the instability leads to the extraction of energy from mean flows via turbulent buoyancy fluxes and by reducing the along isopycnal shear. This lead us on to discussing the two different definitions of symmetric instability currently present in the literature --- namely the \textit{energetic} definition which distinguishes between lateral and geostrophic shear, and the \textit{classical} definition which sees inertial, symmetric and gravitational instabilities as different manifestations of the same phenomena. We next looked briefly at how symmetric instability manifests itself in an axisymmetric flow, reviewing the work of~\citet{Buckingham2021}.

Having examined these classical results, we then go on to investigate how parallel shear flows and axisymmetric flows, similar to the Deep Western Boundary Current, the North Brazil Current and its rings behave in the vicinity of the equator. We examined the structure and growth rate of symmetrically unstable modes by numerically solving the boundary values problems which govern their evolution and begun to get a feel as to how symmetric instability behaves in these tropical circulations.

Finally, we looked at the effect of the complete Coriolis force on parallel shear flows near the equator. We found that meridional currents are unaffected by the force, and showed that any irrotational force will leave the dynamics of the instability unchanged. We showed that the complete Coriolis force will have almost no impact on the Irminger Current and North Brazil Current, but its effects may become noticeable in the Deep Western Boundary Current due to the larger vertical length scale of this flow.
