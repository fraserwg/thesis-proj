\chapter{Symmetric instability}
\begin{quote}
    \textit{In real life there is no algebra} --- Audrey Horne
\end{quote}

This chapter examines symmetric instability and its sub-classifications through several linear instability analyses. This work builds extensively on sections of~citep{Goldsworth2021}. Sections XXXXXX are new.

\section{Inviscid symmetric instability of parallel shear flow}
\label{sec:InviscidInstabilities}
    In this section we will derive some of the classical results concerning the properties of symmetric instability, by considering the stability of an inviscid parallel shear flow on an $f$-plane. In subsequent sections we will build on this classical analysis to investigate how these properties change in less idelised scenarios.

    The Boussinesq equations of motion for a meridionally symmetric, inviscid flow on an $f$-plane are
    \begin{subequations}
    \label{eqs:InviscidBoussinesq}
    \begin{equation}
        % \bigg( \pdv{t} + u \pdv{x} + w \pdv{z} \bigg)
        \frac{Du}{Dt} - fv - \pdv{\phi}{x} = 0 \, ,
    \end{equation}
    \begin{equation}
        \frac{Dv}{Dt} + fu = 0 \, ,
    \end{equation}
    \begin{equation}
        \frac{Dw}{Dt} - \pdv{\phi}{z} - b = 0 \, ,
    \end{equation}
    \begin{equation}
        \pdv{u}{x} + \pdv{w}{z} = 0 \, ,
    \end{equation}
    and
    \begin{equation}
        \frac{Db}{Dt} = 0 \, .
    \end{equation}
    \end{subequations}
Here, $u$, $v$ \& $w$ are the zonal, meridional \& vertical components of velocity, $f$ is the Coriolis parameter, $\phi$ is the geopotential pressure and $b$ is the buoyancy.

    We now consider a flow which is, initially, purely meridional, with $(u, v, w) = (0, V, 0)$, $\phi = \Phi$ and $b = B$. The flow is steady and both geostrophic and hydrostatic balance. Mathematically
    \begin{subequations}
    \label{eqs:InviscidAssumptions}
    \begin{equation}
        f V = \pdv{\Phi}{x} \, ,
    \end{equation}
    \begin{equation}
        B = \pdv{\Phi}{z} \, ,
    \end{equation}
    \begin{equation}
        \pdv{B}{z} = N^2 \, ,
    \end{equation}
    and
    \begin{equation}
        \pdv{B}{x} = M^2 \, .
    \end{equation}
    \end{subequations}
    We can then perturb this balanced flow, giving perturbed variables $(u, v, w) = (u', V + v', w')$, $\phi = \Phi + \phi'$ and $b = B + b'$. Substituting these values into the equations of motion and considering terms only of linear order or lower in the perturbed variables, we find that
    \begin{subequations}
    \label{eqs:InviscidPerturbed}
    \begin{equation}
        \label{eq:ZonalMomentum}
        % geostrophic balance
        \pdv{u'}{t} - fv' + \pdv{\phi '}{x} = 0 \, ,
    \end{equation}
    \begin{equation}
        \pdv{v'}{t} + u' \pdv{V}{x} + w' \pdv{V}{z} + fu' = 0 \, ,
    \end{equation}
    \begin{equation}
        \label{eq:VerticalMomentum}
        % Hydrostatic balance
        \pdv{w'}{t} + \pdv{\phi '}{z} - b' = 0 \, ,
    \end{equation}
    \begin{equation}
        \pdv{b'}{t} + w' \pdv{B}{z} + u' \pdv{B}{x} = 0 \, ,
    \end{equation}
    and
    \begin{equation}
        \label{eq:Continuity}
        \pdv{u'}{x} + \pdv{w'}{z} = 0 \, ,
    \end{equation}
    \end{subequations}

    From equation~(\ref{eq:Continuity}) we see that we can write $u'$ and $w'$ in terms of an overturning streamfunction $\psi$, where $u' = - \partial_z \psi$ and $w' = \partial_x \psi$. We will now aim to obtain an equation of motion for the overturning streamfunction. The first step is to differentiate the horizontal and vertical momentum equations (equations~\ref{eq:ZonalMomentum} and~\ref{eq:VerticalMomentum}) with respect to the vertical and zonal coordinates respectively. Subtracting the two equations, differentiating with respect to time, and substituting in our equations for $\partial_t b'$ and $\partial_t v'$ gives
    \begin{equation}
    \label{eq:OriginalHoskins}
    \pdv[2]{}{t} \bigg(\pdv[2]{}{x} + \pdv[2]{}{z}\bigg) \psi + \bigg( N^2 \pdv[2]{}{x} - 2 f \pdv{V}{z} \pdv[2]{}{x}{z} + f \zeta \pdv[2]{}{z} \bigg) \psi = 0 \, .
    \end{equation}
    This equation describes how overturning motions develop within a parallel shear flow, and their dependence on the background flow and its stratification. The equation forms the starting point of~\citet{Hoskins1974} and can be used to derive many interesting results relating to symmetric instability~\citep[see also][]{Ooyama1966}.
    
    \subsection{The instability criterion}
    In order to derive an instability criterion, \citet{Hoskins1974} uses a trial solution to equation~\ref{eq:OriginalHoskins} of the form $\psi = e^{i k(x\sin\phi + z\cos \phi) + \omega t}$, where $\omega$ is the oscillatory frequency of the perturbations, $k$ the wavenumber, and $\phi$ the angle the perturbation makes with the horizontal. Substituting this into equation~\ref{eq:OriginalHoskins} we find that
    \begin{equation}
        \label{eq:GrowthRate}
        \omega^2 = \cos^2\phi (N^2 \tau^2 - 2f\partial_zV\tau + f\zeta) \, ,
    \end{equation}
    where $\tau = \tan\phi$. If $\omega^2$ is less than zero then the corresponding overturning mode will grow exponentially. For this to occur, the discriminant of the quadratic expression in the above must be negative, i.e. we need $N^2 f \zeta - (f\partial_z V)^2 < 0$. Note that not \textit{all} modes will be unstable if this is the case as the growth rate still depends on the orientation of the perturbations, $\phi$. It is true, however, that if $N^2 f \zeta - (f\partial_z V)^2 \geq 0$ all of the overturning modes will be stable. The instability criteria can be simplified somewhat by noting that as the flow under consideration is in geostrophic balance, $f\partial_zV$ = $\partial_x B$, and so we can wewrite the instability criteria as
    \begin{equation}
        f \bigg[\bigg(f + \pdv{V}{x}\bigg)\pdv{B}{z} - \pdv{V}{z}\pdv{B}{x}\bigg] < 0 \, .
    \end{equation}
    The term in square brackets is recognisable as the potential vorticity of the background flow and so the classical instability criterion of
    \begin{equation}
        f Q < 0
    \end{equation}
    is recovered.

    \subsection{Orientation of the least stable modes}
    From equation~\ref{eq:GrowthRate} we can see that $\omega$ depends on the direction of the perturbation, and so there should be some value of $\phi$ which maximises the growth rate --- i.e. minimses $\omega^2$. The maxima and minima of $\omega^2$ can be found by differentiating eqaution~\ref{eq:GrowthRate} with respect to $\phi$ and setting the result to zero. The extrema of $\omega^2$ then satisfy
    \begin{equation}
        \sin(2 \phi + \arctan(-\frac{2M^2}{N^2 - f \zeta})) = 0 \, .
    \end{equation}
    We can simplify this somewhat if we assume that $\abs{N^2} \gg \abs{f \zeta}$ and that $\abs{N^2} \gg \abs{M^2}$, to give
    \begin{equation}
        \label{eq:UnstableDirections}
        \sin\big(2(\phi - \phi_{iso})\big) = 0 \, ,
    \end{equation}
    where $\phi_{iso}$ is the angle the isopycnals make to the horizontal. This equation has four solutions, corresponding to $\phi$ being either parallel or perpendicular to $\phi_{iso}$. By evaluating the second order derivative of equation~\ref{eq:GrowthRate} we can determine the character of the extrema. We find that with stable stratification, the modes parallel to the isopycnals give minima of $\omega^2$ whereas the modes perpendicular give maxima. Note that the sign of the maxima and minima may be either positive or negative depending on the parameters of the flow, so these modes may be either entirely stable, entirely unstable or a mix of the two.

    In the presence of instability, assuming the isopycnal slope isn't ``too'' steep, and the stratification is stable, the fastest growing mode will be that which is parallel to isopycnals. The growth rate is independent of the magnitude of the wavenumber, however, in the ocean we typically have $\abs{k \sin\phi_{iso}} \ll \abs{k \cos\phi_{iso}}$ and so the the fastest growing modes will tend to have a larger horizontal length scale than in the vertical.

    \subsection{Gravitational \& inertial instability as limiting cases of symmetric instability}
    \citet{Xu1985} explore in detail, how under appropriate coordinate transformations the equations describing the evolution of symmetric instability are mathematically similar to those describing either inertial instability or gravitational instability (an idea first proposed in~\citet{Hoskins1974}). This leads to a useful corallory, that symmetric instability is in many ways a hybrid of gravitational and inertial instability~\citep[e.g.][]{Haine1998}. To understand this we can perform a simple thought experiment.

    Consider a flow, which is free of vertical shear, and therefore has horizontal isopycnals. In this scenario, we have maxima and minima in $\omega^2$ for perturbations which are either purely horizontal or purely vertical. Evaluating $\omega^2$ we find that the two vertical modes correspond to oscillations (or exponentially growing modes if the stratification is unstable) at the Brunt-V\"ais\"al\"a frequency --- this corresponds to gravitational instability. The two horizontal modes correspond to either inertial oscillations or the excitement of inertial instability (depending on the sign of $f \zeta$).

    In the limit of weakly sloping isopycnals, symmetric instability becomes equivalent to inertial instability occuring along isopycnals. Another (and slightly less intuitive) interpretation is that it is equivalent to gravitational instability along isolines of absolute vorticity. Both these interpretations can be traced back the interpretation of potential vorticity as either the absolute vorticity normal to an isopycnal surface or the stratification along absolutely vorticity isolines.
    
    \subsection{Energetics of the instability}
    The rate of change of the kinetic energy of the perturbations is given by
    \begin{equation}
        \label{eq:TurbulentKE}
        \pdv{E_k}{t} = \frac{1}{2}\frac{\partial}{\partial t} (u'^2 + v'^2 + w'^2) \, .
    \end{equation}
    Substituting in equations~\ref{eq:ZonalMomentum} to~\ref{eq:VerticalMomentum}, we can show that this gives
    \begin{equation}
        \pdv{E_k}{t} = - v' (\mathbf{u}'\cdot \grad) V - (\mathbf{u}'\cdot \grad) \phi ' + w' b' 
    \end{equation}
    The first term on the right hand side gives the shear production. It corresponds to a transfer of energy from the shear of the mean flow to the turbulent. The term is occasionally split into a ``geostrophic'' and ``lateral'' shear production term. Typically the excitement of symmetric instability leads to to the excitement of overturning cells oriented along isopycnals, with a horizontal length scale much greater than the vertical. If the turbulent motion is along isopycnals then this term can be thought of as an extraction of energy from the along isopycnal shear of the mean flow.
    
    The second term on the right hand side is the turbulent pressure work term. Using incompressibility of the turbulent flow we can show that
    \begin{equation}
       - (\mathbf{u}'\cdot \grad) \phi ' = - \div{(\mathbf{u'}\phi ')} \, .
    \end{equation}
    This leads to the interpretation of the pressure work term as being a flux. Due to the no normal flow boundary condition, we know this flux must be zero at the boundaries, and all this term does is advect turbulent kinetic energy around the domain, rather than increase the total turbulent kinetic energy. As such it's relatively boring and we won't consider it any further.
    
    The final term in equation~\ref{eq:TurbulentKE} is the turbulent buoyancy flux and corresponds to the turbulent flow extracting gravitational potential energy from the advection of buoyancy anomalies.

    \subsection{The classical and energetic definitions of symmetric instability}
    In the above (and in what follows), we have been referring to the overturning instability that occurs when the potential vorticity of a flow has the opposite sign to the planetary vorticity. We refer to this as the \textit{classical} definition of symmetric instability, as proposed by~\citet{Hoskins1974}. Under this definition, gravitational and inertial instability are both special cases of symmetric instability. 
    
    This isn't, however, the only definition of symmetric instability. \citet{Thomas2013} propose a more fine-grained approach to the taxonomy of these instabilities, in which they can be distinguished by their energy source and the source of their anomalous potential vorticity. They define symmetric instability as getting its energy from the geostrophic shear of the mean flow, inertial instability as getting its energy from the lateral shear of the mean flow and gravitatational instability as getting its energy from buoyancy fluxes. Under these definitions we can have any of pure inertial, hybrid inertial symmetric, pure symmetric, hybrid symmetric gravitational, and pure gravitational instabilities. We refer to this as the \textit{energetic} defintion of symmetric instabilities.

    For the purposes of this work we adopt the classical definition. This is a purely aesthetic choice --- we enjoy the simplicity of the definition and how it preserves the link between symmetric instability, pure gravitational and pure inertial instabilities, potential vorticity, and the extraction of turbulent kinetic energy from along isopycnal shear.

\section{Inviscid symmetric instability of axisymmetric flow}
Parallel shear flow in geostrophic balance is a good first order approximation of a western boundary current's velocity structure. These current systems are often unstable to a myriad of other instabilities, however, and the ``background'' flow is rarely as simple as a parallel shear flow. In the North Brazil Current for instance, we see the spinning up of the North Brazil Current rings --- huge anticyclonic eddies resulting from barotropic instability --- a few degrees North of the equator. With this in mind, we can see how it might be useful to examine how symmetric instability changes when the background flow is that of an eddy rather than that of a western boundary current.

To first order, we can think of eddies as axisymmetric vortices in cyclo-geostrophic balance. The natural coordinate system for such a flow is cylindrical polars, and we can assume symmetry along the azimuthal direction. It is possible to write out the Boussinesq equations of such a flow and linearise them with respect to axisymmetric perturbations. We can then manipulate the equations in order to find an analogue of equation~\ref{eq:OriginalHoskins} for an axisymmetric flow. The resulting equation, however, is nowhere near as elegant, due to the considerably more complex form of $\nabla$ in cyclindrical polar coordinates and the presence of centrifugal forces\footnotemark. Thankfully, the revised instability condition, as derived in~\citet{Buckingham2021}, \textit{is} just as elegant, and is given by
\begin{equation}
    \bigg(f + \frac{2 V_\phi}{r}\bigg)Q < 0 \, .
\end{equation}
Consider now an anticyclone, meaning the term $\flatfrac{2V_\phi}{r}$ has opposite sign to $f$. If the anticyclone is sufficiently strong, then planetary vorticity will be overwhelmed by the vorticity of the vortex, and flows with potential vorticity  opposite in sign to the planetary vorticity will still be stable to symmetric instability. Thus we see that an anticyclonic vortexes are more robust to symmetric instability than the classical instability criterion suggests.
\footnotetext{A full derivation of the axisymmetric analogue is given in appendix A of \citet{Buckingham2021}.}

To see this more clearly we can analyse the dispersion relation, given by
\begin{equation}
    \label{eq:GrowthRateCurv}
    \omega^2 = \cos^2\phi \Bigg(N^2 \tau^2 - 2\bigg(f + \frac{2V_\phi}{r}\bigg)\partial_zV\tau + \bigg(f + \frac{2V_\phi}{r}\bigg)\zeta\Bigg) \, .
\end{equation}
We can see that this is identical to equation~\ref{eq:GrowthRate} which gives the dispersion relation for the case of parallel shear flow, but with the Coriolis parameter $f$ replaced by $f + 2 \flatfrac{V_\phi}{r}$. Here we can see how the curvature doesn't even need to overwhelm the planetary vorticity to reduce the growth rate of symmetric instability.

%% Refine the above paragraph


\begin{itemize}
    \item Explain parallel shear flow like a western boundary current and axisymmetric flow like an eddy
    \item Derive the instability criteria
    \item Look at a Rankine vortex type flow
    \item Explain stabilising and destabilising forces of rotation
    \item Suggest we expect eddy cores to be more stable than outer shells
\end{itemize}



\section{Viscous inertial instability}
citet{Hoskins1974} shows that an inviscid meridional jet, initially in thermal wind balance and symmetric about the meridional axis, may be linearly unstable and produce overturning in the $x$-$z$ plane. The overturning can be represented as a streamfunction, $\psi$, where the zonal and vertical velocities are given by $u = - \partial_z \psi$ and $w = \partial_x \psi$ respectively. citet{Hoskins1974} shows that the streamfunction, to terms linear in $\psi$, satisfies the partial differential equation
\begin{equation}
    \pdv[2]{}{t} \bigg(\pdv[2]{}{x} + \pdv[2]{}{z}\bigg) \psi + \bigg( N^2 \pdv[2]{}{x} - 2 f \pdv{V}{z} \pdv[2]{}{x}{z} + f \zeta \pdv[2]{}{z} \bigg) \psi = 0 \, .
\end{equation}
Here $N$ is the buoyancy frequency which is assumed to be constant, $f$ is the planetary vorticity, $V$ is the basic state meridional velocity, and $\zeta$ is the absolute vorticity of the basic state about the vertical.
The equation is easily generalized to flows with a harmonic vertical viscosity, by replacing $\partial_t$ with $\partial_t - A_r \partial^2_{zz}$, where $A_r$ is the vertical viscosity. 

Solving for $\psi$ in the viscous case is a difficult problem; however, much can be gained by considering a basic flow that is both barotropic and meridional. Thus the term proportional to the vertical shear of the meridional flow is set to zero. The resulting equation is
\begin{equation}
    \label{eq:BarotropicHoskins}
    \bigg(\pdv{}{t} - A_r \pdv[2]{}{z} \bigg)^2 \bigg(\pdv[2]{}{x} + \pdv[2]{}{z}\bigg) \psi + \bigg( N^2 \pdv[2]{}{x} + f \zeta \pdv[2]{}{z} \bigg) \psi = 0 \, ,  
\end{equation}
which, strictly speaking, describes the evolution of an inertial instability due to the flow being free of vertical shear.

We can now try to find solutions of the form $\psi(x, z, t) = \hat{\psi}(x)e^{i(mz - \omega t)}$. Substituting this into~(\ref{eq:BarotropicHoskins}), we obtain the following boundary value problem:
\begin{equation}
    \frac{(\hat{\omega}^2 - N^2 )}{m^2}\, \dv[2]{\hat{\psi}}{x} + f \zeta \hat{\psi} = \hat{\omega}^2 \hat{\psi}\, ,
    \label{eq:FullSchrodinger}
\end{equation}
where $\hat{\omega} = \omega + i A_r m^2$. Upon appropriate non-dimensionalisation of the coordinates and variables, (\ref{eq:FullSchrodinger}) is identical to equation 4 of~citet{Plougonven2009}, who identify it as a Schr\"odinger equation.


From the work of~\citet{Hoskins1974}, it is known that in the inviscid limit, $\hat{\omega}^2 \sim f^2$. For oceanic western boundary flows, typically $N^2 \gg f^2$. This means we can make the approximation $N^2 - \hat{\omega}^2 \approx N^2$. \citet{Plougonven2009} show that this is equivalent to making the hydrostatic approximation and that, for a flow similar to the one considered here, there is a negligible effect on the solutions. After making the hydrostatic approximation, we are then left with
\begin{equation}
    -\frac{N^2}{m^2} \dv[2]{\hat{\psi}}{x} + f\zeta\hat{\psi} \approx \hat{\omega}^2 \hat{\psi} \, .
    \label{eq:LSASchro}
\end{equation}

The eigenfunctions of the equation are $\hat{\psi}$. They define the horizontal structure of the overturning streamfunction. The eigenvalues of the equation are $\hat{\omega}^2$. If the eigenvalue is negative, then it is possible for $\omega$ to be imaginary. If $\omega$ is imaginary, then the overturning circulation may either grow or decay exponentially. It is useful to introduce the quantity $\sigma = \Im(\omega)$, which, if positive, corresponds to the exponential growth rate; if negative it gives the decay rate. The value of $\sigma$ is maximized for the smallest real eigenvalue of equation~\ref{eq:LSASchro}. For each eigenfunction, there exists a spectrum of vertical wave-numbers, each with a characteristic growth rate (or frequency if stable). The relationship between the growth rate and the vertical wavenumber is determined by the eigenvalue, $\hat{\omega}$.

To calculate the eigenfunctions and eigenvalues of the equation, we must first specify a buoyancy frequency and velocity profile, from which vorticity can be calculated. As we are interested in western boundary currents, we will consider an idealized meridional flow: the barotropic Bickley jet. The velocity of the jet can be expressed as
\begin{equation}
    V(x) = V_0 \Bigg( 1 - \tanh[2](\frac{x - x_{mid}}{\delta_b})\Bigg)
\end{equation}
where $V_0$ is the peak velocity of the jet, $x$ is the across stream coordinate, $x_{mid}$ gives the distance of the peak velocity of the jet from the western boundary and $\delta_b$ is the width of the jet. The jet is symmetric in the along stream direction. The jet parameters are set as follows: $V_0 = 0.87$~m\,s$^{-1}$, $x_{mid} = 40$ km and $\delta_b = 30$ km. The parameters are chosen to be similar to those used in the two-dimensional and three-dimensional numerical models described in sections~\ref{sec:2DNumericalModels} and~\ref{sec:3DNumericalModels} at a depth of 200 m. These parameters are, in turn, loosely based on what is seen in the North Brazil Current~citep{Johns1998}. The velocity profile used is shown in figure~\ref{fig:BickleyJet}. In the northern hemisphere, we would expect to see symmetric instability develop in a region to the right of the jet's center. In the southern hemisphere, we may expect to see symmetric instability between the western boundary and the jet's center. In such a configuration, the western boundary would input anomalous vorticity into the flow, something that this idealized framework is unable to represent. For this reason, and the fact that this study focuses on cross-equatorial flows, we do not apply the linear stability analysis to northward flowing jets in the southern hemisphere.

The buoyancy frequency is set to a value of 5$\times$10\textsuperscript{-3}~s\textsuperscript{-1}, which is the mean buoyancy frequency at a depth of between 200 m and 400 m, as estimated from 82 neutral density density profiles taken by an Argo float off the coast of Brazil between January 2016 and February 2017~citep{Argo2019}. The trajectory of the float and the mean neutral density profile are shown in figure~\ref{fig:InitialStratification}. The use of neutral density in calculating the buoyancy frequency means the results of the linear stability analysis will be more applicable to what is seen in the models presented in the following sections rather than the ocean. This is because the neutral density calculation does not reliably preserve vertical buoyancy gradients~citep{Eden1999}. Data from a single Argo float was used as it was readily available and provides a plausible estimate of the density structure of the region. The value of $f$ is set to $1.01 \times 10^{-5}$~s$^{-1}$, corresponding to a latitude of approximately 4$^\circ$N.

For each eigenfunction, which physically corresponds to the horizontal structure of the overturning cell, we can plot the growth rate, $\sigma$, as a function of vertical wavenumber, $m$, and vertical viscosity, $A_r$. This is done in figure~\ref{fig:DispersionRelation}. We find that, for a given vertical viscosity, there is a value of the vertical wavenumber which maximizes the growth rate. As the growth is exponential, within a few e-folding timescales, the vertical mode which maximizes $\sigma$ will dominate the structure of the instability --- assuming non-linear effects have not taken hold before this time. Thus, although a discrete set of horizontal modes and a continuous spectrum of vertical modes may be excited, we may expect a single horizontal and vertical mode to dominate the structure of the instability. However, we can only verify this expectation with the use of a numerical model which takes into account the non-linearities neglected here.

Below a maximum `critical' viscosity, there is a maximum and minimum vertical wavelength at which unstable modes exist; at higher viscosities, all modes are stable. The maximum wavelength is a result of stratification inhibiting vertical motions. Any mode with a wavelength smaller than the minimum will experience strong viscous damping, rendering the mode stable. The minimum vertical wavelength tends to zero in the inviscid limit. The maximum `critical' viscosity is found when the maximum wavelength allowed by the stratification and the minimum wavelength allowed by viscosity are equal.

It is not \textit{a priori} clear whether the viscosity we are interested in should be a molecular or turbulent viscosity. If one is looking for signs of symmetric instability in a sufficiently coarse ocean model, then it is the turbulent viscosity that will set the vertical length scale. This makes sense as it is the only viscosity the fluid is `aware' of. For real fluids, matters become more problematic.~citet{Griffiths2003a} suggests that secondary Kelvin-Helmholtz instabilities form as a result of symmetric instability and play a more dominant role in the vertical scale selection than does viscosity. This means that the findings of this linear-stability analysis apply to the results of numerical models which fail to resolve these secondary instabilities, but the relation to what might be observed in meridional western boundary flows in the ocean is more ambiguous.


For a given vertical viscosity we can also plot the two-dimensional structure of the overturning that the instability generates, as shown in figure~\ref{fig:2DStructure} for a viscosity of $4 \times 10 ^{-4}$ m$^2$\,s$^{-1}$ (this viscosity was chosen as it corresponds to the value used in the numerical models presented in sections 3 \& 4). We see that the instabilities generate a stack of alternating overturning cells. Overlain in the figure is the absolute vorticity (solid line), highlighting that the cells are stongly localized to the region of negative potential vorticity. Despite the localization, the overturning is non-zero outside of the region of negative absolute vorticity. This enables the mixing of waters with positive and negative PV, which over time will create a neutrally stable PV configuration.

The mode shown in figure~\ref{fig:2DStructure} has an e-folding timescale of around 2 days. Seawater has a viscosity of $10^{-6}$ m$^{2}$\,s$^{-1}$, which would lead to overturning cells with a much smaller vertical wavelength of 15 m and a smaller e-folding time scale of approximately 1 day. That the timescale is decreased slightly when using a realistic viscosity suggests that the instability will be at least as efficient at neutralizing anomalous PV in the ocean. Differences between the vertical lengthscales seen here and in other works~citep[e.g.][]{Taylor2009} can be understood in terms of the dependence on vertical viscosity, and the artificially high ‘eddy viscosity’ used here.

A more rigorous analysis of the solutions to the boundary value problem described in this section is given in~citet{Plougonven2009}, who consider barotropic zonal shear flows on an $f$-plane. On an $f$-plane there is no physical distinction between meridional and zonal flows so their findings are broadly applicable, although the vorticity profiles they use differ from those considered here. There also exists a broad literature base on linear stability analyses of symmetric and inertial instabilities on a $\beta$-plane, with a focus on zonal currents~\citep[e.g.][]{Kloosterziel2017, Ribstein2014,Griffiths2003, Hua1997}.

\section{Viscous symmetric instability with the complete Coriolis force}

\label{sec:DrasticSI}
    \citet{Zeitlin2018a} states that ``symmetric instability drastically changes upon inclusion of the full Coriolis force.'' The author goes on to perform a linear stability analysis of a zonally symmetric shear flow on an $f$-plane, taking into account the non-traditional component of the full Coriolis force. They find that the growth rate and spatial structure of unstable modes may be significantly altered by inclusion of the full Coriolis force. Na\"ively, one may expect to observe drastic changes in the above numerical simulations, given that on an ordinary $f$-plane there is no difference between a meridional and zonal flow (as the system has no preferred orientation). However, the inclusion of the non-traditional component of the Coriolis force breaks this symmetry and reintroduces a natural meridional direction. To understand how symmetric instability in meridional flows changes under the complete Coriolis force, we must repeat the linear stability analysis of~\citet{Zeitlin2018a} but for the case of a meridional flow.

    The Boussinesq equations of motion for a meridionally symmetric flow on an $f$-plane, with the complete Coriolis force, and in the limit in which the horizontal length scale is much larger than the vertical length scale, are
    \begin{subequations}
    \begin{equation}
        % \bigg( \pdv{t} + u \pdv{x} + w \pdv{z} \bigg)
        \frac{Du}{Dt} - fv + F_{NT}w - \pdv{\phi}{x} = A_r \pdv[2]{z} u \, ,
    \end{equation}
    \begin{equation}
        \frac{Dv}{Dt} + fu = A_r \pdv[2]{z} v \, ,
    \end{equation}
    \begin{equation}
        \frac{Dw}{Dt} - F_{NT}u - \pdv{\phi}{z} - b = A_r \pdv[2]{z} w \, ,
    \end{equation}
    \begin{equation}
        \pdv{u}{x} + \pdv{w}{z} = 0 \, ,
    \end{equation}
    and
    \begin{equation}
        \frac{Db}{Dt} = \kappa \pdv[2]{z} b \, ,
    \end{equation}
    \end{subequations}
    where, $A_r$ is the viscosity, $\kappa$ is the buoyancy diffusion coefficient and $F_{NT}$ is the non-traditional component of the Coriolis parameter.

    Similarly to section~\ref{sec:InviscidInstabilities} we will now consider a flow which is, initially, purely meridional, with $(u, v, w) = (0, V, 0)$, $\phi = \Phi$ and $b = B$. The flow is steady, in geostrophic balance and hydrostatically balanced. Mathematically
    \begin{subequations}
    \begin{equation}
        f V = \pdv{\Phi}{x} \, ,
    \end{equation}
    \begin{equation}
        B = \pdv{\Phi}{z} \, ,
    \end{equation}
    \begin{equation}
        \pdv{B}{z} = N^2 \, ,
    \end{equation}
    \begin{equation}
        \pdv[2]{B}{z} = 0 \, .
    \end{equation}
    \begin{equation}
        \pdv{B}{x} = M^2 \, ,
    \end{equation}
    and
    \begin{equation}
        \pdv[2]{V}{z} = 0 \, .
    \end{equation}
    \end{subequations}
    Again, we will perturb this balanced flow, giving perturbed variables $(u, v, w) = (u', V + v', w')$, $\phi = \Phi + \phi'$ and $b = B + b'$. Substituting these values into the equations of motion and considering terms only of linear order or lower in the perturbed variables, we find that
    \begin{subequations}
    \begin{equation}
        \label{eq:ZonalMomentumViscous}
        % geostrophic balance
        \pdv{u'}{t'} + F_{NT}w' - fv' + \pdv{\phi '}{x} = 0 \, ,
    \end{equation}
    \begin{equation}
        \pdv{v'}{t'} + u' \pdv{V}{x} + w' \pdv{V}{z} + fu' = 0 \, ,
    \end{equation}
    \begin{equation}
        \label{eq:VerticalMomentumViscous}
        % Hydrostatic balance
        \pdv{w'}{t'} + \pdv{\phi '}{z} - b' - F_{NT}u' = 0 \, ,
    \end{equation}
    \begin{equation}
        \pdv{b'}{t} - \kappa \pdv[2]{b'}{z} + w' \pdv{B}{z} + u' \pdv{B}{x} = 0 \, ,
    \end{equation}
    and
    \begin{equation}
    \label{eq:ContinuityViscous}
        \pdv{u'}{x} + \pdv{w'}{z} = 0 \, ,
    \end{equation}
    \end{subequations}
    where the operator $\flatfrac{\partial}{\partial t'}$ is defined as
    \begin{equation}
        \pdv{t'} = \pdv{t} - A_r \pdv[2]{z} \, .
    \end{equation}
    
    From equation~\ref{eq:ContinuityViscous} we see that we can again write $u'$ and $w'$ in terms of an overturning streamfunction $\psi$, where $u' = - \partial_z \psi$ and $w' = \partial_x \psi$. We can then attempt to find an equation of motion for the streamfunction, analagous to equation~\ref{eq:OriginalHoskins}. As before, we differentiate the zonal and vertical momentum equations (equations~\ref{eq:ZonalMomentumViscous} and~\ref{eq:VerticalMomentumViscous}) with respect to the vertical and zonal coordinates respectively and then subtract them, giving
    \begin{equation}
        \pdv{}{t'} \bigg( \pdv[2]{x} + \pdv[2]{z} \bigg) \psi - f \pdv{v'}{z} + \pdv{b'}{x} + F_{NT} \bigg( \pdv{u'}{x} + \pdv{w'}{z} \bigg) = 0 \, .
    \end{equation}
    We note that the only place $F_{NT}$ enters this equation is in the final term, and that becaue of the continuity equation (equation~\ref{eq:ContinuityViscous}) this term is equal to zero. If we now differentiate the above with respect to $t'$ and substite in the equations for $\partial_{t'}b'$ and $\partial_{t'}v'$ we get
    \begin{equation}
        \frac{\partial^2}{\partial t'^2} \bigg(\pdv[2]{}{x} + \pdv[2]{}{z}\bigg) \psi + \bigg( N^2 \pdv[2]{}{x} - 2 f \pdv{V}{z} \pdv[2]{}{x}{z} + f \zeta \pdv[2]{}{z} \bigg) \psi = 0 \, .
    \end{equation}
    which apart from the viscous terms is indentical to equation~\ref{eq:OriginalHoskins}\footnotemark. This has absolutely no dependence on $F_{NT}$ at all and from this we conclude that the complete Coriolis force does not alter symmetric instability in meridional, parallel shear flows. This finding is not in any way contradictory to the findings of~\citet{Zeitlin2018a}. The difference arises due to the asymmetry between the purely meridional flow considered here and the zonal flow considered in the aforementioned study.
    \footnotetext{Here we have set $\kappa=A_r$. This is done to simplify the notation and the arguments subsequently made do not lose their generality.}

    It is, in fact, possible to make a more general statement about the types of forces which do not alter the evolution of meridional symmetric instabilities. We can modify the momentum equations (\ref{eq:ZonalMomentum} and~\ref{eq:VerticalMomentum}) with the addition of any irrotational force acting in the $xz$-plane and which satisfies the relationship
    \begin{equation}
        \pdv{\mathcal{F}_x}{z} + \pdv{\mathcal{F}_z}{x} = 0 \, ,
    \end{equation}
    where $\mathcal{F}_x$ and $\mathcal{F}_z$ are the zonal and meridional components of the force respectively. This can be understood in the framework of~\citet{Marshall2011} as follows --- an irrotational force is divergent and so will project on to the pressure gradient terms of the momentum equation. A rotational force is non-divergent and so projects entirely onto the acceleration term. An irrotational (divergent) force is not able to alter the acceleration term. In the system described above, the inclusion of the complete Coriolis force may alter the pressure field but not the motions within the $xz$-plane due to its irrotational nature.

    Although the crossing of the equator is a meridional phenomenon, western boundary currents, such as the North Brazil Current or the deep western boundary current, will be oriented at some angle to a meridian, and so have both zonal and meridional components of velocity. In the zonal limit, symmetric instability can change drastically with the inclusion of the full Coriolis force, whereas in the meridional limit there is no change at all. For a realistic (not purely meridional) western boundary current crossing the equator, the structure of symmetric instability may therefore have some dependence on the complete Coriolis force.
    
    The relative importance of non-traditional effects depends on the direction of the current relative to the meridional direction. For a current oriented at an angle $\theta$ to the meridional direction, the findings of~\citet{Zeitlin2018a} apply but with the value of $F_{NT}$ scaling with $\sin \theta$. \citet{Zeitlin2018a} defines a ``non-traditionality'' parameter and using the findings of our work we can generalise it to flows with a meridional component giving
    \begin{equation}
        \gamma = \frac{\cot\phi\,\sin \theta\,H}{L}
    \end{equation}
    where $\gamma$ is the  ``non-traditionality parameter'' and $\phi$ is the latitude. Non-traditional effects are important when $\gamma \sim 1$. Close to the equator, the coast of Brazil forms an angle of $\theta \sim 60^\circ$ to the meridian. From work that will be discussed in subsequent chapters, we suspect symmetric instability in the North Brazil Current to occur at a latitude of around $4^\circ$N, and the current has $H\sim100$ m and $L\sim30$ km giving $\gamma \sim 0.04$. This suggests that non-traditional effects are unlikely to be hugely important in the North Brazil Current. For the deep western boundary current we expect $\phi \sim 1^\circ$N, $H\sim 1,000$ m and $L \sim 50 km$. This gives $\gamma \sim 1$ suggesting non-traditional effects may be important here. Finally, the Irminger current sits at around $60^\circ$N, forms an angle of $\theta \sim 30^\circ$ to the meridian, has $H \sim 100 m$ and $L \sim 30 km$. Due to its high latitude we suspect non-traditional effects will be very weak, and this is reflected in a value of $\gamma \sim 10^{-3}$.