\chapter{Symmetric instability in a cross-equatorial surface current}
\label{chap:3}
\begin{quote}
    \textit{When I find myself in times of trouble, Father Hoskins comes to me, speaking words of wisdom, it's PV.} --- Paul McCartney
\end{quote}

We have seen in the previous chapter how the change in sign of planetary vorticity at the equator, coupled with the conservation of potential vorticity, leads us to expect symmetric instability to be excited in cross-equatorial western boundary currents. In this chapter, we will explore results from high-resolution idealised numerical models of a surface intensified western boundary current, as an analogue of the North Brazil Current.

In section~\ref{sec:NBCandR}, we will describe in detail the North Brazil Current and its rings, and motivate this study. In section~\ref{sec:2DModels}, we introduce a two dimensional numerical model used to investigate how symmetric instability develops in the Tropical Atlantic, before applying a three dimensional model to the problem in section~\ref{sec:3DModels}. We use this model to estimate the amount of mixing that is induced by symmetric instability and to investigate the potential vorticity dynamics in the cuerrent. We end the chapter with a summary our findings in section~\ref{sec:Summary3}.

The content of this chapter is based on a paper published in the Journal of Physical Oceanography in which I took the lead in experimental design, implementation and authoring \citep{Goldsworth2021}.

\section{The North Brazil Current and its rings}
\label{sec:NBCandR}
The tropical Atlantic contains two major western boundary current systems. At the surface there is the North Brazil Current and its rings which flow northward and below it sits the southwards flowing Deep Western Boundary Current. The flows considered in this chapter are dynamically similar to the North Brazil Current; however, the conclusions of this work are expected to apply to cross-equatorial flows in upper-ocean western boundary currents more generally.

The North Brazil Current follows the coast of Brazil from around 10$^\circ$S -- 15$^\circ$S to 4$^\circ$N -- 8$^\circ$N. Here, it retroreflects and breaks up into anticyclonic eddies called the North Brazil Current rings, which continue northward to the Caribbean~\citep{Talley2011AtlOce}. The latitude of retroflection of the North Brazil Current is set by the position of the intertropical convergence zone~\citep{Fonseca2004}. The position of this region is known to depend strongly upon AMOC heat transports and vary seasonally~\citep{Zhang2005, Fuckar2013}. At the intertropical convergence zone, there is a zero in the wind stress curl, and this forces the North Brazil Current to cross the equator. If the wind stress curl was perfectly symmetric about the equator, there would be no gyre contribution to inter-hemispheric water exchange, just sufficient mass exchange to close the AMOC.

Both the North Brazil current and its rings are important pathways in the northward transport of Southern Hemisphere waters and form an important component of the AMOC~\citep{Bower2019}. The current has an annual mean transport of around 32 Sv (1 Sv$\equiv 10^{6}$ m$^3$\,s$^{-1}$) in the upper 600 m of the ocean, with typical peak velocities of around 80 cm\,s$^{-1}$ and a width of approximately 100 km~\citep{Johns1998, Schott1993}. At the equator there is a breakdown in geostrophy due to the vanishing of the Coriolis parameter; however, at latitudes 1$^\circ$ either side of the equator, the current is geostrophically balanced to leading order, while the rings are in cyclogeostrophic balance~\citep{Vianna2003, Castelao2011}. Typically, North Brazil Current rings have a radii of $\sim 100$~km and rotational velocities of $\sim 1$~m\,s$^{-1}$, and much slower translational velocities of $\sim 10^-2$~m\,s$^{-1}$~\citep{Castelao2011}.

The change in sign of the locally vertical component of planetary vorticity at the equator is an important constraint on how water can cross from one hemisphere to the other. The potential vorticity of a fluid parcel is materially conserved in the absence of mechanical and buoyancy forcing and neglecting non-linearities in the equation of state. Waters starting in the Southern Hemisphere typically have negative potential vorticity as a result of the vertical component of their planetary vorticity. Close to the equator, the planetary vorticity varies approximately linearly with latitude --- the $\beta$-plane approximation. As Southern Hemisphere waters flow northward across the equator, their planetary vorticity increases and so, to conserve potential vorticity, the flow generates anti-cyclonic relative vorticity. \citet{Killworth1991} shows that this requirement to conserve potential vorticity inhibits the penetration of fluid from one hemisphere to another, further than a few Rossby deformation radii.

In a numerical model of cross-equatorial flow,~\citet{Edwards1998I} find an anti-cyclonic eddy field generated at the equator. They show that this field is responsible for advecting potential vorticity, of the opposite sign to the planetary vorticity, into a viscous boundary layer where it can be dissipated. This provides a mechanism for the modification of potential vorticity, whose conservation would otherwise inhibit cross-equatorial flow. The eddy field is observed in other models of cross-equatorial flow~\citep[e.g.][]{Jochum2003, Goes2009}. In a follow-up study,~\citet{Edwards1998II} show that the eddy field is generated by barotropic instability, and is enhanced as the Rossby deformation radius is maximal at the equator. It has been proposed this mechanism is behind the generation of the North Brazil Current rings~\citep{Jochum2003}. Viscous boundary layers, however, can play only a limited role in the modification of potential vorticity. As such, turbulent processes in the current's interior are more likely to be of importance. These turbulent processes are of particular interest as they may lead to a short-circuiting of the AMOC through enhanced diapycnal mixing. Symmetric instability is a turbulent process that may be capable of modifying the potential vorticity in cross-equatorial flows.

\section{Two-dimensional numerical models}
\label{sec:2DModels}
The linear stability analysis of section~\ref{subsec:NBClsa} makes predictions of the growth rate of viscous symmetric instability in a meridionally symmetric, parallel shear flow situated at approximately 4$^\circ$N. We will now consider a two-dimensional numerical model of a similar flow, but with vertical shear, variable stratification and finite depth. The two-dimensional nature of the flow will be retained, via the imposition of a periodic meridional boundary condition on a domain one grid cell thick. This means there is no $\beta$-effect and that equatorially enhanced barotropic instability will not occur~\citep{Edwards1998I, Edwards1998II}.

\subsection{Methods}
\label{subsec:2DMethods}
The numerical simulations are performed using the MITgcm model~\citep{Marshall1997}. The model domain consists of a channel 400 km wide in the zonal direction and periodic in the meridional direction. The horizontal resolution is 2 km. In the vertical there are 160 depth levels, varying in size from 6.25 m at the surface to 25 m at the bottom, giving a total depth of 1,500 m. The horizontal resolution is chosen such that the model can resolve the small-scale vorticity structure that is crucial to the excitement of symmetric instability, and the vertical resolution is chosen so that it is smaller than the expected size of the overturning cells the instability will generate.

Simulations are run on an $f$-plane, with $f = 1.01 \times 10 ^{-5}$ s$^{-1}$, corresponding to a latitude of $4^\circ$N, with the non-traditional component of the Coriolis force included also, taking a value of $F_{NT} = 1.45 \times 10^{-4}$ s$^{-1}$. A control integration is run in which the non-traditional component of the Coriolis force is neglected. A further set of integrations at $40^\circ$N and $0^\circ$N, both with and without the non-traditional component of the Coriolis force, are performed. For the integration at $40^\circ$N, $f = 9.35 \times 10^{-5} $ s$^{-1}$ and $F_{NT} = 1.11 \times 10^{-4}$ s$^{-1}$; at $0^\circ$N, $f = 0$ s$^{-1}$ and $F_{NT} = 1.45 \times 10^{-4}$ s$^{-1}$. In all cases, no qualitative difference between integrations with and without the complete Coriolis force is found; this finding is unsurprising given the results presented in section~\ref{sec:DrasticSI}.

Closure of the momentum equations is provided by an adaptive biharmonic Smagorinsky viscosity, which operates on the horizontal derivatives of momentum. This is chosen to minimize damping at the length-scales of interest~\citep{Smagorinsky1963, Griffies2000}, with the choice inspired by its successful use in~\citet{Brannigan2016} which attempts to resolve similar submesoscale processes. A Laplacian viscosity, $A_r$, which operates on vertical derivatives of momentum is also used. It is set to a constant value of $4 \times 10^{-4}$ m$^2$\,s$^{-1}$ for all integrations, apart from a viscous integration for which a value of $6 \times 10^{-3}$ m$^2$\,s$^{-1}$ is used. The standard value is chosen to ensure the vertical structure of the symmetric instability can be adequately resolved by the model grid.

A linear equation of state is employed:
\begin{equation}
    \rho = \rho_{0} \big(1 - \alpha_T (T - T_0 )\big) \, ,
    \label{eq:EOS}
\end{equation}
where $\rho_{0}$ is the background density, $T$ is temperature, $T_0$ is a reference temperature, and $\alpha_T$ is the thermal expansion coefficient. The absence of salinity means that changes in density are modulated solely by changes in temperature. The value of $\rho_{0}$ is set to 1023.35 kg$\,$m$^{-3}$, $\alpha_T$  set to $2 \times 10^{-4}$ K$^{-1}$, and $T_0$ set to $30^{\circ}$C. The linear equation of state is used to avoid the complexities added by non-linear effects such as cabbelling and thermobaricity~\citep[e.g.][]{Groeskamp2016}. The diffusivity of temperature is set to $1 \times 10^{-5}$ m$^2$\,s$^{-1}$. A second order-moment Prather advection scheme with a flux limiter is used~\citep{Prather1986}. The initial density profile is based on observations taken by an Argo float in the tropical Atlantic~\citep{Argo2019}. The neutral density profile and float trajectory are shown in figure~\ref{fig:InitialStratification}, with the temperature profile in the model chosen to match the density profile of the aggregated observations.

The initial velocity profile is based on that of the surface intensified Bickley jet. This can be expressed as
\begin{equation}
    V(x, z) = V_0 \Bigg( 1 - \tanh[2](\frac{x - x_{mid}}{\delta_b})\Bigg) \frac{z + H}{H} \, ,
\end{equation}
where $z$ is the vertical coordinate which becomes increasingly negative below the surface, and $H$ is the absolute value of the depth. The jet parameters are set as follows: $V_0 = 1$ m\,s$^{-1}$, $x_{mid} = 40$ km, $\delta_b = 30$ km, $H = 1,500$ m. The profile is shown in figure~\ref{fig:SurfIntBickleyJet}. To test the sensitivity of the model to the jet's position, two control simulations were also run with $x_{mid}$ set to either 20 km or 80 km: the resulting instability shows no qualitative dependence on the jet's position. The jet parameters are chosen to be similar to those observed in the North Brazil Current by~\citet{Johns1998}. However, it should be noted that the jet described in this work is more intense ($V_{max} \sim 1$ m\,s$^{-1}$ rather than $\sim 0.9$ m\,s$^{-1}$), has less vertical shear ($\partial_z V \sim 6.6\times 10^{-4}$ m$^{-1}$\,s$^{-1}$ rather than $\sim 3.5 \times 10^{-3}$ m$^{-1}$\,s$^{-1}$), and is deeper ($H\sim 1,500$ m rather than $\sim 800$ m) than that observed by~\citet{Johns1998}. The depth is chosen to be large, to prevent the bottom boundary layer from having too strong an influence on the evolution of any instabilities, hence the low shear and deeper current.

\begin{figure} 
    \centering
    \includegraphics{../figures/BickleyJet.pdf}
    \caption{Velocity structure of the surface intensified Bickley jet with $V_0 = 1$~m$\,$s$^{-1}$, $x_{mid} = 40$~km, $\delta_b = 30$~km and $H = 1,500$~m.}
    \label{fig:SurfIntBickleyJet}
\end{figure}

In the 4$^\circ$N integrations, the jet prescribed is symmetrically unstable from the outset. This allows us to explore the evolution of unforced symmetric instability in a `plausible' western boundary current. In a two-dimensional numerical model, the meridional advection of Southern Hemisphere waters with negative potential vorticity into the domain is not possible. As such we must initialise the numerical integration with waters of negative potential vorticity already in place.

At the surface boundary, a rigid lid condition is used. The lateral boundary conditions at the eastern and western edges of the domain are set to no slip. At the bottom, a free-slip boundary condition is used. The sensitivity of the simulations to the choice of boundary condition has been tested by performing model integrations with free-surface, free-slip lateral, and no-slip bottom boundary conditions: in each case, no qualitative differences in the resulting instability are observed.

\subsection{Results and their relation to the linear stability analysis}
Snapshots of potential vorticity at two-week intervals in the standard two-dimensional numerical integration are shown in figure~\ref{fig:2DPVSnapshotsStandard}. After around 5 days the potential vorticity distribution begins to change (not shown) and after two weeks the excitement of the instability is visible. The timescale (the inverse of the growth rate, $\flatfrac{1}{\sigma}$) predicted by the linear stability analysis is $\sim 2$ days, which is of the right order of magnitude for what is seen here. In the leftmost panel (c), we see how the potential vorticity in the initially unstable region has been modified and set to a state of marginal stability --- i.e. $Q = 0$. This modification of potential vorticity is driven by the overturning cells, which were predicted by the linear stability analysis. Due to the along stream symmetry, we can rearrange the equation for the conservation of potential vorticity to give
\begin{equation}
    \label{eq:PVEvolution}
    \pdv{Q}{t} = \pdv{\psi}{z}\pdv{Q}{x} - \pdv{\psi}{x}\pdv{Q}{z} \, ,%- \div{(\grad{b} \cross \mathbf{G})}\, ,
\end{equation}
which explicitly links the unstable overturning generated by the instability and the redistribution of potential vorticity\footnotemark.
\footnotetext{This equation neglects the frictional dissipation of potential vorticity; however, this contribution is generally much smaller than the contribution from advection.}

\begin{figure}
    \centering
    \includegraphics{../figures/2d_pv.pdf}
    \caption{Snapshots of potential vorticity over time in the standard (no-slip) two-dimensional model. Potential vorticity is shown as a function of depth and longitude.}
    \label{fig:2DPVSnapshotsStandard}
\end{figure}

Figure~\ref{fig:2DStandardOverturningStreamfunction} shows the zonal overturning stream-function from the standard two-dimensional numerical integration. We see structures similar to those predicted by the linear stability analysis (c.f. figure~\ref{fig:2DStructure}). The vertical Fourier transform of the stream function is taken at two weeks. At the longitude of the absolute vorticity minimum, there is a maximum in the power spectrum with a vertical wavelength of 100 m. This estimate of the size of the overturning cells is remarkably similar to the prediction of 105 m made by the linear stability analysis. We would expect fairly close agreement as the parameters used in the linear stability analysis are based on the jet used to initialise the model.

\begin{figure}
    \centering
    \includegraphics{../figures/2d_overturning.pdf}
    \caption{Snapshots of the zonal overturning stream-function generated by symmetric instability in the standard (no-slip) two-dimensional model. The five-day moving average of the stream function is taken to mask the effects of internal waves.}
    \label{fig:2DStandardOverturningStreamfunction}
\end{figure}

The resolution of the model means that secondary Kelvin-Helmholtz instabilities that typically accompany the excitement of symmetric instability are absent. Furthermore, the use of an artificially high vertical viscosity, to ensure numerical stability,  means the growth rate of the instability is suppressed and its vertical extent exaggerated. Although symmetric instability may look different in the ocean, the conclusion that symmetric instability is efficient at eliminating anomalous potential vorticity within a few degrees of the equator will not change --- if anything the process will be more efficient due to the lower viscosity of the ocean meaning the growth rate of the instability is larger.

The linear stability analysis found that by setting the vertical viscosity to be sufficiently high, the symmetrically unstable motions can be suppressed. Integrations with a high vertical viscosity of $6 \times 10^{-3}$ m$^2$\,s$^{-1}$ are performed and, indeed, no instability develops within them. The linear stability analysis also predicts no instability if the experiment is performed at a sufficiently high latitude, or at the equator: model integrations at 40$^{\circ}$N and 0$^{\circ}$N verify this is indeed the case (not shown).

\section{Three-dimensional numerical models}
\label{sec:3DModels}
Although a useful tool in understanding the processes at play in cross-equatorial flows, two-dimensional numerical models fail to capture many of their key dynamics.
Perhaps the most important missing ingredients are the $\beta$-effect and meridionally asymmetric motions, both of which are suppressed by the imposition of along-stream symmetry. It is the variation and change in sign of planetary vorticity at the equator which we suspect will lead to symmetric instability. Furthermore,~\citet{Edwards1998II} show that barotropic eddies are an important and robust feature of cross-equatorial flows. These motions cannot form if there is no meridional variation in the flow. To fully understand what is going on in the ocean, we must relax the requirement of along-stream symmetry and turn to three-dimensional numerical models. This section describes an idealised three-dimensional model of the tropical Atlantic based on the two-dimensional model discussed previously.

\subsection{Methods}
The horizontal domain is 816 km in the zonal and 2,688 km in the meridional direction. The horizontal resolution is 2 km. The southernmost boundary is located 512 km south of the equator and the northern boundary is 2,176 km to the north. The velocity at the northern and southern boundaries is prescribed and takes the form of the surface intensified Bickley jet, as shown in figure~\ref{fig:SurfIntBickleyJet}b. The zonal velocity is initially set to zero and the meridional velocity to the same surface intensified Bickley jet prescribed at the northern and southern boundaries. As in the models previously discussed, this means the initial potential vorticity configuration is unstable to symmetric instabilities. In the Northern Hemisphere, within around 3 weeks, the majority of the negative potential vorticity initially present has been neutralized by an initial flurry of symmetric instability (in the standard viscosity runs). After this time the largest source of negative potential vorticity in the Northern Hemisphere is waters advected across the equator.

To absorb any incoming waves or eddies, there exists a sponge region within the domain, which relaxes the velocity to the Bickley jet profile prescribed at the northern and southern boundaries. The sponges stretch from 350 km south of the northern and 40 km north of the southern edges of the domain. The inverse relaxation timescale in these sponge regions varies according to a $\tanh$ function from 0 s$^{-1}$ in the interior of the sponge to $2 \times 10^{-5}$ s$^{-1}$ at the open boundary. The sponge regions are cropped from all figures.

Unlike the two-dimensional model integrations, the three-dimensional integrations evolve in qualitatively different ways depending on whether a no-slip or free-slip lateral boundary condition is imposed. The no-slip boundary condition is taken to be the standard choice, although results from free-slip integrations are also presented. A time step of 144 seconds is used for the no-slip model and 72 seconds for the free-slip model, which has a higher Reynolds number.

Integrations are performed with either a standard vertical viscosity of $4\times 10^{-4}$ m$^2$\,s$^{-1}$, or a high vertical viscosity of $1\times 10^{-2}$ m$^2$\,s$^{-1}$. The high viscosity is larger than the value of $6\times 10^{-3}$ m$^2$\,s$^{-1}$ used in the two-dimensional high viscosity runs. This is because the former value was insufficient to shut down the excitement of symmetric instability at high latitudes. The larger value is still not strong enough to completely shut down the excitement of symmetric instability; however, it does appreciably suppress the growth of the instability.

The $\beta$-plane approximation is made, which gives a linear latitudinal variation of planetary vorticity of $2.3\times 10^{-11}$ m$^{-1}$\,s$^{-1}$. The non-traditional component of the Coriolis force is also included by setting $F_{NT}$ to a constant value\footnotemark~of $1.5\times 10^{-4}$ s$^{-1}$.

\footnotetext{Strictly speaking, in a spherical geometry we would expect $F_{NT} = 2\Omega \cos\theta$ where $\theta$ is the latitude. At the equator, the variation in $F_{NT}$ with latitude is minimal, and over the domain considered here, $F_{NT}$ is constant to 2 significant figures. As such it was decided to ignore the meridional variation in $F_{NT}$ and make the ``non-traditional $f$-plane'' approximation.}

The same thermodynamic scheme and initial stratification are used as in the two-dimensional numerical models previously discussed. In much of the following discussion, analyses are performed at a depth of 50 m. This depth is chosen as it sits below the model's mixed layer but is still close to the surface where the excitement of symmetric instability is expected to be most vigorous.

\subsection{Results \& discussion}
\label{subsec:3DResults}
\subsubsection{No-slip lateral boundary conditions}
Figure~\ref{fig:PVStandardNoSlip3D} shows snapshots of the potential vorticity field in the standard three-dimensional no-slip integration, at a depth of 50 m. Close to the equator, we see the spinning up of anticyclonic eddies of up to 200 km in diameter. The eddies are dynamically similar to those seen in other studies of cross-equatorial flow and are a result of barotropic instability~\citep[e.g.][]{Edwards1998I, Edwards1998II, Jochum2003, Goes2009}. Between around 250 km and 1000 km north of the equator, smaller scale features within the eddies are visible: these arise from the excitement of symmetric instability.

There is an interesting interplay between barotropic and symmetric instabilities. We see in the Northern Hemisphere of figure~\ref{fig:PVStandardNoSlip3D}a that the eddy with anticyclonic potential vorticity initially penetrates a few hundred kilometres before it begins to retroflect. Within around a week, the potential vorticity within the eddy has been reduced and it can propagate further northwards (see figure~\ref{fig:PVStandardNoSlip3D}b). If we look at the submesoscale, we see features with large spatial potential vorticity gradients over short distances. This is a result of the excitement of symmetric instability which acts to redistribute potential vorticity within the eddy. By reducing its potential vorticity the eddy can propagate even further north (figure~\ref{fig:PVStandardNoSlip3D}c). As the eddy moves further northwards the growth rate of symmetric instability increases and any remaining regions of negative potential vorticity become increasingly unstable. Eventually, we are left with an anticyclonic eddy with approximately neutral potential vorticity, corresponding to a state of marginal stability. The core of the eddy is one of the last regions to reach this stable state. Figure~\ref{fig:EddyRelativeAndAbsoluteVorticity} shows a comparison of the relative and potential vorticity of an eddy which has undergone this process. We see that close to the equator regions of negative potential vorticity correspond to regions of negative relative vorticity. In a region from around 500 km to 800 km north of the equator the relationship is less clear. North of this region, the planetary vorticity is larger than the relative vorticity and contributes the most to the absolute vorticity. The excitement of symmetric instability is driving the transition from one regime to the other. South of a distance 500 km north of the equator, the relative vorticity is negative. Between 500 km and 800 km north, symmetric instability is excited. Waters leaving this region then have approximately neutral potential vorticity as a result of the excitement of symmetric instability.

\begin{figure}
    \centering
    \includegraphics[]{../figures/PV3DSnapshotStandardNoSlip.pdf}
    \caption{Snapshots of potential vorticity at 50 m depth for the standard (no-slip) three-dimensional model.}
    \label{fig:PVStandardNoSlip3D}
\end{figure}

The offset of this symmetrically unstable region from the equator can be understood in terms of the growth rate of the instability.

Firstly, in the inviscid limit, the square of the growth rate of symmetric instability is proportional to $f(f + \xi)$, where $\xi$ is the relative vorticity of the fluid. To first order (and in the absence of symmetric instability), changes in the planetary vorticity are approximately balanced by the relative vorticity, meaning $(f + \xi)$ is approximately constant along flow lines. This gives a growth rate that grows linearly with $f$ and hence latitude. Close to the equator where $f$ is small, the growth rate will be so small that other processes will dominate over symmetric instability. When a fluid parcel is advected further north, its growth rate will increase until the excitement of symmetric instability becomes apparent. At this point, symmetric instability begins to modify the potential vorticity and absolute vorticity, removing anomalous vorticity from the flow. At higher latitudes still, the source of negative potential vorticity that the instability feeds off has been largely depleted. This means we only see symmetric instability occurring within a finite latitude range.

\begin{figure}
    \centering
    \includegraphics[]{../figures/VorticityComparison.pdf}
    \caption{Instantaneous (a) relative vorticity and (b) potential vorticity, after 35 days at 50 m depth from the standard (no-slip) three-dimensional models.}
    \label{fig:EddyRelativeAndAbsoluteVorticity}
\end{figure}

Secondly, the fluid takes a finite time to show signs of symmetric instability. During this time the fluid will have been advected northwards away from the equator. This distance, $d$, can be crudely estimated by $d = V \tau_e$, where $V$ is a typical velocity of the fluid and $\tau_e$ is the timescale it takes for symmetric instability to become apparent in the flow. Taking $V\sim 80$ cm\,s$^{-1}$ and $\tau_e \sim 4$ days gives $d \sim 300$ km. From this simple calculation, we shouldn't expect the instability to be well-developed either at or close to the equator.


\begin{figure} 
    \centering
    \includegraphics{../figures/PV3DSliceSnapshotStandardNoSlip.pdf}
    \caption{Snapshots of potential vorticity as a function of depth and longitude at 750 km  North of the equator in the standard no-slip three-dimensional model.}
    \label{fig:PVatFixedLatStandardNoSlip3D}
\end{figure}

It is interesting to look at snapshots of the potential vorticity from the three-dimensional model at a fixed latitude, to track how the symmetric instability evolves, akin to what is shown in figure~\ref{fig:2DPVSnapshotsStandard}.
This is done in figure~\ref{fig:PVatFixedLatStandardNoSlip3D}. Initially the instability behaves similarly to the two-dimensional case, although the features are slightly distorted by `noise' from passing eddies and internal waves. In more realistic model set-ups and the ocean, it may not be possible to see the initial, clean evolution of symmetric instability. As such, other diagnostic quantities may be required to detect its excitement.

As with the two-dimensional numerical model, the coarseness of the grid and the use of an enhanced eddy viscosity are key limitations of the three-dimensional integrations presented here. The grid is unable to resolve fully the evolution of secondary instabilities, will suppress the growth rate, and will enhance the vertical extent of the instability. As before, it seems unlikely that this will change the fact that symmetric instability is efficient at neutralising anomalous potential vorticity in waters that have crossed the equator, especially due to the larger growth rate we expect to see when a lower, more realistic viscosity is used. Overcoming these limitations through the use of model integrations at higher resolutions remains computationally very expensive and hence infeasible. 

The potential vorticity evolution of the viscous no-slip integration is shown in figure~\ref{fig:PV3DSnapshotViscousNoSlip}. The results appear qualitatively similar to those of~\citet{Edwards1998I} whose model does not permit symmetric instability. In this model, the viscosity is leading to the dissipation of potential vorticity, which is not as dominant in the standard no-slip integration. At some latitudes, submesoscale patterns are starting to become apparent, particularly at the edges of the anticyclonic eddies. Figure~\ref{fig:PVatFixedLatForViscous3D} shows how the potential vorticity varies with depth. Immediately clear is the presence of symmetric instability-like features, suggesting the high viscosity has only partially suppressed symmetric instability. Note how the vertical length scale of the disturbance is larger in this integration than in the standard no-slip case. This can be understood with reference to figure~\ref{fig:DispersionRelation} --- higher viscosities lead to larger vertical length scales.

\begin{figure} 
    \centering
    \includegraphics{../figures/PV3DSnapshotViscousNoSlip.pdf}
    \caption{Snapshots of potential vorticity at 50 m depth for the no-slip viscous three-dimensional model.}
    \label{fig:PV3DSnapshotViscousNoSlip}
\end{figure}

\begin{figure} 
    \centering
    \includegraphics{../figures/PV3DSliceSnapshotViscousNoSlip.pdf}
    \caption{Snapshots of potential vorticity as a function of depth and longitude at 750 km north of the equator, in the no-slip viscous three-dimensional model.}
    \label{fig:PVatFixedLatForViscous3D}
\end{figure}

\subsubsection{Free-slip lateral boundary conditions}
The behaviour of the three-dimensional model under no-slip and free-slip lateral boundary conditions is qualitatively different. The main difference is the behaviour of the barotropic eddies. The sensitivity to the choice of lateral boundary condition is not seen in the two-dimensional models as they do not allow these barotropic eddies to develop. The differences can be seen by comparing figures~\ref{fig:PVStandardNoSlip3D} and~\ref{fig:PV3DSnapshotStandardFreeSlip}.

\begin{figure} 
    \centering
    \includegraphics{../figures/PV3DSnapshotStandardFreeSlip.pdf}
    \caption{Snapshots of potential vorticity at 50 m depth for the free-slip three-dimensional model.}
    \label{fig:PV3DSnapshotStandardFreeSlip}
\end{figure}

In the free-slip integrations, the eddies grow to be larger, and propagate northwards more quickly, than in the no-slip integrations. They do not retroflect in the same way as before. As they propagate northwards, they entrain waters of negative potential vorticity and elongate, creating a concentrated pool of symmetrically unstable waters. When the eddies do become unstable, the process is much more explosive --- this can be seen in figure~\ref{fig:PV3DSnapshotStandardFreeSlip}. As the eddies propagate northwards more quickly, the symmetrically unstable region is shifted northwards. This is because it takes a similar time for symmetric instability to develop, but the eddies have moved further north of the equator during this time.

The potential vorticity evolution from a free-slip viscous integration is shown in figure~\ref{fig:PV3DSnapshotViscousFreeSlip}, reinforcing that, with the noise of symmetric instability removed, the barotropic eddies behave very differently under the two different lateral boundary conditions. This raises the question of what the most appropriate boundary condition is. For the idealised model setup, this is likely the no-slip lateral condition. In a realistic model, given the small aspect ratio of the problem, a no-slip bottom boundary and free-slip lateral boundary condition, along with variable bathymetry might better represent the physics. This is a result of gently sloping ocean bathymetry in the region meaning that the sea floor is better characterised as a bottom boundary, rather than a lateral boundary.

\begin{figure} 
    \centering
    \includegraphics{../figures/PV3DSnapshotViscousFreeSlip.pdf}
    \caption{Snapshots of potential vorticity at 50 m depth for the free-slip viscous three-dimensional model.}
    \label{fig:PV3DSnapshotViscousFreeSlip}
\end{figure}

\subsection{Correlations between relative and potential vorticity}
It is useful to be able to define concretely the latitudes at which symmetric instability is occurring. Thus far, this has been done visually by looking at the regions in which submesoscale patterns begin to form in areas of anticyclonic vorticity. It has been noted that regions in which symmetric instability is occurring mark a transition from a regime in which potential vorticity is dominated by relative vorticity to one in which planetary vorticity dominates. Figure~\ref{fig:EddyRelativeAndAbsoluteVorticity} suggests it may be possible to identify latitudes at which symmetric instability is occurring by considering correlations between relative vorticity and potential vorticity.

The relative vorticity and potential vorticity are interpolated onto the same grid. The Pearson correlation coefficient, $r$, of the vertical component of relative vorticity and potential vorticity between 21 days and 49 days is calculated for all grid points between the western boundary and 400 km east at 50 m depth for each latitude. The calculation starts at 21 days so that any instability resulting from the initial conditions has died down. How the correlation coefficient varies with latitude is shown in figure~\ref{fig:VorticityCorrelations}, for both the no-slip and free-slip, and standard and viscous integrations.
In figure~\ref{fig:VorticityCorrelations}, we see that the vertical component of relative vorticity and potential vorticity are very strongly correlated in the Southern Hemisphere. At around 250 km north of the equator, there is an abrupt change in the correlation of the standard no-slip and free-slip integrations. The latitude of the change seems to correspond to the latitude north of which we observe symmetric instability. In the case of the no-slip integration, $r$ reduces to around 0.5, and for the free-slip integration, it is reduced even further, to less than 0.3. This makes sense --- in the free-slip experiment the excitement of symmetric instability appears qualitatively more vigorous than in the no-slip experiment.

\begin{figure} 
    \centering
    \includegraphics{../figures/VorticityCorrelation.pdf}
    \caption{Correlations between relative and potential vorticity as a function of latitude. Shown for integrations with a standard (solid lines) or enhanced (dotted lines) vertical viscosity, and for both no-slip (black lines) and free-slip (grey lines) models.}
    \label{fig:VorticityCorrelations}
\end{figure}

We see a much smaller and more gradual drop in the correlation for the viscous integrations, and the decrease in $r$ appears to begin at much higher latitudes --- around 500 km for the free-slip and 1,500 km for the no-slip integration. This is consistent with what we see in the models: at low latitudes, we see little evidence of symmetric instability; however, its effects become apparent at higher latitudes.

We can understand the change in correlations in terms of the potential vorticity of a meridional flow, given by
\begin{equation}
    Q \approx \bigg(N^2 f - M^2 \pdv{V}{z} \bigg) + N^2 \xi
\end{equation}
where $N^2$ and $M^2$ are the vertical and horizontal buoyancy gradients, respectively. In the absence of symmetric instability, at a given latitude, one would expect there to be a strong correlation between $Q$ and $\xi$. If $N^2$ and $f$ are approximately constant, and $M^2$ and $\partial_z V$ are small, then the bracketed term of the above equation will be approximately constant and a linear relationship between $Q$ and $\xi$ will exist. When symmetric instability is excited, localized vortex stretching and changes in the vertical shear cause a breakdown of this linear relationship. As such, one would expect the correlation between the vertical component of relative vorticity and potential vorticity to be much lower. The fact that the correlation changes only slightly in the viscous runs, which exhibit barotropic eddies but suppressed excitement of symmetric instability, suggests that the changes in the linear relationship are the result of symmetric instability.

\subsection{Spectral energy density}
Symmetric instability is a submesoscale instability, suggesting that when it is excited, we would expect to see an increase in the amount of energy at small length scales. \citet{Yankovsky2019} find that in their models of Arctic overflows, symmetric instability has a fingerprint in the spatial Fourier transform of the vertical kinetic energy (KE). This technique is used here in an attempt to diagnose symmetric instability.

For the standard three-dimensional no-slip integration, the quantity $\flatfrac{w^2}{2}$ is calculated at a depth of 50 m from the western boundary to 200 km east, and from 250 km to 750 km north. The power spectrum is taken along the zonal dimension and the meridional mean is taken. This is plotted at one-week intervals in figure~\ref{fig:SpectralEnergyDensity}a. The same is done for the three-dimensional viscous no-slip integration in figure~\ref{fig:SpectralEnergyDensity}b. The same procedure is performed for the single latitude of the two-dimensional standard and viscous no-slip models and the results are shown in figures~\ref{fig:SpectralEnergyDensity}c and~\ref{fig:SpectralEnergyDensity}d respectively.

\begin{figure} 
    \centering
    \includegraphics{../figures/WVELSpectra.pdf}
    \caption{Spectral distribution of vertical kinetic energy for the (a) standard three-dimensional, (b) viscous three-dimensional, (c) standard two-dimensional, and (d) viscous two-dimensional models. Note that the vertical viscosity used in the viscous two-dimensional integration is $6 \times 10^{-3}$ m$^2\,$s$^{-1}$ compared to $10^{-2}$ m$^2\,$s$^{-1}$ for its three-dimensional counterpart. The darkening line colour corresponds to later model times.}
    \label{fig:SpectralEnergyDensity}
\end{figure}

In each of the panels a, b \& c of figure~\ref{fig:SpectralEnergyDensity}, we see a flattening of the power spectra over time. This corresponds to the development of vertical motions with a structure that varies over small-length scales. In the case of the standard three-dimensional model, within around a week, the power spectrum reaches an equilibrium.

In the viscous three-dimensional model, it takes around three weeks for the flattening to take place. This is due to the increased vertical viscosity suppressing symmetric instabilities. We still see the flattening effect of symmetric instability at later times, as the viscosity is not sufficient to completely inhibit its development.

In the standard two-dimensional model, after an initial flattening of the spectra, there is a subsequent steepening from four weeks onwards. After four weeks, the two-dimensional model has used up most of its finite reserves of negative potential vorticity which fuels the symmetric instability. This means that there is no process to sustain the small-scale vertical motions and they begin to die away. This is in contrast to the three-dimensional model in which the supply of negative potential vorticity is constantly being replenished by the advection of Southern Hemisphere waters across the equator. As the two-dimensional model is not capable of producing barotropic eddies, we can be fairly certain that the changes in the power spectrum are not related to them. The increased variability relative to the three-dimensional models is due to the removal of the meridional averaging step.

In the viscous two-dimensional model, in contrast to its three-dimensional counterpart, we see no flattening at all in the power spectra\footnote{Note that the viscous two-dimensional model uses a viscosity of $6 \times 10^{-3}$ m$^2$\,s$^{-1}$ compared to $10^{-2}$ m$^2$\,s$^{-1}$ for the viscous three-dimensional model.}. This is because the vertical viscosity is sufficient to completely inhibit symmetric instability in this model. The stationarity of the power spectra allows us to say with some confidence that the flattening seen in the other models is due to the presence of symmetric instability. Indeed, the flattening of power spectra may prove useful for identifying symmetric instability in more complex ocean models and in observations, in which the signature of symmetric instability may otherwise be less obvious.

The power spectra appear to show a lot of power concentrated at the grid scale. Inspection of the vertical velocities in the region suggests this is not the result of grid-scale noise, but rather physical flow structures. Note also that the axes in figure~\ref{fig:SpectralEnergyDensity} are logarithmic.

\subsection{Transformation of water masses}
\label{subsubsec:WaterMassTransformation}
Symmetric instability leads to the generation of small-scale overturning cells. Such cells may be expected to lead to the mixing of waters with different densities and contribute to water mass transformation. In the \citet{Walin1982} water mass transformation framework, the rate of formation of water between two isopycnal surfaces $\rho$ and $\rho + \Delta \rho$ is given by
\begin{equation}
    \mathscr{F} = \pdv{\mathcal{V}}{t} + \Delta \psi = G(\rho) - G(\rho + \Delta \rho)
    \label{eq:WallinFormation}
\end{equation}
where $\mathscr{F}$ is the rate of formation of fluid between the isopycnals, $\mathcal{V}$ is the volume bounded by the isopycnals, $\Delta \psi$ is the net volume flux out of the region, and $G(\rho)$ is the diapycnal volume flux across the $\rho$-isopycnal --- also known as the water mass transformation. By convention, a positive value of $G$ corresponds to a flux from lighter to denser waters. These quantities are shown diagrammatically in figure~\ref{fig:WaterMassFormationSketch}.

\begin{figure} 
    \centering
    \includegraphics[width=0.75\textwidth]{../figures/WaterMassFormationSketch.pdf}
    \caption{Sketch of the processes contributing to water mass formation in the Walin framework. Adapted from~\citet{Williams2011}.}
    \label{fig:WaterMassFormationSketch}
\end{figure}

The water mass formation rate is calculated for density classes of $\gamma^n \leq 23.45$, $23.45 < \gamma^n \leq 26.50$~and $\gamma^n > 26.50$. Physically these correspond to a surface layer, the pycnocline and deep ocean, and are marked in figure~\ref{fig:InitialStratification} alongside the initial stratification profile. The formation is diagnosed in latitude bands with boundaries at $-500$~km, $-250$~km, $0$~km, $250$~km, $750$~km, and $1,500$~km for both the standard no-slip and viscous no-slip integrations. There is a large amount of temporal variability in the formation and so it is hard to discern any trends in the volumes of the chosen density layers. To remove this variability, the 3-day rolling average is taken. We also normalise the formation rate by the latitude band width, giving the water mass formation rate in units of Sv\,km$^{-1}$. The results of the calculation are shown in figure~\ref{fig:WaterMassFormation}. By cumulatively integrating the time series, it is easier to identify long-term trends in formation. Figure~\ref{fig:CumulativeWaterMassFormation} shows the cumulative formation (normalised by latitude band widths) in each of the density classes and latitude bands.

\begin{figure} 
    \centering
    \includegraphics{../figures/WaterMassFormation.pdf}
    \caption{Water mass formation rates for the surface, pycnocline and deep ocean (columns) within different latitude bands (rows). Rates are shown for the three-dimensional standard (solid lines) and viscous (dotted lines) no-slip models. High-frequency variability is filtered from the rates by taking a 3-day rolling average.}
    \label{fig:WaterMassFormation}
\end{figure}

\begin{figure} 
    \centering
    \includegraphics{../figures/CumulativeWaterMassFormation.pdf}
    \caption{Cumulatively integrated water mass formation rates for the surface, pycnocline and deep ocean (columns) within different latitude bands (rows). The net formation is plotted for the three-dimensional standard (solid lines) and viscous (dotted lines) no-slip models.}
    \label{fig:CumulativeWaterMassFormation}
\end{figure}

Outside of the 0~km to 250~km latitude band, there are no large changes in the net formation within the surface layers for the standard integration. Within this band, we see a large decrease in cumulative formation. This corresponds to the creation of denser waters by the action of symmetric instability, mixing surface and intermediate waters --- there is an almost equal and opposite increase in the formation of intermediate waters over time. By the time the surface waters are north of 250~km they are largely free of anomalous potential vorticity, meaning there is near-zero further mixing north of this latitude. There is little net-formation in the surface of the viscous integrations, with the vertical viscosity being sufficient to shut down the growth of symmetric instability modes within these layers.

There is a transformation between the intermediate water and deep water density classes from 250~km to 1,500~km north of the equator. The formation rates are larger in the standard than in the viscous integrations as would be expected. In the 250~km to 500~km band, we see the creation of intermediate waters from deep waters and from 500~km to 1,500~km we see the creation of deep waters from intermediate waters. Given the larger width of the northern latitude band, the net effect will be to form denser deep waters from intermediate waters. This occurs at a rate of between 2~Sv and 4~Sv from 250~km to 1,500~km north of the equator.

Symmetric instability-induced mixing is suppressed more strongly by the high viscosity in the intermediate waters than in the deep waters. This is due to the presence of the pycnocline enhancing the suppression of symmetrically unstable modes --- the high stratification acts to restrict the growth of larger modes whereas the high viscosity prevents smaller modes from becoming well established.

A key limitation of the water mass formation rates found here is that they are affected by the vertical viscosity used, suggesting different formation rates would be found if a lower viscosity were employed.~\citet{Yankovsky2019} find that mixing by symmetric instability is largely adiabatic, with irreversible mixing resulting from secondary shear instabilities. The model grid used here is too coarse to resolve such secondary instabilities, suggesting formation rates may be higher if a finer grid were used. It seems unlikely that this limitation will alter the key finding that symmetric instability is efficient at removing anomalous potential vorticity originating from the opposing hemisphere.


\section{Summary}
\label{sec:Summary3}
In this chapter, we presented results from two-dimensional and three-dimensional numerical models of a Bickley jet crossing the equator. In the two-dimensional models, we were able to see the clean evolution of symmetric instability and the generation of stacked overturning cells confined to regions of negative potential vorticity as predicted by the linear stability analyses of section~\ref{sec:ViscousSI} and~\ref{sec:DrasticSI}.

The two-dimensional model neglects the $\beta$-effect and constrains the horizontal motions of the flow. A further degree of realism was added by developing an idealised three-dimensional model of the Tropical Atlantic which incorporates these effects. As the flow evolves in the model, two main classes of instability develop. The first is the spinning up of anticyclonic eddies as the fluid crosses the equator. This is a result of barotropic instability and has been both investigated and observed in the works of~\citet{Edwards1998I, Edwards1998II} and~\citet{Goes2009}. The second instability is identified as symmetric instability for the following reasons:
\begin{itemize}
    \item The characteristic timescale of the instability is of the same order of magnitude as that predicted by linear stability analyses.
    \item The vertical length scales of the overturning cells that develop are consistent with those predicted by linear stability analyses.
    \item The instability is confined to regions of negative potential vorticity with minimal penetration of the instability into regions of positive potential vorticity.
    \item Eddies with strongly negative potential vorticity fuel the excitement of the instability, becoming unstable towards the edge of their cores first, as predicted by linear stability theory.
    \item Water leaving regions of instability have potential vorticity that is neutrally stable to symmetric instability.
    \item The latitude at which the instability occurs is explained well by arguments about the latitudinal dependence of the growth rate of symmetric instability.
    \item The instability can be suppressed in numerical models via the imposition of a sufficiently high vertical viscosity, as predicted by linear stability theory.
    \item The power spectrum of the vertical kinetic energy in the standard three-dimensional integration is similar to that in its two-dimensional counterpart. The latter unambiguously shows symmetric instability and shows that the change in the spectrum is not the result of barotropic instability. Moreover, the power spectrum of the viscous two-dimensional integration, which displays no signs of symmetric instability also remains stationary, with no flattening occurring.
\end{itemize}

We hypothesised the excitement of symmetric instability could lead to the transformation of fluids between different water mass classes. An analysis of the water mass formation rates in the surface, pycnocline and deep ocean suggests that the net contribution of symmetric instability to the AMOC's overturning budget could be significant. We see typical transformation rates of $\pm 2 \times 10^{-2}$ Sv\,km$^{-1}$, but reaching as high $\pm 4$ Sv when integrated over latitude bands. Over longer time scales these transformations imply the formation of deep waters from intermediate waters. A key limitation of these water mass transformation calculations is the viscosity dependence of the overturning cells generated by the instability. In the ocean we expect a lower limit of their vertical extent to be around 15~m, which would likely alter the formation rates we have calculated in this study. Furthermore,~\citet{Yankovsky2019} find that the mixing of waters by symmetric instability is largely adiabatic and that secondary shear instabilities are what cause irreversible mixing. Achieving a high enough resolution with the three-dimensional models presented here may be difficult, however, a modified two-dimensional model may prove useful for obtaining better estimates.

Mixing aside, the key finding of this work is that symmetric instability is an efficient mechanism for removing anomalous potential vorticity originating from the opposite hemisphere in cross-equatorial flows. This work has focussed on currents dynamically similar to the North Brazil Current, however, the findings may also apply to other surface intensified cross-equatorial flows, such as the Somali current, an intense, seasonal, wind-driven western boundary current in the Indian Ocean. In the next chapter, we will explore the excitement of symmetric instability in the Deep Western Boundary Current, the AMOC's return flow, as it crosses the equator. We will compare what is seen at depth, to what we have seen in this chapter, and explore the overturning cells generated by symmetric instabilities further.