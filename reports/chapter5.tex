\chapter{Ekman driven symmetric instability in a high latitude western boundary current}
\label{chap:5}
\begin{quote}
    \textit{I was wondering if it would be possible to request a further 60 kCUs?} --- Fraser Goldsworth
\end{quote}

In previous chapters we have explored the excitement of symmetric instability generated by the change in sign of planetary vorticity at the equator. We will now turn our attention to instabilities generated by changing the sign of potential vorticity. In particular, we will be exploring the role of symmetric instability in deep water formation in the high-lattitude Irminger Current. Here, Ekman pumping by down-front winds leads to the steepening of isopycnals and the subsequent generation of negative potential vorticity. OSNAP observations suggest that symmetric instability is excited as a result. In what follows we will use a model to try and answer questions that these observations can not --- namely how important is symmetric instability in the transformation of water masses in the region.

\section{Water mass formation in the Irminger Sea}
The Irminger sea is a something something just off Greenland.
It has recently been revealed by OSNAP observations to be an important region in the formation of dense North Atlantic Deep Waters which make up the lower limb of the AMOC \citet{Lozier2019}. This finding came as a surprise to many as the majority of the deep water formation in models is seen in the adjacent Labrador Sea. As such there has been a renewed interest in processes which may enhance deep water formation in the Western Sub-polar North Atlantic Ocean.

The Irminger Current is something I will describe in this paragraph when I can get access to the books I need which are paywalled.

The established view that North Atlantic Deep Waters form as a result of buoyancy forcing --- predominantly air-sea heat fluxes. A recent study by \citet{LeBras2022} challenges this view. They study the effect of strong down-front winds which blow along the Irminger Current, and estimate that the Ekman buoyancy flux caused by these winds can be as much as four times stronger than the buoyancy flux resulting from heat loss to the atmosphere (the mechanism by which winds are able to extract buoyancy from a flow are described in section~\ref{sec:EkmanInst}). The resulting buoyancy loss can lead to the excitement of symmetric instability which is capable of generating mixing, and deepening the mixed layer. The winds that drive these events are described now.

The idea of Ekman induced symmetric instability being an important mechanism in the formation of deep waters in the Sub-polar North Atlantic is not a new one. \citet{Straneo2002} propose that the wind-driven Ekman buoyancy flux over the Labrador Sea can be around a third of the air-sea buoyancy flux and that the symmetric instability should be taken into account when modelling deep water formation in the region. 

\citet{Spall2016} investigate the effect of down-front winds in an idealised model of the Irminger Current. They integrate both a two-dimensional and three-dimensional hydrostatic model, with $\Delta x = 500$~m and $\Delta z = 1$~m. They force the model with a uniform meridional wind stress which is ramped up over seven days, then held constant for the remaining thirteen days of the model integrations. In their models, they observe an Ekman buoyancy flux that sets the potential vorticity to near-zero alongside baroclinic instability. These two processes together act to produce water mass transformations, with baroclinic instability (which is only present in the three-dimensional models) greatly enhancing the transformation rates.

The work of \citet{LeBras2022} raises questions about how much water mass transformation is driven by down-front wind events, and whether these highly seasonal events could be a source a AMOC variability. These questions are incredibly difficult to answer with sparse observations, and so we propose the use of idealised models to fill these gaps. Such results could also be used to form the basis of a parameterisation for mixing induced by down-front wind events (although we will not attempt to do this here). The work of \citet{Spall2016} lays the foundations of addressing these questions, however, their study design means it is only able to partially answer them. Their hydrostatic models are too coarse to provide a truly reliable estimate of the mixing induced by symmetric instability. A non-hydrostatic model with a higher resolution is required to resolve the secondary shear instabilities which are known to be important in generating mixing.

Although \citet{Spall2016} set out to model the same barrier wind events investigated by \citet{LeBras2022}, they force their model with a wind-stress which is constant following the first seven days of integration. This means both potential vorticity and buoyancy are constantly being extracted from the flow, meaning their models will not reach a steady state, rather; instability will constantly be excited. This means estimates of mixing at later times in their integrations may be either overestimates or underestimates, depending upon whether the pre-conditioning by the wind-stress at earlier times enhances or suppresses subsequent mixing. To estimate the effect of a wind event on mixing, we must model it as just that --- an isolated event, with a wind-stress which is ramped up and down from some characteristic value over a characteristic period of time.

In section~\ref{sec:Irm2DMeth} we will describe an ensemble of 37 idealised two-dimensional numerical models of the Irminger Current we plan on using to address the above questions. In section~\ref{sec:IrmRef} we will examine in detail one of the ensemble members, before looking at the results from the whole ensemble in section~\ref{sec:IrmEns}. We will summarise the findings in section~\ref{sec:IrmConc}.

\begin{figure} 
    \centering
    \includegraphics{../figures/EnsembleICs.pdf}
    \caption{A camelid}
    \label{fig:EnsembleICs}
\end{figure}

\section{A two-dimensional model of the Irminger current}
\label{sec:Irm2DMeth}
In section~\ref{subsec:2DMethods} we saw a two-dimensional of the North Brazil Current and how it allowed us to probe symmetric instability in a `clean' environment, free from noise from baroclinic and barotropic instabilities. In this chapter we will use similar two-dimensional models to explore symmetric instability in the Irminger current.

We use a non-hydrostatic configuration of the MITgcm~\citep{Marshall1997} to integrate an idealised two-dimensional model of the Irminger current that is symmetric (periodic) in the along-stream direction. The horizontal domain is 150 km wide in the across stream direction and 500 m deep. The horizontal and vertical grid spacings are set to 25 m and 1 m respectively. The resolution is set to be high in order to ensure the Richardson number is sufficiently small that Kelvin-Helmholtz instabilities can be resolved as they are known to be important for obtaining reliable estimates of the amount of diapycnal mixing that is occurring~\citep{Griffiths2003a, Yankovsky2019}. The time step is set to 2 seconds and the model is integrated for a total of 21 days. The model is sited on an $f$-plane with $f$ set to $1.26 \times 10^{-4}$ s$^{-1}$, corresponding to a latitude of $60^\circ$N. 

At the surface, a rigid lid boundary condition is employed, with the lateral and bottom boundaries set to be free-slip. The model has sloping bathymetry, which can be seen in  figure~\ref{fig:EnsembleICs}b. The model is initialised in thermal wind balance, with the velocity and density profiles shown in figure~\ref{fig:EnsembleICs}b. Both of these profiles are based on observations from the OSNAP array \citet{LeBras2022}.

The model is forced using a time varying along stream wind-stress. The stress is spatially uniform and Gaussian in time, with the form
\begin{equation}
    \tau_y = \tau_0 e^{-\flatfrac{(t - t_{mid})^2}{2\delta_t^2}} \, .
\end{equation}
For the reference integration, $\tau_0 = - 0.5$ N\,m$^{-2}$, $\delta_t = 2.5$ days. For all the models $t_{mid}$ is set to $10.5$ days. The along-front wind stress leads to a steepening of isopycnals and reduction in potential vorticity which may eventually lead to the excitement of symmetric instability, and in extreme cases, pure gravitational instability. In order to investigate how different wind forcing affect water mass formation and transformation rates, we also integrate an ensemble of simulations using ten different values of $\tau_0$ ranging linearly from $0$ N\,m$^{-2}$ to $-0.75$ N\,m$^{-2}$ and five different values of $\delta_t$ ranging linearly from 0 days to 5 days. This gives a total of 37 different ensemble members.

As in the previous models, a linear equation of state is used, with a reference density of 1027 kg\,m$^{-3}$, a thermal expansion coefficient of $2 \times 10^{-4}$ K$^{-1}$ and no salinity tracer. The thermal diffusion coefficient is set to $1 \times 10^{-5}$ m$^2$\,s$^{-1}$. A second order-moment Prather advection scheme with a flux limiter is employed. Momentum dissipation is provided by an adaptive biharmonic Smagorinsky viscosity of and a vertical Laplacian viscosity of $4 \times 10^{-4}$ m$^2$\,s$^{-1}$.  

\section{The reference integration}
\label{sec:IrmRef}
For the reference integration $\tau_0$ is set to -0.5 N\,m$^{-2}$ and $\delta_t$ to 2.5 days. Figure DECIDEa shows how the wind stress in the model evolves over time. This wind stress is lower than the typical peak wind-stresses seen over the Irminger Current during Wintertime (typically around -2 N\,m$^{-2}$) but is not atypical for less extreme wind events. The period is COMPARE to period. Wind stresses in the ensemble of models considered here were not increased above $-0.75$ N\,m$^{-2}$ as the model time-step required to ensure stability made the integrations too computationally intensive to perform.

Figure~\ref{fig:EnsStandardPV} shows the potential vorticity in the reference integration after one week, two weeks and three weeks of model runtime. In panel (a) we see how the down-front wind stress has induced an Ekman transport of surface waters towards the shelf, leading to a steepening of isopycnals and generating unstable stratification at the surface. This makes picking apart the contributions to the mixing from gravitational instability and symmetric instability difficult. We counter this difficulty by asserting that gravitational instability \textit{is} symmetric instability and so no separation is necessary. In the surface 25 m or so we can see regions of negative potential vorticity, and just below these areas of low potential vorticity suggesting that mixing is already occurring.

\begin{figure} 
    \centering
    \includegraphics{../figures/run32PV.pdf}
    \caption{A camelid}
    \label{fig:EnsStandardPV}
\end{figure}

A week later, several days after the wind-forcing has peaked, we see this region of low potential vorticity now penetrates deeply, to the bottom of the coastal shelf. Some negative potential vorticity still remains, meaning the flow has not yet fully equilibrated.

Finally, after three weeks, we see near zero potential vorticity throughout the upper 75 m of the shelf and shelf-break region. The majority of the negative potential vorticity has now gone, implying the flow has equilibrated following the wind event and reached a new quasi-steady state. Note how much deeper the mixed layer of this new state is, along with the steepness of the isopycnals which implies a sharp increase in the shear of the current.

\begin{figure} 
    \centering
    \includegraphics{../figures/MLD32.pdf}
    \caption{A camelid}
    \label{fig:EnsStandardMLD}
\end{figure}

We will adopt a quantitative definition for the mixed layer depth, as the depth at which the density drops by $0.05$ kg\,m$^{-3}$ relative to the value at the surface. Figure~\ref{fig:EnsStandardMLD}b shows how the mean (solid line) and the maximum (dashed line) mixed layer depth evolves over time. We see the maximum mixed layer depth increases almost monotonically, reaching the base of the model after around 11 days. This suggests the model could have done with more vertical levels, however; this was not possible due to computational constraints. The mean mixed layer depth also undergoes a net deepening, from around 95 m to 150 m.

The definition of the mixed layer depth employed here is somewhat arbitrary and so in figure~\ref{fig:EnsStandardMLD}c we also show the mean and maximum depths of several isopycnals over time. They are labelled as class 0 to 3 and correspond to the lower boundaries of the different water mass classes shown in figure~\ref{fig:WaterClass} and defined in table~\ref{tab:Class}. We  see that for classes 1 \& 2 there is a large change in the maximum depth, with these isopycnals now reaching the base of the model. The changes in the mean depths are much more modest, however, focussing on classes 1 \& 2 we see there is a shallowing of the isopycnal depth as the wind event gets under way, followed by a deepening. This is caused by the Ekman transport forcing the upwelling of waters within these classes. The isopycnals then return to slightly deeper depths following the wind event as a result of diapycnal mixing altering the density structure.

\begin{figure} 
    \centering
    \includegraphics[width=\textwidth]{../figures/place-holder.jpeg}
    \caption{A camelid}
    \label{fig:WaterClass}
\end{figure}

\begin{table}[]
    \caption{Water mass classes}
    \label{tab:Class}
    \begin{tabular}{lccl}
        \hline
    \multirow{2}{*}{Class} & \multicolumn{2}{c}{$\gamma^n$ (kg\,m$^{-3}$)}  & \multirow{2}{*}{Name}     \\
                           & Lower boundary & Upper boundary &                           \\ \hline \hline
    0                      & 26.92          & None           & shelf waters              \\
    1                      & 26.98          & 26.92          & shelf-break waters        \\
    2                      & 27.05          & 26.98          & upper intermediate waters \\
    3                      & 27.1211        & 27.05          & lower intermediate waters \\
    4                      & None           & 27.1211        & deep waters               \\ \hline
    \end{tabular}
    \end{table}

\begin{figure} 
    \centering
    \includegraphics{../figures/WMT32.pdf}
    \caption{A camelid}
    \label{fig:EnsStandardWMT}
\end{figure}

The effect of mixing on the different water massclasses can be examined through an analysis of the model's volume budget. Figure~\ref{fig:EnsStandardWMT} which shows the change in volume of each class over time. We see that there is a huge increase in the amount of class 1 shelf-break waters and a decrease in the volumes of class 0, class 2 \& class 3 waters. We can see that the increase in class 2 waters is largely driven by densification of the lightest waters, and to some extent the lightening of dense waters that sit below. The class 4 deep waters remain relatively unmixed. An attempt was made to calculate the instantaneous rate of change of the volumes in each class (the water mass formation rate), however; the signals were too noisy to be able to draw any meaningful conclusions from them\footnotemark. Instead, in figure~\ref{fig:EnsStandardWMT}b we show the formation rate calculated by differentiating the volume anomaly over time. This gives a much smoother signal. We have expressed the units in Sv\,km$^{-1}$ --- this model is a two dimensional model, so to get an estimate of the transformation rates across the whole length of the Irminger Current, one should mulitply the value by $\sim 10^2$ km.
\footnotetext{Apart from, perhaps, that the variance of water mass formation rates is huge!}

Given that \citet{Spall2016} show that baroclinic instability enhances the water mass transformation produced by symmetric instability, it is pertinent to question how reliably the water mass transformation calculations in this model actually are. The ensemble of models is too computationally intensive to run over a large domain, however; we will attempt to run the reference model at the required resolution to compare the wmts. 

\begin{figure} 
    \centering
    \includegraphics{../figures/DeltaMLD.pdf}
    \caption{A camelid}
    \label{fig:DeltaMLD}
\end{figure}



\begin{figure} 
    \centering
    \includegraphics{../figures/EnsembelVolAnom.pdf}
    \caption{A camelid}
    \label{fig:EnsVolAnom}
\end{figure}

\begin{figure} 
    \centering
    \includegraphics{../figures/EnsembelPeakFormation.pdf}
    \caption{Maybe this would be better with envelopes for the standard deviation of the tendency over time.}
    \label{fig:EnsPkFormation}
\end{figure}

\section{An ensemble of models}
\label{sec:IrmEns}

\section{Implications \& conclusions}
\label{sec:IrmConc}