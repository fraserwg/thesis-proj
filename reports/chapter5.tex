\chapter{Ekman driven symmetric instability in the East Greenland Current}
\label{chap:5}
% \begin{quote}
%     \textit{I was wondering if it would be possible to request a further 60 kCUs?} --- Fraser Goldsworth
% \end{quote}

In previous chapters, we have explored the excitement of symmetric instability generated by the change in sign of planetary vorticity at the equator. We will now turn our attention to instabilities generated by changing the sign of the potential vorticity of a fluid. In particular, we will be exploring the role of symmetric instability in deep water formation in the high-latitude East Greenland Current.

In section~\ref{sec:IrmIntro} we will introduce the East Greenland Current and explore previous works looking at symmetric instability in the Sub-polar North Atlantic. In section~\ref{sec:Irm2DMeth} we will describe an ensemble of 37 idealised two-dimensional numerical models of the East Greenland Current that we use to quantify the water mass transformation rates induced by symmetric instability and secondary instabilities. In section~\ref{sec:IrmRef} we will examine in detail one of the ensemble members, before looking at the results from the whole ensemble in section~\ref{sec:IrmEns}. We will summarise our findings in section~\ref{sec:IrmConc}.

This work is very much still in progress and leaves a lot of questions unanswered due to time constraints. I believe that despite the work's infancy we are still able to draw meaningful conclusions which contribute to a better understanding of the mixing processes taking place in the East Greenland Current. It will form the basis of a paper written jointly with those observing symmetric instability in the Sub-polar North Atlantic.

\section{Water mass formation in the Irminger Sea}
\label{sec:IrmIntro}
The Irminger Sea is the region of the North Atlantic that sits between the East Coast of Greenland and the West Coast of Iceland. It has recently been revealed by OSNAP observations to be an important region in the formation of dense North Atlantic Deep Waters which make up the lower limb of the AMOC \citep{Lozier2019}. This finding came as a surprise to many, with most models suggesting deep water formation occurs mainly in the adjacent Labrador Sea. As such, there has been a renewed interest in processes that may enhance deep water formation in the Eastern Sub-polar North Atlantic Ocean.

The East Greenland Current is the western boundary current of the Irminger sea that flows southwards along the east coast of Greenland. It has a speed of around 20 cm\,s$^{-1}$ and transports 16 Sv of water~\citep{Talley2011ArcOce, Talley2011AtlOce}. The strong baroclinicity of the flow means the current is a hotbed of mesoscale eddy activity~\citep{Foldvik1988}. The eddies are much smaller than those seen in the Tropical Atlantic due to the smaller Rossby deformation radius in the region, with typical eddy length-scales of $\sim 10$~km as opposed to $\sim 200$~km as is the case in the North Brazil Current. This leads to interesting interactions between the mesoscale and sub-mesoscale, with the lengthscales over which the two regimes are active having significant overlap~\citep{Gula2022}.

The established view is that North Atlantic Deep Waters form as a result of surface buoyancy loss, predominantly through air-sea heat fluxes. A recent study by \citet{LeBras2022} challenges this view. They study the effect of strong down-front winds which blow along the East Greenland Current (often referred to as ``barrier wind events'') and estimate that the Ekman buoyancy flux induced by these winds can be as much as four times stronger than the buoyancy flux resulting directly from heat loss to the atmosphere (the mechanism by which winds can extract buoyancy from a flow is described in section~\ref{sec:EkmanInst}). The resulting buoyancy loss can lead to the excitement of symmetric instability which is capable of generating mixing and deepening the mixed layer. The winds that drive these Ekman buoyancy fluxes occur frequently during winter and spring, typically lasting around 3 days and exerting maximal stresses of up to 2 N\,m$^{-2}$ on the sea surface \citep{LeBras2022}.

\begin{figure} 
    \centering
    \includegraphics{../figures/EnsembleICs.pdf}
    \caption{(a) The Sub-polar North Atlantic. The red box highlights the region where the observations which inspired the initial conditions used in the model were taken. (b) The meridional velocity used to instantiate model integrations, overlain with the isopycnals which denote the lower boundaries of each of the water mass classes we use in subsequent analysis (see table~\ref{tab:Class}).}
    \label{fig:EnsembleICs}
\end{figure}

The idea of Ekman-induced symmetric instability being an important mechanism in the formation of deep waters in the Sub-polar North Atlantic is not a new one. \citet{Straneo2002} proposed that the wind-driven Ekman buoyancy flux over the Labrador Sea could be around a third of the air-sea buoyancy flux, and that symmetric instability should be taken into account when modelling deep water formation in the region. 

\citet{Spall2016} investigate the effect of down-front winds in an idealised model of the East Greenland Current. They integrate both two-dimensional and three-dimensional hydrostatic models, with a horizontal grid spacing of 500~m and a vertical grid spacing of 1~m. They force their models with a uniform meridional wind stress which is ramped up over seven days and then held constant for the remaining thirteen days of model integration. In their models, they observe an Ekman buoyancy flux that sets the potential vorticity to near zero, alongside baroclinic instability. These two processes together act to produce water mass transformations, with baroclinic instability (which is only present in the three-dimensional models) greatly enhancing the transformation rates.

The work of \citet{LeBras2022} raises questions about how much water mass transformation is driven by down-front wind events, and whether these highly seasonal events could be a source of AMOC variability. These questions are incredibly difficult to answer with sparse observations, and so we explore the use of idealised models to fill these gaps. Such results could also be used to form the basis of a parameterisation for mixing induced by down-front wind events (although we will not attempt to do this here). The work of \citet{Spall2016} lays the foundations for addressing these questions; however, their study design means it is only able to partially answer them. Their hydrostatic models are too coarse to provide a truly reliable estimate of the mixing induced by symmetric instability. A non-hydrostatic model with a higher resolution is required to resolve the secondary shear instabilities which are known to be important in generating mixing \citep{Taylor2009}.

Although \citet{Spall2016} set out to model the same barrier wind events investigated by \citet{LeBras2022}, they force their model with a wind stress which is held constant after the first seven days of model integration. This means both potential vorticity and buoyancy are constantly being extracted from the flow, meaning their models will not reach a steady state; rather, instability will constantly be excited. This means estimates of mixing at later times in their integrations may be either overestimates or underestimates, depending upon whether the pre-conditioning by the wind stress at earlier times enhances or suppresses subsequent mixing. To estimate the effect of a wind event on mixing, we must model it as just that --- an isolated event, with a wind stress which is ramped up and down to some characteristic value over a characteristic period of time.

\section{A two-dimensional model of the East Greenland current}
\label{sec:Irm2DMeth}
In section~\ref{subsec:2DMethods} we used a two-dimensional model of the North Brazil Current to probe symmetric instability in a ``clean'' environment, free from noise from baroclinic and barotropic instabilities. In this chapter, we will use similar two-dimensional models to explore symmetric instability in the East Greenland current.

We use a non-hydrostatic configuration of the MITgcm~\citep{Marshall1997} to integrate an idealised two-dimensional model of the East Greenland current that is symmetric (periodic) in the along-stream direction. The domain is 150~km wide in the horizontal cross-stream direction and 500~m deep. The horizontal and vertical grid spacings are set to 25~m and 1~m, respectively. The resolution is set to be high to ensure the Richardson number is sufficiently small that Kelvin-Helmholtz instabilities can be resolved, as they are known to be important for obtaining reliable estimates of the amount of diapycnal mixing that is occurring~\citep{Griffiths2003a, Yankovsky2019}. The time step is set to 2 seconds and the model is integrated for a total of 21 days. The model is sited on an $f$-plane with $f$ set to $1.26 \times 10^{-4}$ s$^{-1}$, corresponding to a latitude of $60^\circ$N. 

At the surface, a rigid lid boundary condition is employed, with the lateral and bottom boundaries set to be free-slip. The model has sloping bathymetry, which can be seen in  figure~\ref{fig:EnsembleICs}b. The model is initialised in thermal wind balance, with the velocity field and density profiles shown in figure~\ref{fig:EnsembleICs}b. Both of these fields are based on observations from the OSNAP array \citep{LeBras2022}.

The model is forced using a time-varying along stream wind-stress. The stress is spatially uniform and temporally Gaussian, taking the form
\begin{equation}
    \tau_y = \tau_0 e^{-\flatfrac{(t - t_{mid})^2}{2\delta_t^2}} \, ,
\end{equation}
where $\tau_0$ is the maximum wind stress, $t_mid$ is the time at which the wind stress peaks and $\delta_t$ is the duration of the wind event. For the reference integration, $\tau_0 = - 0.5$ N\,m$^{-2}$ and $\delta_t = 2.5$ days. For each of the model integrations $t_{mid}$ is set to $10.5$ days. The along-front wind stress leads to a steepening of isopycnals and reduction in potential vorticity which may eventually lead to the excitement of symmetric instability, and in extreme cases, pure gravitational instability. To investigate how different wind forcing affects water mass formation and transformation rates, we also integrate an ensemble of simulations using ten different values of $\tau_0$ ranging linearly from $0$ N\,m$^{-2}$ to $-0.75$ N\,m$^{-2}$ and five different values of $\delta_t$ ranging linearly from 0 days to 5 days. This gives a total of 37 different ensemble members.

As in the previous models, a linear equation of state is used, with a reference density of 1027 kg\,m$^{-3}$, a thermal expansion coefficient of $2 \times 10^{-4}$ K$^{-1}$ and no salinity tracer. The thermal diffusion coefficient is set to $1 \times 10^{-5}$ m$^2$\,s$^{-1}$. A second order-moment Prather advection scheme with a flux limiter is employed. Momentum dissipation is provided by an adaptive biharmonic lateral Smagorinsky viscosity of and a vertical Laplacian viscosity of $4 \times 10^{-4}$ m$^2$\,s$^{-1}$.  

\section{The reference integration}
\label{sec:IrmRef}
For the reference integration $\tau_0$ is set to $-0.5$~N\,m$^{-2}$ and $\delta_t$ to~2.5 days. Figure~\ref{fig:EnsStandardMLD}a shows how the wind stress in the model evolves. This wind stress is lower than the typical peak wind stresses seen over the East Greenland Current during Wintertime (typically around $-2$~N\,m$^{-2}$) but is not atypical for less extreme wind events. The duration is similar to the wind events seen in observations. Wind stresses in the ensemble of models considered here were not increased above $-0.75$ N\,m$^{-2}$ as the model time-step required to ensure stability made the integrations too computationally intensive to perform.

Figure~\ref{fig:EnsStandardPV} shows the potential vorticity in the reference integration, overlain with isopycnal contours, after one week, two weeks and three weeks of model runtime. In panel (a) we see how the down-front wind stress has induced an Ekman transport of surface waters towards the shelf, leading to a steepening of isopycnals and generating unstable stratification at the surface (where dense isopycnals lie over lighter ones). This will generate strong mixing in the region of unstable stratification, and some weaker mixing directly below it.

A week later, several days after the wind-forcing has peaked, we see the region of low potential vorticity now stretches to the bottom of the coastal shelf. Some negative potential vorticity remains, meaning the flow has not yet fully equilibrated. Between around 20 m and 50 m of depth the isopycnals become very steep. These waters are stable to gravitational instability but have small magnitude negative potential vorticity arising from the strong shear and baroclinicity of the flow. These waters are susceptible to symmetric instability and produce mixing in the waters below the convectively mixed layer.

Finally, after three weeks, we see near zero potential vorticity throughout the upper 75 m of the shelf and shelf-break region. The majority of the negative potential vorticity has now gone, implying the flow has equilibrated following the wind event and reached a new (almost) steady state. Note how vertical the isopycnals are in the mixed layer, well below the convectively mixed surface layer.

\begin{figure} 
    \centering
    \includegraphics{../figures/run32PV.pdf}
    \caption{Potential vorticity after (a) 1 week, (b) 2 weeks and (c) 3 weeks of model integration in the reference integration, overlain with isopycnals corresponding to the lower boundaries of density classes 0 to 3 (see table~\ref{tab:Class}).}
    \label{fig:EnsStandardPV}
\end{figure}

The net effect of the winds, the gravitational instability\footnotemark and the symmetric instability is to set the potential vorticity of the flow to zero by modifying its stratification. Figure~\ref{fig:EnsStandardStrat} shows snapshots of the vertical buoyancy gradient over time. We can see how regions of low vertical stratification in figure~\ref{fig:EnsStandardStrat}c line up remarkably well with the regions of low potential vorticity in~\ref{fig:EnsStandardPV}c. In figures~\ref{fig:EnsStandardStrat}a and b, we see a shallow reservoir of gravitationally unstable waters towards the surface of the model; however waters outside of this region possess near zero stratification after two weeks of integration and almost exactly zero stratification after three weeks. Initially, Ekman transport leads to a reduction in the stratification by steepening the isopycnals. Upon becoming sufficiently steep symmetric instability is excited. The excitement of symmetric instability leads to slantwise convection --- overturning along isopycnals. This, in turn, leads to the excitement of Kelvin Helmholtz instability in the across-stream plane, which produces diapycnal mixing. This mixing reduces the isopycnal slope until the potential vorticity is approximately zero. Due to the high latitude, isopycnals can become very steeply sloping before becoming symmetrically unstable\footnotemark, which means the vertical stratification of the neutral state is very low. Furthermore, the billows generated by Kelvin Helmholtz instability can become susceptible to further gravitational instabilities which act to erode the vertical stratification further.
\footnotetext{which is really just an extreme case of symmetric instability as we saw in chapter~\ref{chap:2}}
\footnotetext{If we assume $Q \approx f \partial_z b - \flatfrac{(\partial_x b)^2}{f}$, take $f \sim 10^{-4}$ and $\partial_z b \sim 10^6$, and note that $\tan\Phi_{iso} = \flatfrac{\partial_x b}{\partial_z b}$, we can show $\Phi_{iso}$ must be greater than approximately $5^\circ$ for symmetric instability to occur at this latitude. This is a 1 in 10 gradient which is incredibly steep in oceanic terms.}

\begin{figure} 
    \centering
    \includegraphics{../figures/run32Strat.pdf}
    \caption{Stratification ($\partial_z b$) in the reference integration after (a) 1 week, (b) 2 weeks and (c) 3 weeks of model integration, overlain with isopycnals corresponding to the lower boundaries of density classes 0 to 3 (see table~\ref{tab:Class}).}
    \label{fig:EnsStandardStrat}
\end{figure}

In all, this makes discerning the contributions to mixing from the different ``flavours'' of symmetric instability and its secondary instabilities difficult. \citet{Taylor2010} talk of low potential vorticity layers with near zero potential vorticity but non-zero stratification being generated by wind events such as these --- the low potential vorticity layer being distinct from, and deeper than the convectively mixed layer. Their model resolution is a factor of ten higher than ours in the horizontal, allowing them to better resolve the steeply slanted isopycnals which give a small but non-zero vertical stratification. A key limitation of our model (and the ensemble) is that they struggle to differentiate the low potential vorticity layer from the convectively mixed layer, with both appearing to have near-zero stratification.

\begin{figure}
    \centering
    \includegraphics{../figures/StandardInstabilityRegion.pdf}
    \caption{Plot showing the total number of grid cells susceptible to (a) gravitational instability and (b) symmetric instability as a function of depth and time.}
    \label{fig:GravVsSym}
\end{figure}

Despite it being difficult to separate the low potential vorticity layer from the convectively mixed layer, we can still try to distinguish between the regions where gravitational instability and symmetric instability dominate the dynamics. Figure~\ref{fig:GravVsSym} shows the total number of grid points susceptible to gravitational and symmetric instabilities as a function of depth and time in the reference model. We see that gravitational instability is dominant in the surface layers. From around 20 m and lower, symmetric instability is more active. This ties in well with our interpretation of the potential vorticity, and the density contours seen in figure~\ref{fig:EnsStandardPV}.

\begin{figure} 
    \centering
    \includegraphics{../figures/MLD32.pdf}
    \caption{(a) Wind forcing, (b) mixed layer depth and the lower boundary of water mass class depths, (c) volume anomaly and (d) water mass formation rates from the reference integration. Solid lines in panel (b) represent zonally averaged depths and dashed lines maximum depths.}
    \label{fig:EnsStandardMLD}
\end{figure}

We will now consider the effect of both symmetric and gravitational instabilities on the depth of the well-mixed layer. We will define the mixed layer depth as the depth at which the density drops by $0.05$ kg\,m$^{-3}$ relative to the value at the surface. This layer typically includes both the upper convectively mixed layer and the deeper low potential vorticity layer. Whenever we refer to the mixed layer hereafter, we mean this well-mixed region. Figure~\ref{fig:EnsStandardMLD}b shows how the mean (solid line) and the maximum (dashed line) mixed layer depth evolves. We see the maximum mixed layer depth increases almost monotonically, reaching the base of the model after around 11 days. This suggests it may have been useful to increase the model's maximum depth; however, this was not possible due to computational constraints. The mean mixed layer depth also undergoes a net deepening, from around 95 m to 150 m.

The definition of the mixed layer depth employed here is somewhat arbitrary and so in figure~\ref{fig:EnsStandardMLD}c we also show the mean and maximum depths of several isopycnals over time. They are labelled as classes 0 to 3 and correspond to the lower boundaries of the different water mass classes shown in figure~\ref{fig:WaterClass} and defined in table~\ref{tab:Class}. We see that for classes 1 \& 2 there is a large change in the maximum depth, with these isopycnals now reaching the base of the model. The changes in the mean depths are much more modest, however. Focussing on classes 1 \& 2 we see there is a shallowing of the isopycnal depth as the wind event gets underway, followed by isopycnal deepening. This is caused by the Ekman transport forcing the upwelling of waters within these classes. The isopycnals then return to slightly deeper depths following the wind event as a result of diapycnal mixing altering the density structure.

\begin{table}[tp]
    \caption{Density bands use to define Water mass classes 0 to 4. Water mass class definitions are based on the areas they occupy at model instantiation.}
    \label{tab:Class}
    \begin{tabular}{lccl}
        \hline
    \multirow{2}{*}{Class} & \multicolumn{2}{c}{$\gamma^n$ (kg\,m$^{-3}$)}  & \multirow{2}{*}{Name}     \\
                           & Lower boundary & Upper boundary &                           \\ \hline \hline
    0                      & 26.92          & None           & shelf waters              \\
    1                      & 26.98          & 26.92          & shelf-break waters        \\
    2                      & 27.05          & 26.98          & upper intermediate waters \\
    3                      & 27.1211        & 27.05          & lower intermediate waters \\
    4                      & None           & 27.1211        & deep waters               \\ \hline
    \end{tabular}
\end{table}

\begin{figure}[p]
    \centering
    \includegraphics{../figures/WaterMassClass.pdf}
    \caption{Water mass classes at model instantiation.}
    \label{fig:WaterClass}
\end{figure}

The effect of mixing on the different water mass classes can be examined through an analysis of the model's volume budget. Figure~\ref{fig:EnsStandardMLD}c shows the change in volume of each water mass class over time. We see that there is a huge increase in the amount of class 1 shelf-break waters and a decrease in the volume of waters in classes 0, 2~\&~3. We can see that the increase in class 2 waters is largely driven by densification of the lightest waters, and to some extent the lightening of dense waters that sit below. The class 4 deep waters remain relatively unmixed. An attempt was made to calculate the instantaneous rate of change of the volumes in each class (the water mass formation rate); however, the time-series were too noisy to be able to draw any meaningful conclusions. Instead, in figure~\ref{fig:EnsStandardMLD}d we show the formation rate calculated by differentiating the volume anomaly over time. This gives a much smoother signal. We have expressed the units in Sv\,km$^{-1}$ --- this model is a two-dimensional model, so to get an estimate of the transformation rates across the whole length of the East Greenland Current, one should multiply the value by $\sim 10^2$ km.

We see that the water mass formation rates are greatest in classes 1 \& 0 --- this is unsurprising given they are in the closest proximity to the forcing. In the lightest two classes the formation rates first peak at a similar time to the wind-forcing, whereas in the deeper classes, there is a small lag of around a day or so. There is then a subsequent peak at around 12.5 days in the formation rates of the class 1 \& 0 waters, produced by the delayed mixing with water masses in the densest classes. The message from the volume anomaly and formation rate time series is clear, however --- symmetric instability and subsequent secondary instabilities are leading to densification of shelf waters and lightening of intermediate waters by mixing them with shelf-break waters. This leads to an increase in the volume of waters in the shelf-break density class.

Given that \citet{Spall2016} show that baroclinic instability enhances the water mass transformation produced by symmetric instability, it is pertinent to question how reliable the water mass transformation calculations in this model are; with two-dimensional models unable to represent baroclinic instability. The ensemble of simulations is too computationally expensive to run over a fully three-dimensional domain; however, in future work, it would be interesting to examine the water mass transformation rates in a three-dimensional configuration of the reference integration.

\section{An ensemble of model simulations}
\label{sec:IrmEns}
The ensemble of model simulations was forced with wind-stresses ranging in strength from 0~N\,m$^{-2}$ to~0.75 N\,m$^{-2}$ and with a duration of between 0 days and 5 days. It is useful to define the time-integrated wind stress which is given by
\begin{equation}
    \tau_{int} = \int^{t_{end}}_{t=0} \tau_y(t; \tau_0, \delta_t)\dd t \,
\end{equation}
and has units of N\,s\,m$^{-2}$. This is a measure of the total wind-forcing experienced by an ensemble member and combines the original two dependent variables into a single scalar.

For each ensemble member, we calculated the change in the mean mixed layer depth\footnotemark\ between the start and end of the integration. This is plotted as a function of the time-integrated wind stress in figure~\ref{fig:DeltaMLD}. Changes in the mixed layer depth behave largely as expected --- the larger the integrated wind stress the deeper the mixed layer. Interestingly, the change in mixed layer depth depends on both the maximal wind stress and the wind event duration. There are signs that for a given wind event duration there is a limit on the change in mixed layer depth, with increasing wind strengths leading to no further changes. This suggests the duration of a wind event is a primary control of how deep mixing can occur. Verification of this requires modelling with even larger wind stresses, which is computationally challenging. Furthermore, with the mixed layer often stretching to the model floor (see e.g. figure~\ref{fig:EnsStandardMLD}b), bathymetry may be affecting the relationship.
\footnotetext{Defined as before, as the depth at which the density changes by $0.05$ kg\,m$^{-3}$ relative to the surface density.}

\begin{figure} 
    \centering
    \includegraphics{../figures/DeltaMLD.pdf}
    \caption{Change in the mixed layer depth from the start to the end of each model integration, plotted as a function of integrated wind stress. Shapes and colours of markers identify the duration of the wind event.}
    \label{fig:DeltaMLD}
\end{figure}

Next, we look at the volume anomaly in each of the previously defined water mass classes, at the end of each model integration. Figure~\ref{fig:EnsVolAnom} shows this plotted as a function of the time-integrated wind stress. We can see that, generally, the integrated wind stress is a reasonable predictor of the volume anomaly in each class. The primary exceptions to this are the high-stress five-day wind events. During these events, the volume anomalies in classes 2 \& 3 deviate from the patterns seen in the shorter duration wind events. The longer wind events produce deeper mixing, which means the winds generate more mixing between the denser (and deeper) class 2 \& class 3 water masses.

\begin{figure} 
    \centering
    \includegraphics{../figures/EnsembleVolAnom.pdf}
    \caption{Volume anomaly at the end of each model integration for water mass classes 0 to 4 (a-e), plotted as a function of integrated wind stress. Shapes and colours correspond to different wind durations.}
    \label{fig:EnsVolAnom}
\end{figure}

In figure~\ref{fig:EnsPkFormation}, we plot the maximum and minimum water mass formation rate as a function of the time-integrated wind stress, for each water mass class in each ensemble member. The colours and shapes of the markers indicate the duration of the wind events, with filled markers corresponding to maximal formation rates and unfilled markers corresponding to minimal formation rates. In class 0 the maximum formation rate is approximately zero. For wind stresses of above 2~N\,s\,m$^{-2}$, in classes 0 and 1, the minimum and maximum formation rates, respectively, saturate at around $1.5 \times 10^{-2}$~Sv\,km$^{-1}$. This implies a maximum mixing rate between these two water mass classes exists. In class 1, the minimum formation rate is approximately zero for all integrated wind stresses. The exception to this is for large wind stresses during five-day wind events. During these events, the minimum formation rate becomes large and negative, and there is a similar uptick in the maximum formation rate of class 2 waters. This implies that during longer wind events there is a transformation between the class 1 \& 2 waters, reinforcing the idea that longer wind events lead to the mixing of deeper waters.

In general, formation rates of class 2 waters are much more modest, as would be expected given their reduced proximity to the source of the buoyancy forcing. The main exception to this is the negative formation rate at high integrated wind-stresses, balanced by the formation of class 1 \& class 3 waters. At high wind stresses during the longer wind events, there are large formation rates of class 3 waters, which saturate at approximately $2 \times 10^{-2}$~Sv\,km$^{-1}$. Regardless of wind stress, the formation rates of class 4 waters are close to zero, as expected for deep waters.

\begin{figure} 
    \centering
    \includegraphics{../figures/EnsemblePeakFormation.pdf}
    \caption{Maximum and minimum water mass formation rates for water mass classes 0 to 4 (a-e) plotted as a function of integrated wind stress. Filled shapes give maximum and unfilled shapes give minimum water mass formation rates. The duration of the winds is given by the shape and colour of the points.}
    \label{fig:EnsPkFormation}
\end{figure}


\section{Summary}
\label{sec:IrmConc}
The East Greenland Current is the western boundary current of the Irminger Sea, which is one of the key regions where North Atlantic Deep Waters are formed \citep{Lozier2019}. There is observational evidence of symmetric instability being excited in the current when strong northerly winds blow along it~\citep{LeBras2022}. These winds induce an Ekman Transport which acts to steepen isopycnals, making the potential vorticity negative~\citep[e.g.][]{Straneo2002}.

The observations show a significant deepening of the mixed layer~\citep{LeBras2022}, however, water mass formation rates are particularly challenging to diagnose from spatially sparse mooring observations. In this chapter we used an ensemble of idealised two-dimensional models of the East Greenland Current to diagnose the changes in the mixed layer depth, changes in the volume of different water mass classes, and the rate of water mass formation within these classes. We forced each ensemble member with temporally Gaussian wind stresses of different durations and peak stresses. The resolution of the models was set to 25 m to ensure the adequate resolution of secondary Kelvin-Helmholtz instabilities which are known to be important in producing irreversible mixing~\citep{Yankovsky2019, Griffiths2003a}. A two-dimensional model was employed to keep the computational resources required by the models affordable\footnotemark.
\footnotetext{The ensemble required around 60 kCUs to run on the Archer2 HPC facility. A typical annual allocation of resources is around 25 kCUs for individual users.}

In a reference integration, we saw the effect of down-front winds on the isopycnal and potential vorticity structure of the current. In the surface 20~m we saw the formation of a convectively mixed layer with zero stratification. Below this a low potential vorticity layer formed, which extended much deeper. Due to the high latitude, the stratification in a zero potential vorticity flow is close to zero which makes distinguishing between the convectively mixed layer and the zero potential vorticity layer difficult. Due to the dominant effect of symmetric instability on the stratification, however, it may be that existing mixed layer parameterisations of gravitational instability, such as KPP, may be sufficient to adequately reproduce the density structure formed by unresolved instabilities and mixing in coarse resolution models. Comparing the results of high-resolution symmetric instability resolving models with coarser models using KPP or symmetric instability specific parameterisations would be a good first step in assessing whether this approach is sensible.

Across all integrations in the ensemble, we saw how large volume anomalies in the surface water mass classes could be generated by instantaneous water mass formation rates of up to around 1 Sv when integrated along a 100~km long stretch of the current. The wind events causing these large water mass formation rates are, however, highly seasonal and only occur several times a year during winter and spring. When averaged out over a year, these formation rates are much less substantial. This leads to the question of how significant these water mass formation rates are. We suggest that they may be important in explaining some high-frequency variability in the AMOC during winter and spring times. We also hypothesise that the deepening of the mixed layer by the wind events may be important in pre-conditioning waters before they undergo subsequent density transformations.

We have seen how under some circumstances the time-integrated wind stress can act as a good predictor for the water mass formation rates in different density classes as well as the volume anomalies within these classes. The integrated wind stress was, however, particularly poor in predicting mixed layer deepening. The duration of wind events appeared to set a limit on the maximum depth of the mixed layer, with longer wind events producing deeper mixing for a given integrated wind stress. This result is reflected in the ensemble volume anomalies and water mass formation rates, as well as the mixed layer depths. A more appropriate predictor than wind stress may be the time-integrated Ekman buoyancy flux. The buoyancy flux is given by
\begin{equation}
    B_{wind} = -\frac{\tau_y \partial_x b}{\rho_0 f} \, ,
\end{equation}
and is commonly used in other studies. We have not used this measure here due to time constraints.

The finding that baroclinic instability plays an important role in generating mixing during down-front wind events over currents similar to the East Greenland Current \citep{Spall2016} remains a real limitation on the conclusions we have just drawn: baroclinic instability cannot be resolved in two-dimensional models, such as those used in this study.
Our setup, however, differed from that of~\citet{Spall2016} as we used time-varying rather than constant winds to force our models. We hypothesise that during (short) wind events, symmetric instability is the predominant driver of mixing and restratification in the surface mixed layer. This leads to the generation of a deep mixed layer in which $\partial_z b$ is near-zero and $\partial_x b$ is large. The pre-conditioning of the mixed layer in this way creates a density structure in which the growth rate of baroclinic instability is greatly enhanced~\citep{Eady1949}. OSNAP observations coupled with further idealised modelling efforts would be well suited to help test this hypothesis.