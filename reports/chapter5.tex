\chapter{Ekman driven symmetric instability in a high latitude western boundary current}
\begin{quote}
    \textit{Brrrrrrr} --- Fraser Goldsworth
\end{quote}

In previous chapters we have explored the excitement of symmetric instability generated by the change in sign of planetary vorticity at the equator. We will now turn our attention to instabilities generated by changing the sign of potential vorticity. In particular, we will be exploring the role of symmetric instability in deep water formation in the high-lattitude Irminger Current. Here, Ekman pumping by down-front winds leads to the steepening of isopycnals and the subsequent generation of negative potential vorticity. OSNAP observations suggest that symmetric instability is excited as a result. In what follows we will use a model to try and answer questions that these observations can not --- namely how important is symmetric instability in the transformation of water masses in the region.

\section{Irminger Current}
pass
\section{A two-dimensional model of the Irminger current}
In section~\ref{subsec:2DMethods} we saw a two-dimensional of the North Brazil Current and how it allowed us to probe symmetric instability in a `clean' environment, free from noise from baroclinic and barotropic instabilities. In this chapter we will use similar two-dimensional models to explore symmetric instability in the Irminger current
\subsection{Model description}
We use a non-hydrostatic configuration of the MITgcm~\citep{Marshall1997} to integrate an idealised two-dimensional model of the Irminger current that is symmetric (periodic) in the along-stream direction. The horizontal domain is 150 km wide in the across stream direction and 500 m deep. The horizontal and vertical grid spacings are set to 25 m and 1 m respectively. The resolution is set to be high in order to ensure the Richardson number is sufficiently small that Kelvin-Helmholtz instabilities can be resolved as they are known to be important for obtaining reliable estimates of the amount of diapycnal mixing that is occurring~\citep{Griffiths2003a, Yankovsky2019}. The time step is set to 2 seconds and the model is integrated for a total of 21 days. The model is sited on an $f$-plane with $f$ set to $1.26 \times 10^{-4}$ s$^{-1}$, corresponding to a latitude of $60^\circ$N. 

At the surface, a rigid lid boundary condition is employed, with the lateral and bottom boundaries set to be free-slip. The model has sloping bathymetry, which can be seen in  figure~\ref{fig:ICICs}X. The model is initialised in thermal wind balance, with the velocity and density profiles shown in figure~\ref{fig:ICICs}X. Both of these profiles are based on observations from the OSNAP array~/cite{IsabelaUnPublished}.

The model is forced using a time varying along stream wind-stress. The stress is spatially uniform and Gaussian in time, with the form
\begin{equation}
    \tau_y = \tau_0 e^{-\flatfrac{(t - t_{mid})^2}{2\delta_t^2}} \, .
\end{equation}
For the reference integration, $\tau_0 = - 0.75$ N\,m$^{-2}$, $\delta_t = 1.25$ days and $t_{mid} = 10.5$ days. The along-front wind stress leads to a steepening of isopycnals and reduction in potential vorticity which may eventually lead to the excitement of symmetric instability, and in extreme cases, pure gravitational instability. In order to investigate how different wind forcing affect water mass formation and transformation rates, we also integrate an ensemble of simulations using ten different values of $\tau_0$ ranging linearly from $0$ N\,m$^{-2}$ to $-0.75$ N\,m$^{-2}$ and five different values of $\delta_t$ ranging linearly from 0 days to 5 days. This gives a total of 37 different ensemble members.

\begin{figure} 
    \centering
    \includegraphics{../figures/EnsembleICs.pdf}
    \caption{A camelid}
    \label{fig:EnsembleICs}
\end{figure}

As in the previous models, a linear equation of state is used, with a reference density of 1027 kg\,m$^{-3}$, a thermal expansion coefficient of $2 \times 10^{-4}$ K$^{-1}$ and no salinity tracer. The thermal diffusion coefficient is set to $1 \times 10^{-5}$ m$^2$\,s$^{-1}$. A second order-moment Prather advection scheme with a flux limiter is employed. Momentum dissipation is provided by an adaptive biharmonic Smagorinsky viscosity of and a vertical Laplacian viscosity of $4 \times 10^{-4}$ m$^2$\,s$^{-1}$.  

\section{The reference integration}



\begin{figure} 
    \centering
    \includegraphics{../figures/run32PV.pdf}
    \caption{A camelid}
    \label{fig:EnsStandardPV}
\end{figure}

\begin{figure} 
    \centering
    \includegraphics{../figures/WMT32.pdf}
    \caption{A camelid}
    \label{fig:EnsStandardWMT}
\end{figure}

\begin{figure} 
    \centering
    \includegraphics{../figures/MLD32.pdf}
    \caption{A camelid}
    \label{fig:EnsStandardMLD}
\end{figure}

\begin{figure} 
    \centering
    \includegraphics{../figures/DeltaMLD.pdf}
    \caption{A camelid}
    \label{fig:DeltaMLD}
\end{figure}

\begin{figure} 
    \centering
    \includegraphics{../figures/EnsembelVolAnom.pdf}
    \caption{A camelid}
    \label{fig:EnsVolAnom}
\end{figure}

\begin{figure} 
    \centering
    \includegraphics{../figures/EnsembelPeakFormation.pdf}
    \caption{Maybe this would be better with envelopes for the standard deviation of the tendency over time.}
    \label{fig:EnsPkFormation}
\end{figure}

\subsection{Water mass transformation in the model}

\section{An ensemble of models}

\section{Implications \& conclusions}