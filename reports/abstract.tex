\cleardoublepage
\phantomsection
\addcontentsline{toc}{chapter}{Abstract}
\begin{abstractlong}
    \begin{spacing}{1.25}
    The role of symmetric instability in mixing momentum, buoyancy and potential vorticity is examined in several constituent currents of the Atlantic Meridional Overturning Circulation: the North Brazil Current and its rings, the Deep Western Boundary Current in the Tropical Atlantic, and the East Greenland Current of the Sub-polar North Atlantic.

    Linear stability theory predicts the structure and growth rates of overturning cells generated by symmetric instability in cross-equatorial western boundary current systems. Stratification, mixing, and curvature in the velocity field stabilise these symmetrically unstable motions. A measure of the importance of the non-traditional component of the Coriolis force in modifying symmetrically unstable motions is derived.

    Idealised models of the North Brazil Current system are used to assess predictions from the linear stability analysis. They show symmetric instability is effective at neutralising negative potential vorticity originating from the Southern Hemisphere, enabling large-scale interhemispheric water exchange. Symmetric instability drives $\sim 4 \times 10^6$~m$^3$\,s$^{-1}$ of water mass transformation between intermediate and deep waters in this system.

    Idealised models of the Deep Western Boundary Current demonstrate that symmetric instability remains an efficient mechanism for mixing potential vorticity despite dynamical differences to the North Brazil Current. Overturning cells produced by the instability generate density staircases. These provide a new diagnostic for the detection of symmetric instability.

    Finally, an ensemble of idealised models of the East Greenland Current are forced with varying down front winds, which generate Ekman buoyancy fluxes. These fluxes deepen the convectively mixed layer and through the excitement of symmetric instability create a deeper low potential vorticity layer below it. This layer has potential vorticity of approximately zero and near zero stratification.

    Symmetric instability is an efficient mechanism for mixing momentum, buoyancy and potential vorticity in both tropical and high-latitude components of the Atlantic's overturning circulation.
    \end{spacing}

\end{abstractlong}