\chapter{Conclusions}

This work set out to investigate the role of symmetric instability in different western boundary current systems, which act as important pathways for waters participating in the global-scale Atlantic Meridional Overturning Circulation. 

Although chapter 2 largely laid the groundwork for what followed, it also contained several new results that are worth remarking upon here. The chapter was a demonstration of the power of analytical theory, in which theory was applied to predict the character of symmetric instability in the North Brazil Current and its rings, and the deep western boundary current, as they cross the equator. Although some strong assumptions were made in the application of theory to these real-world currents, the theory's predictions were invaluable in identifying symmetric instability in the more complex numerical models subsequently presented. I see great potential in the use of simple numerical models, like those developed here, in diagnosing symmetric instability in observations and more-realistic General Circulation Model outputs. Using linear theory we were also able to challenge the assumption that the complete Coriolis force would modify the properties of symmetric instability in flows such as the North Brazil Current as they cross the equator. Building on the work of \citet{Zeitlin2018a} we were able to create a new measure of the importance of the complete Coriolis force on a flow, which takes into account its direction. We also defined a class of forces that leave the structure and growth rates of symmetric instabilities unchanged --- this has led to a greater understanding of the kinds of forces that should be considered when studying symmetric instabilities.

In chapter 3 we examined symmetric instability in the North Brazil Current, examining the instability's role in changing the sign of potential vorticity of Southern Hemisphere waters and in generating diapycnal mixing. Through the use of high-resolution modelling, we were able to demonstrate the effect of symmetric instability in setting the absolute vorticity of the flow to zero. Most General Circulation Models change the sign of the potential vorticity of Southern Hemisphere Waters as they cross the equator through lateral dissipation provided by an eddy viscosity \citep{Edwards1998I}. That the effect of the instability is to set the absolute vorticity of the flow to zero suggests that, to first order, such a viscosity may parameterise the process adequately within coarse resolution models. This is further backed up by the observation of limited net water mass transformation produced by the excitement of symmetric instability in our idealised models, implying the amount of mixing generated by the instability is small. A caveat to this is that this idealised model is likely underestimating the transformation rates due to the resolution being too coarse to resolve secondary Kelvin-Helmholtz instabilities which are known to be important in the generation of water mass transformation by symmetric instability \citep[e.g.][]{Yankovsky2019, Griffiths2003a}.

In chapter 4 we examined the excitement of symmetric instability in the Deep Western Boundary Current as it crossed the equator. We showed that despite the growth rate of symmetric instability in the current being so small (due to the smallness of both the planetary vorticity and the relative vorticity of the current) symmetric instability is still excited and sets the potential vorticity of the flow to zero. If this prediction of the presence of symmetric instability in the Deep Western Boundary Current is borne out, it would be one of the first examples of symmetric instability occurring away from either the surface or bottom boundary layer of the ocean. This work also predicts the latitudes at which we expect the instability to be most active, which may aid in searching for observational evidence of its excitement. We also demonstrate that symmetric instability (and any overturning instability in a stratified flow) can lead to the generation of density staircases. A clear next step in searching for observational evidence of symmetric instability in the Deep Western Boundary Current would be to analyse existing CTD casts in the region which may show step-like features with a length scale of between approximately 10~m to 250~m. Furthermore, the linear theory of chapter 2 demonstrates the link between the vertical step height of the staircase (set by the height of the overturning cells) and the rate at which secondary instabilities dissipate momentum --- as parameterised by a vertical viscosity. Applying the linear theory to observed currents as an inverse-type model would enable an estimation of the viscosity which is best used for parameterising the secondary shear instabilities which equilibrate unstable flows. Such a viscosity may serve as a crude first-order parameterisation of Kelvin-Helmholtz instability in symmetric instability permitting General Circulation Models.

Finally, in chapter 5 we investigated the role of wind-induced symmetric instability in the East Greenland Current. We showed that strong down-front wind events have a significant impact on the mixed layer depth and generate large amounts of diapycnal mixing. Baroclinic instability is not present in the idealised models used to study the process, so it remains to be seen how significant the enhancement of the mixing by baroclinic instability is \citep{Spall2016}. That the winds can generate up to 2 Sv of mixing (when integrated over the length of the current) suggests that the instability explains some short time-scale variability seen in AMOC observations. Despite the significant amounts of instantaneous diapycnal mixing generated by down-front wind events in our models, it is clear that the wind events themselves are not frequent enough for the amount of diapycnal mixing generated to account for more than a few tenths of a Sverdrup of deep water formation when integrated over the length of the current and over a 12 month period. However, this is not to say that the process is insignificant. The deepening and restratification of the mixed layer during these wind events may be important in pre-conditioning surface waters, ready for other processes to transform them into North Atlantic Deep Waters --- this should be investigated further. There are also hints that for large wind-stresses, changes in mixed layer depth may saturate.

Significantly, chapter 5 suggests that a simple mixed layer parameterisation of gravitational instability, such as KPP, may adequately represent the mixing produced by Ekman-induced symmetric instability in the East Greenland Current. The effectiveness of such a parameterisation should be evaluated and compared to the results we present here, as should the effectiveness of parametrisations specifically targeted at representing the effects of symmetric instability. The KPP parameterisation is already included in many Climate and General Circulation Models, meaning minimal work would be required to ensure we are adequately representing the re-stratifying effects of down-front winds in these types of models.

This thesis made heavy use of many pieces of open-source software, and although it was not one of our primary aims, I believe this thesis clearly demonstrates the value open-source software brings to science. Given studies showing that the vast majority of scientists use open-source software without contributing, I finish with an open and personal question for you, the reader: how can I, as a scientist, add more value to the open-source projects on which my research depends?


%Buoyancy flux is given by a simple relation between winds and lateral surface density gradients. We can use that to calculate the buoyancy in the surface that should be (despite the absence of the isopycnal steepening). We then apply a vertical mixing scheme --- perhaps we need to evaluate the PV though?

%Mixed layer and its role in transferring heat into the deep ocean and as a storage heater. Deeper mixed layer means more space for storing heat.

%Heat content of the mixed layer doesn't mean anything in our models as we have no salinity.