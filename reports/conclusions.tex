\chapter{Conclusions}
\begin{quote}
    \textit{You live and learn. At any rate, you live.} --- Douglas Adams
\end{quote}
This thesis set out to investigate the role of symmetric instability in different western boundary current systems, which act as important pathways for waters participating in the global-scale Atlantic Meridional Overturning Circulation. In particular, we focussed on the North Brazil Current and the Deep Western Boundary Current as they cross the equator, and the East Greenland Current during winter and spring-time barrier wind events.

In chapter~\ref{chap:1}, we examined the importance of the AMOC and saw how mixing in its many constituent currents is of fundamental importance to its large-scale dynamics. We learnt how symmetric instability can be an efficient way to generate sub-mesoscale motions capable of mixing buoyancy, momentum, potential vorticity and other tracers of interest in currents with potential vorticity opposite in sign to that of the planetary vorticity. We outlined how this mixing may go some way toward solving the ``cross-equatorial flow problem''. The crux of this problem is that in order for large-scale interhemispheric water exchanges to occur, there must exist a process capable of changing the sign of the potential vorticity of the waters crossing the equator \citet{Killworth1991}. The conservation of potential vorticity, however, places limits on the kinds of processes which may modify potential vorticity and requires the mechanism to move velocity and buoyancy gradients to small scales. The AMOC's climatic importance is partly derived from its role in facilitating the interhemispheric transport of waters and so solving this problem is of fundamental importance. Away from the equator, we saw how observations of symmetric instability in the East Greenland Current show that the instability is capable of producing significant deepening of the mixed layer there, but that these observations leave questions about the amount of diapycnal mixing the instability can generate unanswered.

This leads us to chapter~\ref{chap:2}, where we took an in-depth look at symmetric instability and its properties. We explored two different ways of exciting it --- by either changing the sign of the planetary vorticity or changing the sign of the potential vorticity of a fluid parcel. The first mechanism for generating symmetric instability is unique to currents which cross the equator. A water parcel stable to symmetric instability in one Hemisphere will not be stable to symmetric instability in the opposing Hemisphere. This instability can be generated outside the surface and bottom boundary layers where potential vorticity forcing occurs, making symmetric instability generated in this way distinct from instability generated by buoyancy or momentum forcing.

We then examined how Ekman buoyancy forcing by down-front wind events can generate anomalous potential vorticity and hence symmetric instability (although alternate methods of generating anomalous potential vorticity exist). Geostrophically balanced currents have outcropping isopycnals. When the wind blows along these currents, the surface waters experience Ekman transport perpendicular to the wind, which can act to steepen the isopycnals. If the winds are sufficiently strong this can lead to a change in sign of the potential vorticity. This mechanism has been observed to be at play in the East Greenland Current and in chapter~\ref{chap:5} we go about modelling the phenomenon.

In the following sections we explored where the symmetric instability criterion comes from, examined the orientation of the overturning cells generated by the instability, and looked at gravitational and inertial instabilities as limiting cases of symmetric instability --- symmetric instability can be interpreted as inertial instability along isopycnals or gravitational instability along vortex tubes. This led us to an examination of the energetics of the instability, which in turn informed a discussion of the ``classical'' and ``energetic'' definitions of symmetric instability. In this thesis, we employed the classical definition of symmetric instability, in which gravitational, inertial and slantwise convective instabilities are different aspects of symmetric instability.

These previous results were derived by considering a symmetric Eady type problem, that of a parallel shear flow. We then looked at the axisymmetric problem, which can be thought of as modelling the behaviour of eddies. We saw how rotation modifies the instability criterion, with the cores of anticyclonic eddies being stabilised by the additional centrifugal forces. Having reviewed the established theory of symmetric instability, we broke new ground, applying linear stability analyses to both the North Brazil Current and its rings and the Deep Western Boundary Current as they cross the equator. We used these results to make predictions about the timescales at play and the structures we may see when these currents cross the equator. Although some strong assumptions were made in the application of theory to these real-world currents, the theory's predictions were invaluable in identifying symmetric instability in the more complex numerical models subsequently presented. I see great potential in the use of simple numerical models, like those developed here, in diagnosing symmetric instability in observations and more-realistic General Circulation Model outputs. Using linear theory we were also able to challenge the assumption that the complete Coriolis force would modify the properties of symmetric instability in flows such as the North Brazil Current as they cross the equator. Building on the work of \citet{Zeitlin2018a} we were able to create a new measure of the importance of the complete Coriolis force on a flow, which takes into account its direction. We also defined a class of forces that leave the structure and growth rates of symmetric instabilities unchanged --- this has led to a greater understanding of the kinds of forces that should be considered when studying symmetric instabilities.

In chapter~\ref{chap:3}, we examined symmetric instability in the North Brazil Current, investigating the instability's role in ridding the current of anomalous potential vorticity originating in Southern Hemisphere waters and in generating diapycnal mixing. Through the use of high-resolution modelling, we were able to demonstrate the effect of symmetric instability in setting the absolute vorticity of the flow to zero. Most General Circulation Models neutralise the anomalous potential vorticity of Southern Hemisphere Waters as they cross the equator through lateral dissipation provided by an eddy viscosity \citep{Edwards1998I}. That the effect of the instability is to set the absolute vorticity of the flow to zero suggests that, to first order, such a viscosity may parameterise the process adequately within coarse resolution models. A caveat to this is that our model is likely underestimating water mass transformation rates which affect the density structure. This is due to the resolution being too coarse to resolve secondary Kelvin-Helmholtz instabilities which are known to be important in the generation of water mass transformation by symmetric instability \citep[e.g.][]{Yankovsky2019, Griffiths2003a}.

In chapter~\ref{chap:4} we examined the excitement of symmetric instability in the Deep Western Boundary Current as it crossed the equator. We showed that despite the growth rate of symmetric instability in the current being very small (due to the smallness of both the planetary vorticity and the relative vorticity of the current) symmetric instability is still excited and sets the potential vorticity of the flow to zero. If this prediction of the presence of symmetric instability in the Deep Western Boundary Current is borne out by observations, it will be one of the first examples of symmetric instability occurring away from either the surface or bottom boundary layers of the ocean. This work also predicts the latitudes at which we expect the instability to be most active, which may aid in searching for observational evidence of its excitement. We also demonstrate that symmetric instability (and any overturning instability in a stratified flow) can lead to the generation of density staircases. A clear next step in searching for observational evidence of symmetric instability in the Deep Western Boundary Current would be to analyse existing CTD casts in the region which may show step-like features with a length scale of between approximately 10~m to 250~m. Furthermore, the linear theory of chapter~\ref{chap:2} demonstrates the link between the vertical step height of the staircase (set by the height of the overturning cells) and the rate at which secondary instabilities dissipate momentum --- as parameterised by a vertical viscosity. Applying the linear theory to observed currents as an inverse-type model would enable an estimation of the viscosity which is best used for parameterising the secondary shear instabilities which equilibrate unstable flows. Such a viscosity may serve as a crude first-order parameterisation of Kelvin-Helmholtz instability in general circulation models of sufficient resolution to permit symmetric instability. The finding in chapter~\ref{chap:4} that symmetric instability is occurring much closer to the equator (around 25~km away) in the Deep Western Boundary Current than in the North Brazil Current (around 400~km away) is a great vindication of the power of linear theory. Simple scaling arguments were able to predict these numbers before any modelling had taken place.

Finally, in chapter~\ref{chap:5} we investigated the role of Ekman-driven symmetric instability in the East Greenland Current. We showed that strong down-front wind events have a significant impact on the depth of the upper convectively mixed layer and the deeper low potential vorticity layer through the excitement of symmetric instability, and the subsequent diapycnal mixing it generates. Baroclinic instability is not present in the idealised models used to study the process, so it remains to be seen how significant the enhancement of the mixing by baroclinic instability is \citep{Spall2016}.
That the winds can generate up to 2 Sv of mixing (when integrated over the length of the East Greenland Current) suggests that the instability may explain some short timescale overturning variability seen in AMOC observations at the OSNAP East array. Despite the significant amounts of instantaneous diapycnal mixing generated by down-front wind events in our model, it is clear that the wind events themselves are not frequent enough for the amount of diapycnal mixing generated to account for more than a few tenths of a Sverdrup of deep water formation when integrated over the length of the current and a 12-month period. However, this is not to say that the process is insignificant. The deepening and restratification of the mixed layer during these wind events may be important in pre-conditioning surface waters, ready for other processes to transform them into North Atlantic Deep Waters --- this should be investigated further. In particular, the role of baroclinic instability in producing further mixing should be investigated, although other mixed layer instabilities could also be important.

Significantly, chapter~\ref{chap:5} suggests that a simple mixed layer parameterisation of gravitational instability, such as KPP, may adequately represent the mixing produced by Ekman-induced symmetric instability in the East Greenland Current. This is due to the high latitude meaning the zero potential vorticity state is very close to a zero stratificaiton state. The effectiveness of such a parameterisation should be evaluated and compared to the results we present here, as should the effectiveness of parametrisations specifically targeted at representing the effects of symmetric instability. The KPP parameterisation is already included in many climate and general circulation models, meaning minimal work would be required to ensure we are adequately representing the re-stratifying effects of down-front winds in these types of models.

This thesis has examined the role of symmetric instability in three very different currents --- the surface intensified North Brazil Current, the much slower Deep Western Boundary Current as it crosses the equator, and the highly baroclinic East Greenland Current. Symmetric instabilities are often seen as sitting on a spectrum, with gravitational instability at one end and inertial instability at the other. We have explored both extremes of the spectrum in this work and, I hope, demonstrated how a classical definition of symmetric instability aids in understanding the relationship between these different phenomena. We have seen how symmetric instability plays a role in the mixing of buoyancy, momentum and potential vorticity in each of these currents. These three currents, however, are only a small subset of those which combine to form the AMOC. The role of submesoscale instability in many other currents which contribute to this global scale circulation is yet to be examined. Our understanding of the interior cross-equatorial pathways in the Tropical Atlantic remains incomplete --- in part due to the difficulty in making observations resulting from the breakdown of geostrophy. Our understanding of deep water formation sites and processes in the Sub-polar North Atlantic is increasing year on year, with unprecedented observations from the ONSAP array revealing the complexity of the circulations in the Sub-polar North Atlantic. In short, theory and models still have a way to go to provide a complete description of the AMOC. It is only by utilising observations when developing theory (as in chapter~\ref{chap:5}) or by utilising theory to inform observations (as I hope to with the staircases from chapter~\ref{chap:4}) that we will be able to answer the most pressing questions in the field of AMOC science. It is imperative that we both understand, and are able to adequately model the ocean's response to anthropogenic climate change. Small-scale processes are key to determining the amount of heat and carbon the ocean will take up, and the large-scale circulation will determine how these tracers are redistributed across the globe.
