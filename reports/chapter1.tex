\chapter{The global mixer}
\begin{quote}
\textit{The shape of the sea is always the same. Or rather it's always different, but in the same way.} --- Philip Marsden
\end{quote}

The Atlantic Meridional Overturning Circulation (AMOC) is a climatically important circulation that spans the entirety of the Atlantic Ocean basin. In its surface limb, it draws water from the Southern Hemisphere northward across the equator, redistributing climatically import tracers such as heat and carbon as it does so \citep{Buckley2015}. At 26$^\circ$N the AMOC transports around 1.25 PW of heat northward \citep{Bryden2020}, which accounts for around 25\% of the meridional transport of heat by the global atmosphere and ocean at the latitude \citep{Srokosz2012}. The AMOC's contribution to poleward heat transport warms Western Europe by around 5$^\circ$C \citep{Jackson2015}.

The AMOC is primarily driven by buoyancy forcing, wind forcing, and diapycnal mixing \citep[e.g.][]{Johnson2019, Munk1998}. In the high-latitude North Atlantic \& polar seas, strong buoyancy forcing produces dense North Atlantic Deep Waters, which sink and flow southwards. These waters are primarily formed in the Irminger and Iceland basins, and the Nordic Seas, although a small amount of formation occurs in the Labrador Sea too \citep{Lozier2019}. This formation of deep waters in the North Atlantic is balanced by diapycnal upwelling as waters flow southwards \citep{Cimoli2022,Ferrari2016,Mashayek2017,McDougall2017}, and wind-driven Ekman upwelling in the Southern Ocean \citep{Gnanadesikan1999, Marshall2012}. The circulation is closed by net northward transport of waters in the Atlantic's upper 1,000 m.

The AMOC can be represented by a zonally integrated meridional overturning stream function in density space, defined as
\begin{equation}
    \psi(y, \rho) = \int_{x' = x_w}^{x_e} \int_{z' = -H}^{z(x', y, \rho)} V(x', y, z') \dd{z'} \dd{x'} \, ,
\end{equation}
where $x$ is longitude, $x_e$ and $x_w$ are the Eastern and Western boundaries, $y$ is latitude, $z$ is the vertical coordinate, $H$ is depth, $\rho$ is density and $V$ is meridional velocity. This naturally leads to a conceptual framework in which the AMOC is viewed as a meridionally coherent and zonally uniform ``global conveyor''. In recent years, it has become apparent that although this framework can be useful for understanding the zonally integrated overturning cell, its dynamics \citep{Weijer2019}, and its impacts \citep{Zhang2019, Little2019}, the framework has its limitations. In particular, this world view offers little insight when considering the smaller scale, non-uniform dynamics which generate the diapycnal mixing which is so important in maintaining the circulation, and the mesoscale and sub-mesoscale dynamics which can strongly influence the variability of the currents which contribute to the AMOC \citep[e.g.][]{Thomas2013}. It is now increasingly common to think of the AMOC in relation to its many constituent currents and the gyre circulations which contribute to the net-overturning~\citep{Bower2019}.

Sub-mesoscale motions are characterised by a Rossby number and a Froude number of order one --- that is that rotation, inertia and stratification are of similar importance \citep{Gula2022, McWilliams2016, Thomas2008}. The sub-mesoscale dynamics of the ocean have only recently started to be explored due to the small length scales (around 100 m to 10 km) and time scales (from hours to weeks) over which they operate making observations difficult and modelling computationally expensive \citep{Thomas2008}. It has, however, become abundantly clear that sub-mesoscale flows and instabilities may be key in generating diapycnal mixing \citep{Gula2022}, and in cascading energy towards dissipative scales \citep{Muller2005, Callies2013}. The hyper-local nature of the sub-mesoscale means that we are still building a picture as to where, when and how much mixing these currents produce, and their impact on larger-scale circulations. Understanding the spatial and temporal patterns of mixing generated by the sub-mesoscale is key to understanding how waters are transformed from dense to light and vice versa in the AMOC.

This thesis will investigate the role of symmetric instability (a sub-mesoscale instability that generates overturning cells) in producing mixing --- be this of buoyancy, momentum, potential vorticity or other tracers of interest --- in several western boundary current systems which contribute to the AMOC. Buoyancy is defined mathematically as
\begin{equation}
    b = -\frac{g}{\rho_{ref}}(\rho - \rho_{ref})
\end{equation}
where $g$ is the Earth's gravitational acceleration and $\rho_{ref}$ is a reference density. Buoyancy and momentum can be combined to form the composite scalar known as Ertel potential vorticity, which is given by
\begin{equation}
    \label{eq:EPV}
    Q = (\mathbf{f} + \curl{\mathbf{u}}) \cdot \grad{b}
\end{equation}
where $\mathbf{f}$ is the planetary vorticity vector and $\mathbf{u}$ is the velocity field in the rotating frame of reference. One can interpret the potential vorticity as either the component of absolute vorticity normal to isopycnal surfaces and scaled by the isopycnal thickness, or the stratification along a vortex tube and scaled by the absolute vorticity of the vortex, with the former being the most common way of understanding the quantity.

In a fluid subjected to no momentum or buoyancy forcing, potential vorticity is materially conserved, i.e.
\begin{equation}
    \label{eq:PVConservation}
    \frac{DQ}{Dt} = 0 \, .
\end{equation}
This means if we follow an individual fluid parcel around we expect its potential vorticity to remain constant. This is a fundamental material invariant which arises from a particle relabelling symmetry on isopycnal surfaces in an inviscid fluid~\citep{Salmon1998}.

If a fluid is subject to external or dissipative momentum and buoyancy forcing the conservation of potential vorticity  becomes
\begin{equation}
    \frac{DQ}{Dt} = \frac{D\mathbf{\zeta}}{Dt} \cdot  \grad{b} + \mathbf{\zeta} \cdot \frac{D\grad{b}}{Dt} \, .
\end{equation}

Symmetric instability is intimately linked to potential vorticity: a necessary but not sufficient condition for the excitement of symmetric instability is that the vertical component of planetary vorticity and the potential vorticity of a fluid have opposite signs. Mathematically we require that
\begin{equation}
    \label{eq:PVSI1}
    f Q < 0 \, ,
\end{equation}
where $f$ is the vertical component of the planetary vorticity vector, also known as the Coriolis parameter. This criterion is derived in section~\ref{subsec:InstabilityCriterion}. Waters which satisfy this instability criterion are often described as having anomalous potential vorticity.

The overturning cells generated during the excitement of symmetric instability occur in a plane perpendicular to the mean flow and oriented so that they are parallel to isopycnal surfaces --- as such symmetric instability is often said to produce slantwise convection. The overturning motions cause the mixing of waters with anomalous potential vorticity, reconfiguring the momentum and stratification of the fluid to produce waters with potential vorticity of approximately zero. This corresponds to the absolute vorticity vector being parallel to isopycnal surfaces. Secondary shear instabilities further enhance this mixing.

The role of symmetric instability in mixing anomalous potential vorticity in the North Brazil Current and the Deep Western Boundary Current systems of the Tropical Atlantic is hitherto unclear (section~\ref{sec:CrossEqSI} describes the source of this anomalous potential vorticity). Presently, no theories satisfactorily answer the question of how these currents can rid themselves of anomalous potential vorticity. \citet{Edwards1998I, Edwards1998II} suggest that potential vorticity modification occurs in frictional boundary layers, however, we know that this can only be the case in models which use eddy viscosities several orders of magnitude higher than that of seawater and hence exaggerate the role of friction. We suggest as an alternate hypothesis that the excitement of symmetric instability generates sufficient mixing of potential vorticity to rid the currents of any anomalous potential vorticity they may have. The idea that symmetric instability may be present in flows such as these, has not previously been seriously entertained due to preconceptions about the smallness of the growth rate of symmetric instability in the vicinity of the equator. This thesis will challenge these preconceptions and demonstrate that they are misplaced.

Observations from the Sub-polar North Atlantic suggest that symmetric instability may also be at play in the East Greenland Current during wintertime wind events~\citep{LeBras2022}. These observations raise questions regarding the amount of diapycnal mixing produced by symmetric instability (and secondary instabilities), and the role of this mixing in pre-conditioning the mixed layer such that conditions are conducive to the further formation of dense waters. Using highly idealised high-resolution models this thesis will begin to answer these questions.

That symmetric instability is present in the North Brazil Current, Deep Western Boundary Current, and East Greenland Current systems is significant as it highlights regions where coarse-grained ocean models and climate models may not accurately represent the mixing that is occurring in the ocean. We also contribute to theories on the spatio-temporal distribution of sub-mesoscale processes in the ocean, from which an increased understanding of the spatio-temporal patterns of mixing can be inferred. Given the importance of mixing within the AMOC, perhaps it is appropriate to think of the circulation as a \textit{global mixer} connected by along-isopycnal transports, as well as a \textit{global conveyor} connected by mixing?

In chapter~\ref{chap:2} we will take a closer look at symmetric instability and its ``character''. We will state several properties of the instability and go on to prove them using linear stability theory. Next, we will apply this theory to the North Brazil Current and its rings, the Deep Western Boundary Current in the Tropical Atlantic, and the East Greenland Current, in order to make predictions about the symmetric instabilities we may expect to see in these currents.

We will then move away from theory and towards using idealised numerical models to address the key aim of this work --- the effect of symmetric instability in generating mixing in components of the AMOC. In chapter~\ref{chap:3} we will focus on the North Brazil Current, and use a hierarchy of models to demonstrate the presence of symmetric instability. We will show that the excitement of the instability produces sufficient mixing to neutralise ``wrong-signed'' potential vorticity. We will then quantify the water mass formation and transformation rates in the model and examine the limitations of these calculations.

Next, we will look at the generation of symmetric instability in the Deep Western Boundary Current of the Tropical Atlantic (chapter~\ref{chap:4}). Symmetric instability here is generated by a similar mechanism to that which causes instability in the North Brazil Current. We show that, again, symmetric instability is excited and examine the differences in location and strength to those seen in the North Brazil Current, explaining the different spatio-temporal characteristics of the instability. We also show that symmetric instability is capable of generating so-called ``density staircases'' --- regions of low stratification separated by high stratification filaments. These staircases have secondary impacts on diapycnal mixing, although we do not attempt to quantify these effects.

We will then head to the Sub-polar North Atlantic where we will look at how wind-forcing in the Irminger Current can lead to the excitement of symmetric instability (chapter~\ref{chap:5}). We use an ensemble of large eddy simulations to investigate how the rates of water mass formation vary with the strength and duration of the wind events. The effect of these events on the mixed layer depth is examined, a set of calculations which could form the foundations of a simple parameterisation of symmetric instability in the region.
