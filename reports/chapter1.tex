\chapter{The global mixer}
\label{chap:1}

\begin{quote}
\textit{The shape of the sea is always the same. Or rather it's always different, but in the same way.}
\newline Philip Marsden, the Silent Landscape
\end{quote}

The Atlantic Meridional Overturning Circulation (AMOC) is a climatically important circulation that spans the entirety of the Atlantic Ocean basin. In its surface limb, it draws water from the Southern Hemisphere northward across the equator, redistributing climatically important tracers such as heat and carbon as it does so \citep{Buckley2015}. At 26$^\circ$N the AMOC transports around 1.25 PW of heat northward \citep{Bryden2020}, which accounts for around 25\% of the meridional transport of heat by the global atmosphere and ocean at this latitude \citep{Srokosz2012}. The AMOC's contribution to poleward heat transport warms North West Europe by around 5$^\circ$C \citep{Jackson2015}.

The AMOC is driven by buoyancy forcing, wind forcing, and diapycnal mixing \citep[e.g.][]{Munk1998, Johnson2019}.
In the high-latitude North Atlantic \& polar seas, strong buoyancy forcing produces dense North Atlantic Deep Waters, which sink and flow southwards \citep[e.g.][]{Marshall1999}. These waters are primarily formed in the Irminger and Iceland basins, and the Nordic Seas, although a small amount of formation occurs in the Labrador Sea too \citep{Lozier2019}. The formation of deep waters in the North Atlantic is balanced by diapycnal upwelling as waters flow southwards \citep{Munk1966, Ferrari2016, Mashayek2017, McDougall2017, Callies2018, Cimoli2022}, and wind-driven Ekman upwelling in the Southern Ocean \citep{Gnanadesikan1999, Marshall2012}. The circulation is closed by a net northward transport of waters in the Atlantic's upper 1,000 m.

%TODO: Could we add a schematic of the AMOC (including highlighting the NBC, rings and EGS). Also the MOC streamfunction in ECCO for instance?

The AMOC can be represented by a zonally integrated meridional overturning stream function in density space, defined as
\begin{equation}
    \psi(y, \rho) = \int_{x' = x_w}^{x_e} \int_{z' = -H}^{z(x', y, \rho)} V(x', y, z') \dd{z'} \dd{x'} \, ,
\end{equation}
where $x$ is the longitude, $x_e$ and $x_w$ are the Eastern and Western boundaries, $y$ is the latitude, $z$ is the vertical coordinate, $H$ is the depth, $\rho$ is the density and $V$ is the meridional velocity. This naturally leads to a conceptual framework in which the AMOC is viewed as a meridionally coherent and zonally uniform ``global conveyor''. In recent years, it has become apparent that although this framework can be useful for understanding the zonally integrated overturning cell, its dynamics \citep{Weijer2019}, and its impacts \citep{Little2019, Zhang2019}, the framework has its limitations. In particular, this world view offers little insight when considering the smaller scale, non-uniform dynamics which generate the diapycnal mixing which is so important in maintaining the circulation, and the mesoscale and submesoscale dynamics which can strongly influence the variability of the currents which contribute to the AMOC \citep[e.g.][]{Thomas2013}. It is now increasingly common to think of the AMOC in relation to its many constituent currents and the gyre circulations which contribute to the net-overturning~\citep{Bower2019}.

Submesoscale motions are characterised by a Rossby number and a Froude number of order one --- i.e. rotation, inertia and stratification are of similar importance \citep{Thomas2008, McWilliams2016, Gula2022}. The submesoscale dynamics of the ocean have only recently started to be explored due to the small length scales (around 100 m to 10 km) and time scales (from hours to weeks) over which they operate making observations difficult and modelling computationally expensive \citep{Thomas2008}. It has, however, become abundantly clear that submesoscale flows and instabilities may be key in generating diapycnal mixing \citep{Gula2022}, and in cascading energy towards dissipative scales \citep{Muller2005, Callies2013, Yu2019, Evans2020}. The hyper-local nature of the submesoscale means that we are still building a picture of the spatiotemporal patterns and mechanisms of mixing generated by currents at this scale and their impact on larger-scale circulations. Understanding these mixing patterns is key to understanding how waters are transformed in the constituent currents of the AMOC \citep{Groeskamp2019}.

This thesis will investigate the role of symmetric instability, a submesoscale instability that generates overturning cells, in producing mixing --- be this of buoyancy, momentum, potential vorticity or other tracers of interest --- in several western boundary current systems which contribute to the AMOC.

Buoyancy is defined as
\begin{equation}
    b = -\frac{g}{\rho_{0}}(\rho - \rho_{0})
\end{equation}
where $g$ is the Earth's gravitational acceleration and $\rho_{0}$ is a reference density. Buoyancy and momentum can be combined to form the composite scalar known as Ertel potential vorticity, which is given by
\begin{equation}
    \label{eq:EPV}
    Q = (\mathbf{f} + \curl{\mathbf{u}}) \cdot \grad{b}
\end{equation}
where $\mathbf{f}$ is the planetary vorticity vector and $\mathbf{u}$ is the velocity field in the rotating frame of reference. Potential vorticity is materially conserved in the absence of diabatic forcing \citep{Ertel1942}. One can interpret the potential vorticity as either the component of absolute vorticity normal to isopycnal surfaces and scaled by the isopycnal thickness, or the stratification along a vortex tube and scaled by the absolute vorticity of the vortex, with the former being the most common way of understanding the quantity.

Symmetric instability is intimately linked to potential vorticity: a necessary but not sufficient condition for the excitement of symmetric instability is that the vertical component of planetary vorticity and the potential vorticity of a fluid have opposite signs. Mathematically we require that
\begin{equation}
    \label{eq:PVSI1}
    f Q < 0 \, ,
\end{equation}
where $f$ is the vertical component of the planetary vorticity vector, also known as the Coriolis parameter\footnotemark.
\footnotetext{This criterion is derived in section~\ref{subsec:InstabilityCriterion}.}
Waters which satisfy this instability criterion are often described as having anomalous potential vorticity.


The overturning cells generated during the excitement of symmetric instability occur in a plane perpendicular to the mean flow and are oriented such that they are parallel to isopycnal surfaces --- symmetric instability is often said to produce slantwise convection because of this \citep[e.g.][]{emanuelSlantwiseConvection1994}. The overturning motions cause the mixing of waters with anomalous potential vorticity, moving gradients of momentum and buoyancy towards dissipative scales, and producing waters with potential vorticity of approximately zero \citep{Taylor2009}. This corresponds to the absolute vorticity vector being parallel to isopycnal surfaces.

% TODO: Add some more context about the NBC and its rings in the below paragraph. Possibly split it up into two different paragraphs.

The North Brazil Current and its rings, and the Deep Western Boundary Current are two of the most important cross-equatorial pathways of the AMOC, accounting for around 70\% of the AMOC's surface \citep{Fratantoni2000} and $\flatfrac{2}{3}$rds of its deep \citep{Richardson1999, Bower2019} interhemispheric water exchange. The North Brazil Current is an intense northward flowing western boundary current, and its rings are large anticyclonic eddies spun up as a result of barotropic instability \citep{Johns1998,Castelao2011}. The Deep Western Boundary Current is a much slower and deeper boundary current which flows southwards \citep{Schott2005}. The role of symmetric instability in mixing anomalous potential vorticity in the North Brazil Current and the Deep Western Boundary Current systems of the Tropical Atlantic is hitherto unclear; however, we have reason to suspect it may play an important role in the region. 

Away from the tropics, the planetary contribution to potential vorticity typically dominates, meaning that large-scale currents tend to be stable to symmetric instability --- i.e. $fQ \geq 0$. Due to the material conservation of potential vorticity, we would expect waters in the North Brazil Current to have negative potential vorticity and in the Deep Western Boundary Current to have positive potential vorticity, since they originate away from the equator in regions where planetary vorticity will dominate the potential vorticity balance. As these currents advect waters equatorward, we would expect their potential vorticity to remain the same. At the equator itself, planetary vorticity changes sign, meaning that upon crossing the equator, the potential vorticity in these currents renders them unstable to symmetric instability. \citet{Edwards1998I, Edwards1998II} suggest that the modification of this anomalous potential vorticity occurs in frictional boundary layers as the currents flow poleward; however, we know that this can only be the case in models, as they employ eddy viscosities several orders of magnitude higher than that of seawater and hence exaggerate the role of friction \citep{Akuetevi2015}.

We suggest an alternative hypothesis: the excitement of symmetric instability generates sufficient mixing of potential vorticity to rid cross-equatorial oceanic currents of their anomalous potential vorticity. A similar hypothesis has been made for western boundary currents in the Martian atmosphere \citep{Joshi1994}, whereas in the Earth's atmosphere it is thought that friction and diabatic heating are able to dissipate anomalous potential vorticity \citep{Rodwell1995}. The idea that symmetric instability may be present in oceanic cross-equatorial flows has not previously been entertained due to preconceptions about the smallness of the growth rate of symmetric instability in the vicinity of the equator \citep{Edwards1998I, Haine1998}. This thesis will challenge these preconceptions and demonstrate that they are misplaced.

% TODO: Descibe the EGC in more detail.

The Sub-polar North Atlantic is another region of fundamental importance to the AMOC. It is here that North Atlantic Deep Waters are formed as a result of buoyancy loss \citep[e.g.]{Marshall1999, Pickart2003, DeJong2016, Lozier2019}. Observations from the Sub-polar North Atlantic suggest that symmetric instability may also be at play in the East Greenland Current during strong wind events that occur in Winter and Spring time~\citep{LeBras2022}. These observations raise questions regarding the amount of diapycnal mixing produced by symmetric instability (and secondary instabilities), and the role of this mixing in pre-conditioning the mixed layer such that conditions are conducive to the further formation of dense waters. This thesis will begin to answer some of the questions that observations in the region are not able to answer alone.

Demonstrating the presence of symmetric instability in the North Brazil Current, Deep Western Boundary Current, and East Greenland Current systems is significant as it highlights regions where coarse-resolution ocean models and climate models may not accurately represent the mixing that is occurring in the ocean. As we have just seen, such small-scale processes are key mediators of the AMOC's dynamics, so it's clear that either resolving or parameterising them accurately is key to accurately representing the AMOC in models. Given the importance of mixing within the AMOC, perhaps it is appropriate to think of the circulation as a \textit{global mixer} connected by along-isopycnal transports, as well as a \textit{global conveyor} connected by mixing?

In summary, this thesis aims to answer:
\begin{itemize}
    \item What role does symmetric instability play in the dynamics of the AMOC and its constituent currents?
    \item How is potential vorticity transformed in cross-equatorial western boundary currents?
\end{itemize}
The former question is broad. We will focus in particular on the aforementioned North Brazil Current, Deep Western Boundary Current and East Greenland Current systems, looking at how symmetric instability alters their potential vorticity dynamics and stratification, and its role in the formation of different water masses. The latter question is much more specific and technically encompassed by the former; however, it is a long-standing question of fundamental importance, hence we have stated it explicitly here. Without adequate potential vorticity modification in the Tropical Atlantic's western boundary current systems, large-scale interhemispheric water exchange is not possible meaning any meridional overturning cells would be confined to a single hemisphere \citep{Csanady1985,  Nof1990, Killworth1991, Johnson1993}. As such explaining the potential vorticity modification mechanism is a foundational problem of AMOC science.

In chapter~\ref{chap:2} we will take a closer look at symmetric instability and its ``character''. We will state several properties of the instability and go on to prove them using linear stability theory. In the latter half of the chapter, we will apply this theory to the North Brazil Current and its rings, the Deep Western Boundary Current in the Tropical Atlantic, and the East Greenland Current, in order to make predictions about the symmetric instabilities we may expect to see in them.

We will then move away from theory and towards using idealised numerical models to investigate the effect of symmetric instability in generating mixing in components of the AMOC. In chapter~\ref{chap:3} we will focus on the North Brazil Current, and use a hierarchy of models to demonstrate the presence of symmetric instability. We will show that the excitement of the instability produces sufficient mixing to neutralise ``wrong-signed'' potential vorticity, which begins to address the second of our aims. We will then quantify the water mass formation and transformation rates in the model and examine the limitations of these calculations.

Next, we will look at the generation of symmetric instability in the Deep Western Boundary Current of the Tropical Atlantic (chapter~\ref{chap:4}). Symmetric instability here is generated by a similar mechanism to that which causes instability in the North Brazil Current. We will show that, again, symmetric instability is excited and we will examine the differences in the spatiotemporal characteristics of the instability relative to the North Brazil Current. We will also show that symmetric instability is capable of generating so-called ``density staircases'' --- regions of low stratification separated by high stratification filaments. These staircases have secondary impacts on diapycnal mixing, although we will not attempt to quantify these effects here \citep{Schmitt2005}.

We will then head to the Sub-polar North Atlantic where we will look at how wind-forcing over the East Greenland Current can lead to the excitement of symmetric instability (chapter~\ref{chap:5}). We will use an ensemble of large eddy simulations to investigate how the rates of water mass formation vary with the strength and duration of the wind events. We also examine the effect of these events on the mixed layer depth. Calculations such as these could form the foundations of a simple parameterisation of symmetric instability in the region.

Concluding remarks are made, and the main findings of the thesis summarised in chapter~\ref{chap:Conclusions}.
