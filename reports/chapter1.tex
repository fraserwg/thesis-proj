\chapter{The Atlantic Meridional Overturning Circulation \& the importance of mixing}
\begin{quote}
\textit{The shape of the sea is always the same. Or rather it's always different, but in the same way.} --- Philip Marsden
\end{quote}

The Atlantic Meridional Overturning Circulation (AMOC) is a climatically important circulation that spans the entirety of the Atlantic Ocean basin. In its surface limb, it draws water from the Southern Hemisphere northward across the equator, redistributing climatically import tracers such as heat and carbon as it does so \citep{Buckley2015}. At 26$^\circ$N the AMOC transports around 1.25 PW of heat northward \citep{Bryden2020}, which accounts for around 25\% of the meridional transport of heat by the global atmosphere and ocean at the latitude \citep{Srokosz2012}. The AMOC's contribution to poleward heat transport warms Western Europe by around 5$^\circ$C \citep{Jackson2015}.

The AMOC is primarily driven by buoyancy and wind forcing, and diapycnal mixing \citep{Johnson2019}. In the Arctic, strong buoyancy forcing produces dense North Atlantic Deep Waters, which sink and flows southwards. These waters are primarily formed in the Irminger and Iceland basins, although a small amount of formation occurs in the Labrador Sea too \citep{Lozier2019}. This formation of deep waters in the North Atlantic is balanced by diapycnal upwelling as waters flow southwards \citep{Cimoli2022,Ferrari2016,Mashayek2017,McDougall2017}, and wind-driven Ekman upwelling in the Southern Ocean \citep{Gnanadesikan1999, Marshall2012}. The circulation is closed by net northward transport of waters in the Atlantic's surface.

The AMOC can be represented by a zonally integrated meridional overturning stream-function in density space, defined as
\begin{equation}
    \psi(y, \rho) = \int_{x' = x_w}^{x_e} \int_{z' = -H}^{z(x', y, \rho)} V(x', y, z') \dd{z'} \dd{x'} \, ,
\end{equation}
where $x$ is longitude, $x_e$ and $x_w$ are the Eastern and Western boundaries, $y$ is latitude, $z$ is the vertical coordinate, $H$ is depth, $\rho$ is density and $V$ is meridional velocity. This naturally leads to a conceptual framework in which the AMOC is viewed as a meridionally coherent and zonally uniform ``global conveyor''. In recent years, it has become apparent that although this framework can be useful for understanding the zonally integrated overturning cell, its dynamics \citep{Weijer2019}, and its impacts \citep{Zhang2019, Little2019}, the framework has its limitations. In particular this world view offers little insight when considering the smaller scale, non-uniform dynamics which generate the diapycnal mixing which is so important in maintaining the circulation, and the mesoscale and sub-mesoscale dynamics which can strongly influence the variability of the currents which contribute to the AMOC \citep[e.g.][]{Thomas2013}. It is now increasingly common to think of the AMOC in relation to its many constituent currents and the gyre circulations which contribute to the net-overturning~\citep{Bower2019}.

Sub-mesoscale motions are characterised by a Rossby number and a Froude number of order one --- the is that rotation, inertia and stratification are of similar importance \citep{Gula2022, McWilliams2016, Thomas2008}. The sub-mesoscale dynamics of the Ocean have only recently started to be explored due to the small length scales (around 100 m to 10 km) and time scales (from hours to weeks) over which they operate making observations difficult, and modelling computationally expensive \citep{Thomas2008}. It has, however, become abundantly clear that sub-mesoscale flows and instabilities may be key in generating diapycnal mixing \citep{Gula2022}, and in cascading energy towards dissipative scales \citep{Muller2005, Callies2013}. The hyper-local nature of the sub-mesoscale means that we are still building a picture as to where, when and how much mixing these currents produce, and their impact on larger scale circulations. Understanding the spatial and temporal patterns of mixing generated by the sub-mesoscale is key to understanding how waters are transformed from dense to light and vice versa in the AMOC.

This work will investigate the role of symmetric instability (a sub-mesoscale instability that generates overturning cells) in producing mixing in several western boundary current systems which contribute to the AMOC. In particular, we will examine the North Brazil Current and Deep Western Boundary Current systems of the Tropical Atlantic, and the Irminger Current of the Sub-polar North Atlantic. We will use tools ranging from analytical theory to idealised numerical models in exploring the role symmetric instability may play in each of these current systems.

That symmetric instability is present in these current systems is significant as it highlights regions where coarse-grained ocean models and climate models may not accurately represent the mixing that is occurring in the Ocean. We also contribute to theories on the spatio-temporal distribution of sub-mesoscale processes in the ocean, from which an increased understanding of the spatio-temporal patterns of mixing can be inferred.

In chapter~\ref{chap:2} we will take a closer look at symmetric instability and its ``character''. We will state several properties of the instability and go onto prove them using linear stability theory. Next we will apply this theory to the North Brazil Current and its rings, the Deep Western Boundary Current in the Tropical Atlantic, and the Irminger Current, in order to make predictions about the symmetric instabilities we may expect to see in these currents. In the subsequent chapters we will move away from theory and towards verifying the presence of symmetric instability and quantifying the mixing it induces in a series of idealised models.

We will then move away from theory and towards using idealised numerical models to address the key aim of this work --- the effect of symmetric instability in generating diapycnal mixing in components of the AMOC. In chapter~\ref{chap:3} we will focus on the North Brazil Current, and use a hierarchy of models to demonstrate the presence of symmetric instability. We then quantity the water mass formation and transformation rates in the model and examine the limitations of these calculations. We also look at the role of mixing generated by symmetric instability in modifying potential vorticity in the current.

Next we look at the generation of symmetric instability in the Deep Western Boundary Current of the Tropical Atlantic (chapter~\ref{chap:4}). Symmetric instability here is generated by a similar mechanism to that which causes instability in the North Brazil Current. We show that, again, symmetric instability is excited and examine the differences in location and strength to those seen in the North Brazil Current, explaining the different spatio-temporal characteristics of the instability. We also show that symmetric instability is capable of generating so-called ``density staircases'' --- regions of low stratification separated by high stratification filaments. These staircases have secondary impacts on diapycnal mixing, although we do not attempt to quantify these effects.

We will then head to the Sub-polar North Atlantic where we will look at how wind-forcing in the Irminger Current can lead to the excitement of symmetric instability (chapter~\ref{chap:5}). We use an ensemble of large eddy simulations to investigate how the rates of water mass formation vary with the strength and duration of the wind events. The effect of these events on the mixed layer depth is examined, a set of calculations which could form the foundations of a simple parameterisation of symmetric instability in the region.
